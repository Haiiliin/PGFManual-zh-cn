% Copyright 2006 by Till Tantau
%
% This file may be distributed and/or modified
%
% 1. under the LaTeX Project Public License and/or
% 2. under the GNU Free Documentation License.
%
% See the file doc/generic/pgf/licenses/LICENSE for more details.


\section{The Soft Path Subsystem}
\label{section-soft-paths}

\makeatletter

This section describes a set of commands for creating \emph{soft paths} as
opposed to the commands of the previous section, which created \emph{hard
paths}. A soft path is a path that can still be ``changed'' or ``molded''. Once
you (or the \pgfname\ system) is satisfied with a soft path, it is turned into
a hard path, which can be inserted into the resulting |.pdf| or |.ps| file.

Note that the commands described in this section are ``high-level'' in the
sense that they are not implemented in driver files, but rather directly by the
\pgfname-system layer. For this reason, the commands for creating soft paths do
not start with |\pgfsys@|, but rather with |\pgfsyssoftpath@|. On the other
hand, as a user you will never use these commands directly, they are described
as part of the low-level interface.


\subsection{Path Creation Process}

When the user writes a command like |\draw (0bp,0bp) -- (10bp,0bp);| quite a
lot happens behind the scenes:
%
\begin{enumerate}
    \item The frontend command is translated by \tikzname\ into commands of the
        basic layer. In essence, the command is translated to something like
        %
\begin{codeexample}[code only]
\pgfpathmoveto{\pgfpoint{0bp}{0bp}}
\pgfpathlineto{\pgfpoint{10bp}{0bp}}
\pgfusepath{stroke}
\end{codeexample}
        %
    \item The |\pgfpathxxxx| commands do \emph{not} directly call ``hard''
        commands like |\pgfsys@xxxx|. Instead, the command |\pgfpathmoveto|
        invokes a special command called |\pgfsyssoftpath@moveto| and
        |\pgfpathlineto| invokes |\pgfsyssoftpath@lineto|.

        The |\pgfsyssoftpath@xxxx| commands, which are described below,
        construct a soft path. Each time such a command is used, special tokens
        are added to the end of an internal macro that stores the soft path
        currently being constructed.
    \item When the |\pgfusepath| is encountered, the soft path stored in the
        internal macro is ``invoked''. Only now does a special macro iterate
        over the soft path. For each line-to or move-to operation on this path
        it calls an appropriate |\pgfsys@moveto| or |\pgfsys@lineto| in order
        to, finally, create the desired hard path, namely, the string of
        literals in the |.pdf| or |.ps| file.
    \item After the path has been invoked, |\pgfsys@stroke| is called to insert
        the literal for stroking the path.
\end{enumerate}

Why such a complicated process? Why not have |\pgfpathlineto| directly call
|\pgfsys@lineto| and be done with it? There are two reasons:
%
\begin{enumerate}
    \item The \pdf\ specification requires that a path is not interrupted by
        any non-path-construction commands. Thus, the following code will
        result in a corrupted |.pdf|:
        %
\begin{codeexample}[code only]
\pgfsys@moveto{0}{0}
\pgfsys@setlinewidth{1}
\pgfsys@lineto{10}{0}
\pgfsys@stroke
\end{codeexample}
        %
        Such corrupt code is \emph{tolerated} by most viewers, but not always.
        It is much better to create only (reasonably) legal code.
    \item A soft path can still be changed, while a hard path is fixed. For
        example, one can still change the starting and end points of a soft
        path or do optimizations on it. Such transformations are not possible
        on hard paths.
\end{enumerate}


\subsection{Starting and Ending a Soft Path}

No special action must be taken in order to start the creation of a soft path.
Rather, each time a command like |\pgfsyssoftpath@lineto| is called, a special
token is added to the (global) current soft path being constructed.

However, you can access and change the current soft path. In this way, it is
possible to store a soft path, to manipulate it, or to invoke it.

\begin{command}{\pgfsyssoftpath@getcurrentpath\marg{macro name}}
    This command will store the current soft path in \meta{macro name}.
\end{command}

\begin{command}{\pgfsyssoftpath@setcurrentpath\marg{macro name}}
    This command will set the current soft path to be the path stored in
    \meta{macro name}. This macro should store a path that has previously been
    extracted using the |\pgfsyssoftpath@getcurrentpath| command and has
    possibly been modified subsequently.
\end{command}

\begin{command}{\pgfsyssoftpath@invokecurrentpath}
    This command will turn the current soft path in a ``hard'' path. To do so,
    it iterates over the soft path and calls an appropriate |\pgfsys@xxxx|
    command for each element of the path. Note that the current soft path is
    \emph{not changed} by this command. Thus, in order to start a new soft path
    after the old one has been invoked and is no longer needed, you need to set
    the current soft path to be empty. This may seem strange, but it is often
    useful to immediately use the last soft path again.
\end{command}

\begin{command}{\pgfsyssoftpath@flushcurrentpath}
    This command will invoke the current soft path and then set it to be empty.
\end{command}


\subsection{Soft Path Creation Commands}

\begin{command}{\pgfsyssoftpath@moveto\marg{x}\marg{y}}
    This command appends a ``move-to'' segment to the current soft path. The
    coordinates \meta{x} and \meta{y} are given as normal \TeX\ dimensions.

    \example One way to draw a line:
    %
\begin{codeexample}[code only]
\pgfsyssoftpath@moveto{0pt}{0pt}
\pgfsyssoftpath@lineto{10pt}{10pt}
\pgfsyssoftpath@flushcurrentpath
\pgfsys@stroke
\end{codeexample}
    %
\end{command}

\begin{command}{\pgfsyssoftpath@lineto\marg{x}\marg{y}}
    Appends a ``line-to'' segment to the current soft path.
\end{command}

\begin{command}{\pgfsyssoftpath@curveto\marg{a}\marg{b}\marg{c}\marg{d}\marg{x}\marg{y}}
    Appends a ``curve-to'' segment to the current soft path with controls
    $(a,b)$ and $(c,d)$.
\end{command}

\begin{command}{\pgfsyssoftpath@rect\marg{lower left x}\marg{lower left y}\marg{width}\marg{height}}
    Appends a rectangle segment to the current soft path.
\end{command}

\begin{command}{\pgfsyssoftpath@closepath}
    Appends a ``close-path'' segment to the current soft path.
\end{command}


\subsection{The Soft Path Data Structure}

A soft path is stored in a standardized way, which makes it possible to modify
it before it becomes ``hard''. Basically, a soft path is a long sequence of
triples. Each triple starts with a \emph{token} that identifies what is going
on. This token is followed by two dimensions in braces. For example, the
following is a soft path that means ``the path starts at $(0\mathrm{bp},
0\mathrm{bp})$ and then continues in a straight line to $(10\mathrm{bp},
0\mathrm{bp})$''.
%
\begin{codeexample}[code only]
\pgfsyssoftpath@movetotoken{0bp}{0bp}\pgfsyssoftpath@linetotoken{10bp}{0bp}
\end{codeexample}

A curve-to is hard to express in this way since we need six numbers to express
it, not two. For this reasons, a curve-to is expressed using three triples as
follows: The command
%
\begin{codeexample}[code only]
\pgfsyssoftpath@curveto{1bp}{2bp}{3bp}{4bp}{5bp}{6bp}
\end{codeexample}
%
\noindent results in the following three triples:
%
\begin{codeexample}[code only]
\pgfsyssoftpath@curvetosupportatoken{1bp}{2bp}
\pgfsyssoftpath@curvetosupportbtoken{3bp}{4bp}
\pgfsyssoftpath@curvetotoken{5bp}{6bp}
\end{codeexample}

These three triples must always ``remain together''. Thus, a lonely
|supportbtoken| is forbidden.

In details, the following tokens exist:
%
\begin{itemize}
    \item \declare{|\pgfsyssoftpath@movetotoken|} indicates a move-to
        operation. The two following numbers indicate the position to which the
        current point should be moved.
    \item \declare{|\pgfsyssoftpath@linetotoken|} indicates a line-to
        operation.
    \item \declare{|\pgfsyssoftpath@curvetosupportatoken|} indicates the first
        control point of a curve-to operation. The triple must be followed by a
        |\pgfsyssoftpath@curvetosupportbtoken|.
    \item \declare{|\pgfsyssoftpath@curvetosupportbtoken|} indicates the second
        control point of a curve-to operation. The triple must be followed by a
        |\pgfsyssoftpath@curvetotoken|.
    \item \declare{|\pgfsyssoftpath@curvetotoken|} indicates the target of a
        curve-to operation.
    \item \declare{|\pgfsyssoftpath@rectcornertoken|} indicates the corner of a
        rectangle on the soft path. The triple must be followed by a
        |\pgfsyssoftpath@rectsizetoken|.
    \item \declare{|\pgfsyssoftpath@rectsizetoken|} indicates the size of a
        rectangle on the soft path.
    \item \declare{|\pgfsyssoftpath@closepath|} indicates that the subpath
        begun with the last move-to operation should be closed. The parameter
        numbers are currently not important, but if set to anything different
        from |{0pt}{0pt}|, they should be set to the coordinate of the original
        move-to operation to which the path ``returns'' now.
\end{itemize}
