% \section{Introduction}
\section{简介}

% Welcome to the documentation of \tikzname\ and the underlying \pgfname\ system. What began as a small \LaTeX\ style for creating the graphics in my (Till Tantau's) PhD thesis directly with pdf\LaTeX\ has now grown to become a full-blown graphics language with a manual of over a thousand pages. The wealth of options offered by \tikzname\ is often daunting to beginners; but fortunately this documentation comes with a number slowly-paced tutorials that will teach you almost all you should know about \tikzname\ without your having to read the rest.

欢迎使用\tikzname\ 和底层\pgfname\ 系统的文档。\tikzname\ 最初是一种小型\LaTeX\ 样式,可以直接用pdf\LaTeX\ 在我(Till Tantau)的博士论文中创建图形,现在已经发展成为一种功能强大的图形语言,其手册超过一千页。  \tikzname\ 提供的丰富选择通常令初学者望而却步; 但是幸运的是,本文档随附了一些节奏缓慢的教程,这些教程将教您几乎所有有关\tikzname\ 的知识,而无需阅读其余内容。

% I wish to start with the questions ``What is \tikzname?'' Basically, it just defines a number of \TeX\ commands that draw graphics. For example, the code |\tikz \draw (0pt,0pt) -- (20pt,6pt);| yields the line \tikz \draw (0pt,0pt) -- (20pt,6pt); and the code |\tikz \fill[orange] (1ex,1ex) circle (1ex);| yields \tikz \fill[orange] (1ex,1ex) circle (1ex);. In a sense, when you use \tikzname\ you ``program'' your graphics, just as you ``program'' your document when you use \TeX. This also explains the name: \tikzname\ is a recursive acronym in the tradition of ``\textsc{gnu}'s Not Unix'' and means ``\tikzname\ ist \emph{kein} Zeichenprogramm'', which translates to ``\tikzname\ is not a drawing program'', cautioning the reader as to what to expect. With \tikzname\ you get all the advantages of the ``\TeX-approach to typesetting'' for your graphics: quick creation of simple graphics, precise positioning, the use of macros, often superior typography. You also inherit all the disadvantages: steep learning curve, no \textsc{wysiwyg}, small changes require a long recompilation time, and the code does not really ``show'' how things will look like.

我希望首先从``什么是\tikzname ?''这个问题开始,它基本上只定义了许多绘制图形的\TeX\ 命令。 例如,代码 |\tikz\draw(0pt,0pt) -- (20pt,6pt);| 产生线条 \tikz\draw(0pt,0pt) -- (20pt,6pt);,代码 |\tikz \fill[orange] (1ex,1ex) circle (1ex);| 产生圆 \tikz \fill[orange] (1ex,1ex) circle (1ex);。 从某种意义上说,当您使用\tikzname\ 时,您将对图形进行``编程'',使用\TeX 时就像对文档进行``编程''一样。 这也解释了名称:\tikzname \ 是``\ textsc {gnu}'s Not Unix''传统的递归首字母缩写,意思是``\tikzname ist \emph {kein} Zeichenprogramm'' , 翻译为``\tikzname\ 不是一种绘图程序'',提醒读者注意什么。 使用\tikzname\ ,您可以获得图形的``\TeX 排版设置方法''的所有优势:快速创建简单图形,精确定位,宏的使用通常为了更为出色的排版。 您还会继承所有缺点:学习曲线陡峭,没有\textsc{wysiwyg},小的更改需要很长的重新编译时间,并且代码并没有真正“展示”事物的外观。

% Now that we know what \tikzname\ is, what about ``\pgfname''? As mentioned earlier, \tikzname\ started out as a project to implement \TeX\ graphics macros that can be used both with pdf\LaTeX\ and also with the classical (PostScript-based) \LaTeX. In other words, I wanted to implement a ``portable graphics format'' for \TeX\ -- hence the name \pgfname. These early macros are still around and they form the ``basic layer'' of the system described in this manual, but most of the interaction an author has theses days is with \tikzname\ -- which has become a whole language of its own.

现在我们知道\tikzname\ 是什么了,那么``\pgfname''呢? 如前所述,\tikzname\ 最初是一个实现\TeX\ 图形宏的项目,该宏既可以与pdf \LaTeX\ 一起使用,也可以与经典的(基于PostScript的)\LaTeX 一起使用。 换句话说,我想为\TeX\ 实现一种``便携式图形格式'',因此命名为\pgfname。 这些早期的宏仍然存在,它们构成了本手册中描述的系统的``基础层'',但是作者如今的大部分交互都是与\tikzname\ 交互的,它已成为其自身的完整语言。 

% \subsection{The Layers Below \tikzname}
\subsection{\tikzname 系统下的层级}

% It turns out that there are actually \emph{two} layers below \tikzname:

事实上\tikzname 系统下有2个层级:
\iffalse
\begin{description}
    \item[System layer:] This layer provides a complete abstraction of what is going on ``in the driver''. The driver is a program like |dvips| or |dvipdfm| that takes a |.dvi| file as input and generates a |.ps| or a |.pdf| file. (The |pdftex| program also counts as a driver, even though it does not take a |.dvi| file as input. Never mind.) Each driver has its own syntax for the generation of graphics, causing headaches to everyone who wants to create graphics in a portable way. \pgfname's system layer ``abstracts away'' these differences. For example, the system command |\pgfsys@lineto{10pt}{10pt}| extends the current path  to the coordinate $(10\mathrm{pt},10\mathrm{pt})$ of the current |{pgfpicture}|. Depending on whether |dvips|, |dvipdfm|, or |pdftex| is used to process the document, the system command will be converted to different |\special| commands. The system layer is as ``minimalistic'' as possible since each additional command makes it more work to port \pgfname\ to a new driver.

    As a user, you will not use the system layer directly.
    \item[Basic layer:] The basic layer provides a set of basic commands that allow you to produce complex graphics in a much easier manner than by using the system layer directly. For example, the system layer provides no commands for creating circles since circles can be composed from the more basic Bézier curves (well, almost). However, as a user you will want to have a simple command to create circles (at least I do) instead of having to write down half a page of Bézier curve support coordinates. Thus, the basic layer provides a command |\pgfpathcircle| that generates the necessary curve coordinates for you.

    The basic layer consists of a \emph{core}, which consists of several interdependent packages that can only be loaded \emph{en bloc}, and additional \emph{modules} that extend the core by more special-purpose commands like node management or a plotting interface. For instance, the \textsc{beamer} package uses only the core and not, say, the |shapes| modules.
\end{description}
\fi

\begin{description}
    \item[系统层:]该层提供了“驱动程序”中正在发生的事情的完整抽象。驱动程序是 |dvips| 或 |dvipdfm| 之类的程序,需要一个 |.dvi| 文件作为输入并生成 |.ps| 或 |.pdf| 文件。( |pdftex| 程序也算作驱动程序,即使它没有输入 |.dvi| 文件。也没关系。)每个驱动程序都有自己的语法来生成图形,这使每个想要以可移植方式创建图形的人都感到头疼。\pgfname 的系统层“抽象”了这些差异。例如,系统命令 |\pgfsys@lineto{10pt}{10pt}| 将当前路径扩展到当前 |{pgfpicture}| 的坐标$(10\mathrm{pt},10\mathrm{pt}$。使用 |dvips|,|dvipdfm| 或 |pdftex| 处理文档时,系统命令将转换为不同的 |\special| 命令。由于每个附加命令使将\pgfname\ 移植到新驱动程序的工作量更大,因此系统层应尽可能“简化”。
    作为用户,您不会直接使用系统层。
    \item[基础层:] 基础层提供了一组基本命令,这些命令使您可以比直接使用系统层更容易地生成复杂的图形。 例如,系统层不提供用于创建圆的命令,因为圆可以由更基本的贝塞尔曲线组成(当然,基本是这样)。 但是,作为用户,您将需要一个简单的命令来创建圆(至少我要这样做),而不必写下半页的贝塞尔曲线来生成坐标。 因此,基础层提供了命令 |\pgfpathcircle| 为您生成必要的曲线坐标。

    基础层由一个\emph{core}组成,该\emph{core}由几个只能相互加载的相互依赖的包\emph{en bloc}组成,另外的\emph{modules}通过诸如节点管理之类的更多专用命令扩展了核心或绘图接头。 例如,\textsc{beamer}包仅使用核心,而不使用 |shapes| 模块。
\end{description}

% In theory, \tikzname\ itself is just one of several possible ``frontends''. which are sets of commands or a special syntax that makes using the basic layer easier. A problem with directly using the basic layer is that code written for this layer is often too ``verbose''. For example, to draw a simple triangle, you may need as many as five commands when using the basic layer: One for beginning a path at the first corner of the triangle, one for extending the path to the second corner, one for going to the third, one for closing the path, and one for actually painting the triangle (as opposed to filling it). With the \tikzname\ frontend all this boils down to a single simple \textsc{metafont}-like command:

从理论上讲,\tikzname\ 本身只是几种可能的``前端''之一。 是一类命令集或特殊语法,使得使用基础层更加容易。 直接使用基础层的问题是为此编写的代码通常太``冗长''。 例如,要绘制一个简单的三角形,使用基础层时可能需要多达五个命令:一个用于在三角形的第一个角点处开始绘制路径,一个用于将路径扩展到第二个角点处,一个用于将路径扩展到第二个角点。 第三个命令用于闭合路径,另一个用于实际绘制三角形(而不是填充三角形)。 使用\tikzname\ 前端,所有这些都归结为一个简单的类似于\textsc{metafont}的命令:

\begin{verbatim}
\draw (0,0) -- (1,0) -- (1,1) -- cycle;
\end{verbatim}

% In practice, \tikzname\ is the only ``serious'' frontend for \pgfname. It gives you access to all features of \pgfname, but it is intended to be easy to use. The syntax is a mixture of \textsc{metafont} and \textsc{pstricks} and some ideas of myself. There are other frontends besides \tikzname, but they are intended more as ``technology studies'' and less as serious alternatives to \tikzname. In particular, the |pgfpict2e| frontend   reimplements the standard \LaTeX\ |{picture}|  environment and commands like |\line| or |\vector| using the \pgfname\ basic layer. This layer is not really ``necessary'' since the |pict2e.sty| package does at least as good a job at reimplementing the |{picture}| environment. Rather, the idea behind this package is to have a simple demonstration of how a frontend can be implemented.

实际上,\tikzname\ 是\pgfname 的唯一``正经''的前端。它使您可以访问\pgfname 的所有功能,但旨在使其易于使用。它的语法是\textsc{metafont}和\textsc{pstricks}以及我自己的一些想法的混合体。事实上也有\tikzname 之外还有其他前端,但它们更多地被用作``技术研究'',而不是\tikzname 的正经替代品。特别是 |pgfpict2e| 前端重新实现标准\LaTeX\ |{picture}| 环境和命令,例如 |\line| 或 |\vector| 使用\pgfname\ 基础层。由于 |pict2e.sty| 宏包在重新实现上至少与 |{picture}| 环境表现得同样出色,这一层并不是真正的“必需”的。而是,此程序包背后的想法是简单演示如何实现前端。

% Since most users will only use \tikzname\ and almost no one will use the system layer directly, this manual is mainly about \tikzname\ in the first parts; the basic layer and the system layer are explained at the end.

由于大多数用户只会使用\tikzname\ ,几乎没有人会直接去使用系统层,因此本手册的第一部分主要涉及\tikzname\ ;最后再来说明基础层和系统层。

%\subsection{Comparison with Other Graphics Packages}
\subsection{与其它图形宏包的比较}

% \tikzname\ is not the only graphics package for \TeX. In the following, I try to give a reasonably fair comparison of \tikzname\ and other packages.

\tikzname\ 并不是\TeX 的唯一图形宏包。 下面我将尝试将\tikzname\ 与其他宏包进行合理的比较。
%
\iffalse
\begin{enumerate}
    \item The standard \LaTeX\ |{picture}| environment allows you to create simple graphics, but little more. This is certainly not due to a lack of knowledge or imagination on the part of \LaTeX's designer(s). Rather, this is the price paid for the |{picture}| environment's portability: It works together with all backend drivers.
    \item The |pstricks| package is certainly powerful enough to create any conceivable kind of graphic, but it is not really portable. Most importantly, it does not work with |pdftex| nor with any other driver that produces anything but PostScript code.

    Compared to \tikzname, |pstricks| has a similar support base. There are many nice extra packages for special purpose situations that have been contributed by users over the last decade. The \tikzname\ syntax is more consistent than the |pstricks| syntax as \tikzname\ was developed ``in a more centralized manner'' and also ``with the shortcomings on |pstricks| in mind''.
    \item The |xypic| package is an older package for creating graphics. However, it is more difficult to use and to learn because the syntax and the documentation are a bit cryptic.
    \item The |dratex| package is a small graphic package for creating a graphics. Compared to the other package, including \tikzname, it is very small, which may or may not be an advantage.
    \item The |metapost| program is a powerful alternative to \tikzname. It used to be an external program, which entailed a bunch of problems, but in Lua\TeX\ it is now built in. An obstacle with |metapost| is the inclusion of labels. This is \emph{much} easier to achieve using \pgfname.
    \item The |xfig| program is an important alternative to \tikzname\ for users who do not wish to ``program'' their graphics as is necessary ith \tikzname\ and the other packages above. There is a conversion program that will convert |xfig| graphics to \tikzname.
\end{enumerate}
\fi

\begin{enumerate}
    \item 标准\LaTeX\ |{picture}| 环境允许您创建简单的图形,但仅此而已。 这当然不是由于\LaTeX 设计师缺乏知识或想象力造成的。 相反,这是为 |{picture}| 环境的可移植性付出的代价:它与所有后端驱动程序一起使用。
    \item |pstricks| 软件包肯定足够强大,可以创建任何可能的图形,但是它并不是真正可移植的。 最重要的是,它不适用于 |pdftex| 也不使用任何其他产生PostScript代码以外的驱动程序。
    与\tikzname 相比,|pstricks| 具有类似的基础支持。在过去十年中,用户为特殊用途提供了许多不错的额外程序包。\tikzname\ 语法比 |pstricks| 更一致。像\tikzname\ 这样的语法是“以更集中的方式开发的”,并且“在心里清楚 |pstricks| 的缺点”。
    \item |xypic| 宏包是用于创建图形的较旧的宏包。然而,由于语法和文档有点晦涩难懂,因此使用和学习起来更加困难。
    \item |dratex| 宏包包是用于创建图形的小型图形软件包。 与包括\tikzname 在内的其他宏包相比,它非常小,可能有优势也可能没有优势。
    \item |metapost| 程序是\tikzname 的强大替代方案。 它曾经是一个外部程序,并且带来了很多问题,但现在已内置在Lua\TeX\ 中。|metapost| 的障碍是包含标签。。 它比使用\pgfname 更容易实现。
    \item 对于不希望使用\tikzname\ 或上面的其他宏包包进行``编程''的用户,|xfig| 程序是\tikzname\ 的一种重要替代方案。 有一个转换程序可以将 |xfig| 图形转换为\tikzname\ 。
\end{enumerate}

% \subsection{Utility Packages}
\subsection{工具包}

% The \pgfname\ package comes along with a number of utility package that are not really about creating graphics and which can be used independently of \pgfname. However, they are bundled with \pgfname, partly out of convenience, partly because their functionality is closely intertwined with \pgfname. These utility packages are:

\pgfname\ 宏包附带了许多实用程序包,这些实用程序包实际上与创建图形无关,并且可以独立于\pgfname 使用。但是,它们与\pgfname 捆绑在一起,部分是出于方便,部分是因为其功能与\pgfname 紧密相关。这些实用程序包是:
%
\iffalse
\begin{enumerate}
    \item The |pgfkeys| package defines a powerful key management facility. It can be used completely independently of \pgfname.
    \item The |pgffor| package defines a useful |\foreach| statement.
    \item The |pgfcalendar| package defines macros for creating calendars. Typically, these calendars will be rendered using \pgfname's graphic engine, but you can use |pgfcalendar| also typeset calendars using normal text. The package also defines commands for ``working'' with dates.
    \item The |pgfpages| package is used to assemble several pages into a single page. It provides commands for assembling several ``virtual pages'' into a single ``physical page''. The idea is that whenever \TeX\ has a page ready for ``shipout'', |pgfpages| interrupts this shipout and instead stores the page to be shipped out in a special box. When enough ``virtual pages'' have been accumulated in this way, they are scaled down and arranged on a ``physical page'', which then \emph{really} shipped out. This mechanism allows you to create ``two page on one page'' versions of a document directly inside \LaTeX\ without the use of any external programs. However, |pgfpages| can do quite a lot more than that. You can use it to put logos and watermark on pages, print up to 16 pages on one page, add borders to pages, and more.   
\end{enumerate}
\fi

\begin{enumerate}
    \item |pgfkeys| 软件包定义了强大的密钥管理工具。它可以完全独立于\pgfname 来使用。
    \item |pgffor| 包定义了有用的 |\foreach| 声明。
    \item |pgfcalendar| 包定义用于创建日历的宏。通常,这些日历将使用\pgfname 的图形引擎呈现,但是您可以使用 |pgfcalendar| 还可以使用普通文本排版日历。该软件包还定义了用于``处理''日期的命令。
    \item |pgfpages| 包用于将多个页面组合成一个页面。它提供了用于将多个``虚拟页面''组合成一个``物理页面''的命令。这个想法是,每当\TeX\ 有一个页面准备好“发货”时,|pgfpages| 中断此发货,而是将要发货的页面存储在特殊框中。当以这种方式积累了足够的``虚拟页面''时,它们会按比例缩小并排列在``物理页面''上,然后\emph{真正}运出。这种机制允许您直接在\LaTeX\ 内部创建文档的``一页一页''文档,而无需使用任何外部程序。但是,|pgfpages| 可以做的还不止这些。您可以使用它在页面上放置徽标和水印,在一页上最多打印16页,在页面上添加边框等等。
\end{enumerate}

% \subsection{How to Read This Manual}
\subsection{如何阅读本手册}

% This manual describes both the design of \tikzname\ and its usage. The organization is very roughly according to ``user-friendliness''. The commands and subpackages that are easiest and most frequently used are described first, more low-level and esoteric features are discussed later.

本手册介绍了\tikzname\ 的设计及其用法。该组织大致是根据“用户友好性”来确定的。首先介绍最简单和最常用的命令和子包,然后再讨论更底层和深奥的功能。

% If you have not yet installed \tikzname, please read the installation first. Second, it might be a good idea to read the tutorial. Finally, you might wish to skim through the description of \tikzname. Typically, you will not need to read the sections on the basic layer. You will only need to read the part on the system layer if you intend to write your own frontend or if you wish to port \pgfname\ to a new driver.

如果尚未安装\tikzname ,请先阅读安装说明。其次,阅读教程部分可能是一个好主意。最后,您可能希望略过\tikzname 的描述。通常,您无需阅读基础层上的章节。如果您打算编写自己的前端,或者希望将\pgfname\ 移植到新驱动程序,则只需阅读系统层上的该部分。

% The ``public'' commands and environments provided by the system are described throughout the text. In each such description, the described command, environment or option is printed in red. Text shown in green is optional and can be left out.

全文中介绍了系统提供的``公共''命令和环境。在每个这样的描述中,所描述的命令,环境或选项均以红色打印。用绿色显示的文本是可选的,可以省略。

% \subsection{Authors and Acknowledgements}
\subsection{作者与致谢}
\label{section-authors}

% The bulk of the \pgfname\ system and its documentation was written by Till Tantau. A further member of the main team is Mark Wibrow, who is responsible, for example, for the \pgfname\ mathematical engine, many shapes, the decoration engine, and matrices. The third member is Christian Feuers\"anger who contributed the floating point library, image externalization, extended key processing, and automatic hyperlinks in the manual.

\pgfname\ 系统及其文档的大部分由Till Tantau编写。主团队的另一位成员是(Mark Wibrow,他负责如\pgfname\ 数学引擎,许多形状,装饰引擎和矩阵的编写。第三个成员是Christian Feuers\"anger,他在手册中贡献了浮点库,图像外部化,扩展键处理和自动超链接。

% Furthermore, occasional contributions have been made by Christophe Jorssen, Jin-Hwan Cho, Olivier Binda, Matthias Schulz, Ren\'ee Ahrens, Stephan Schuster, and Thomas Neumann.

此外,Christophe Jorssen,Jin-Hwan Cho,Olivier Binda,Matthias Schulz,Ren\'ee Ahrens,Stephan Schuster和Thomas Neumann 也做出了贡献。

% Additionally, numerous people have contributed to the \pgfname\ system by writing emails, spotting bugs, or sending libraries and patches. Many thanks to all these people, who are too numerous to name them all!

此外,许多人通过写电子邮件,发现错误或发送库和补丁程序为\pgfname\ 系统做出了贡献。非常感谢所有这些人,这些人太多了,无法为所有人标注姓名!

% \subsection{Getting Help}
\subsection{获取帮助}

% When you need help with \pgfname\ and \tikzname, please do the following:

当您需要有关\pgfname\ 和\tikzname 的帮助时,请执行以下操作:

\iffalse
\begin{enumerate}
    \item Read the manual, at least the part that has to do with your
        problem.
    \item If that does not solve the problem, try having a look at the GitHub development page for \pgfname\ and \tikzname\ (see the title of this document). Perhaps someone has already reported a similar problem and someone has found a solution.
    \item On the website you will find numerous forums for getting help. There, you can write to help forums, file bug reports, join mailing lists, and so on.
    \item Before you file a bug report, especially a bug report concerning the installation, make sure that this is really a bug. In particular, have a look at the |.log| file that results when you \TeX\ your files. This |.log| file should show that all the right files are loaded from the right directories. Nearly all installation problems can be resolved by looking at the |.log| file.
    \item \emph{As a last resort} you can try to email me (Till Tantau) or, if the problem concerns the mathematical engine, Mark Wibrow. I do not mind getting emails, I simply get way too many of them. Because of this, I cannot guarantee that your emails will be answered in a  timely fashion or even at all. Your chances that your problem will be fixed are somewhat higher if you mail to the \pgfname\ mailing list (naturally, I read this list and answer questions when I have the time).
\end{enumerate}
\fi

\begin{enumerate}
    \item 阅读手册,至少是与您的问题有关的部分。
    \item 如果那不能解决问题,请尝试查看GitHub上\pgfname\ 和\tikzname\ 的开发页面(请参见本文档的标题)。 也许有人已经报告了类似的问题,并且有人找到了解决方案。
    \item 在网站上,您会找到许多论坛以获得帮助。 在这里,您可以编写帮助讨论,文件错误报告,加入邮件列表等等。
    \item 在提交错误报告(尤其是有关安装的错误报告)之前,请确保这确实是一个错误。 特别要看一下你用\TeX 编译生成的 |.log| 文件。这个 |.log| 文件应显示所有正确的文件均已从正确的目录中加载。 通过查看 |.log| 文件,几乎可以解决所有安装问题。
    \item \emph{作为最后的手段}您可以尝试给我(Till Tantau)发送电子邮件,或者,如果问题与数学引擎有关,请发送电子邮件给Mark Wibrow。 我不介意收到电子邮件,我只是收到太多了。 因此,我不能保证您的电子邮件会及时得到答复,甚至根本不会得到答复。 如果您将邮件邮寄到\pgfname\ 邮件列表,则解决问题的机会会更高(自然,我会阅读此列表并在有时间的时候回答问题)。
\end{enumerate}