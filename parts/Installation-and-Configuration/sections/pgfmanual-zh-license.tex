% Copyright 2018 by Till Tantau
%
% This file may be distributed and/or modified
%
% 1. under the LaTeX Project Public License and/or
% 2. under the GNU Free Documentation License.
%
% See the file doc/generic/pgf/licenses/LICENSE for more details.


% \section{Licenses and Copyright}
\section{许可证和版权}
\label{section-license}

% \subsection{Which License Applies?}
\subsection{必需遵守哪些许可证?}

% Different parts of the \pgfname\ package are distributed under different licenses:

分发\pgfname\ 宏包的不同部分遵守不同的许可证:

%
\begin{enumerate}
    % \item The \emph{code} of the package is dual-license. This means that you can decide which license you wish to use when using the \pgfname\ package. The two options are:
    \item 宏包的\emph{代码}是双重许可的。这意味着当使用 \pgfname\ 宏包时您可以选择其中一个许可证。这两个许可证是:
        %
        \begin{enumerate}
            % \item You can use the \textsc{gnu} Public License, version 2.
            \item 您可以使用\textsc{gnu}公共许可证,v2。
            % \item You can use the \LaTeX\ Project Public License, version 1.3c.
            \item 您可以使用\LaTeX\ 项目公共许可证,v1.3c。
        \end{enumerate}
    % \item The \emph{documentation} of the package is also dual-license. Again, you can choose between two options:
    \item 宏包的\emph{文档}也是双重许可的,同样,您可以选择其中一个许可证。这两个许可证是:
        %
        \begin{enumerate}
            % \item You can use the \textsc{gnu} Free Documentation License, version 1.2.
            \item 您可以使用\textsc{gnu}免费文档许可证,v1.2。
            % \item You can use the \LaTeX\ Project Public License, version 1.3c.
            \item 您可以使用\LaTeX\ 项目公共许可证,v1.3c。
        \end{enumerate}
\end{enumerate}

% The ``documentation of the package'' refers to all files in the subdirectory |doc| of the |pgf| package. A detailed listing can be found in the file |doc/generic/pgf/licenses/manifest-documentation.txt|. All files in other directories are part of the ``code of the package''. A detailed listing can be found in the file |doc/generic/pgf/licenses/manifest-code.txt|.

``宏包的文档''指是指 |pgf| 宏包子目录 | doc| 中的所有文件。 详细列表可在文件 |doc/generic/pgf/licenses/manifest-documentation.txt| 中找到。 其他目录中的所有文件都是``程序包代码''的一部分。 详细列表可在文件 |doc/generic/pgf/licenses/manifest-code.txt| 中找到。

% In the rest of this section, the licenses are presented. The following text is copyrighted, see the plain text versions of these licenses in the directory |doc/generic/pgf/licenses| for details.

在本节的其余部分,将介绍许可证。 以下文本受版权保护,请在目录 |doc/generic/pgf/licenses| 中查看这些许可证有关的详细信息的纯文本版本。 

% The example picture used in this manual, the Brave \textsc{gnu} World logo, is taken from the Brave \textsc{gnu} World homepage, where it is copyrighted as follows: ``Copyright (C) 1999, 2000, 2001, 2002, 2003, 2004 Georg C.~F.\ Greve. Permission is granted to make and distribute verbatim copies of this transcript as long as the copyright and this permission notice appear.''

本手册中使用的示例图片 B​​rave \textsc{gnu} World 徽标取自Brave \textsc{gnu} World 主页,其版权如下:``Copyright (C) 1999, 2000, 2001, 2002, 2003, 2004 Georg C.~F.\ Greve。 只要出现版权和本许可声明,就可以授予制作和分发该笔录的逐字记录副本的权限。''


% \subsection{The GNU Public License, Version 2}
\subsection{GNU 通用公共许可证,V2}

% \subsubsection{Preamble}
\subsubsection{前言}

% The licenses for most software are designed to take away your freedom to share and change it.  By contrast, the \textsc{gnu} General Public License is intended to guarantee your freedom to share and change free software---to make sure the software is free for all its users.  This General Public License applies to most of the Free Software Foundation's software and to any other program whose authors commit to using it.  (Some other Free Software Foundation software is covered by the \textsc{gnu} Library General Public License instead.)  You can apply it to your programs, too.

大多数软件许可证的用意在于剥夺您共享和修改软件的自由。相反,\textsc{gnu} 通用公共许可证力图保证您共享和修改自由软体的自由 —— 保证自由软件对所有使用者都是自由的。GNU GPL 适用于大多数自由软体基金会的软体,以及任何经作者授权使用的其他软件。(有些自由软体基金会软件受 \textsc{gnu} 函数库通用许可证 GNU LGPL 的保护)。您也可以将它用到您的程序中。

% When we speak of free software, we are referring to freedom, not price. Our General Public Licenses are designed to make sure that you have the freedom to distribute copies of free software (and charge for this service if you wish), that you receive source code or can get it if you want it, that you can change the software or use pieces of it in new free programs; and that you know you can do these things.

当我们谈到自由软件时,我们谈的是自由而不是价格。我们把 GNU 通用公共许可证设计成您的保障,确保您拥有发布自由软体的自由(您可以自由决定是否要对此项服务收费); 确保您能收到程序源码或者在您需要时能得到它;确保您能修改软件或将它的一部分用于新的自由软件;而且还确保您知道您拥有这些权利。

% To protect your rights, we need to make restrictions that forbid anyone to deny you these rights or to ask you to surrender the rights.  These restrictions translate to certain responsibilities for you if you distribute copies of the software, or if you modify it.

为了保护您的权利,我们需要作出规定:禁止任何人剥夺您的权利,或者要求您放弃这些权利。如果您修改了自由软件或者发布了软件的副本,这些规定就转化为您的责任。

% For example, if you distribute copies of such a program, whether gratis or for a fee, you must give the recipients all the rights that you have.  You must make sure that they, too, receive or can get the source code.  And you must show them these terms so they know their rights.

例如,如果您发布这样一个程序的副本,不管是免费的还是收费的,您必须将您具有的一切权利给予您的接受者;您必须确认他们能收到或得到源码;并且必须向他们展示这些条款的内容,使他们知道他们有这样的权利。

% We protect your rights with two steps: (1) copyright the software, and (2) offer you this license which gives you legal permission to copy, distribute and/or modify the software. 

我们采取两项措施来保护您的权利:(1)用版权来保护软件;以及(2)提供您许可证,赋予您复制、发布和/或修改这些软件的法律许可。

% Also, for each author's protection and ours, we want to make certain that everyone understands that there is no warranty for this free software.  If the software is modified by someone else and passed on, we want its recipients to know that what they have is not the original, so that any problems introduced by others will not reflect on the original authors' reputations. 

同样,为了保护每个作者和我们自己,我们需要清楚地让每个人明白,自由软件没有担保。如果由于某人修改了软件,并继续加以传播,我们需要它的接受者明白:他们所得到的并不是原来的自由软体。由其他人引入的任何问题,不应损害原作者的声誉。

% Finally, any free program is threatened constantly by software patents. We wish to avoid the danger that redistributors of a free program will individually obtain patent licenses, in effect making the program proprietary.  To prevent this, we have made it clear that any patent must be licensed for everyone's free use or not licensed at all.

最后,由于任何自由软件不断受到软件专利的威胁,故我们希望避免这样的风险,即如果自由软件的再发 布者以个人名义获得专利许可证,也就等同将软体变为私有。为防止这一点,我们必须明确声明:任何专利必 须以允许每个人自由使用为前提,否则就不应授予专利。

% The precise terms and conditions for copying, distribution and modification follow.

下面是有关复制、发布和修改的确切的条款和条件。


% \subsubsection{Terms and Conditions For Copying, Distribution and Modification}
\subsubsection{复制、分发与修改的条款与条件}

\begin{enumerate}
    \addtocounter{enumi}{-1}
    % \item This License applies to any program or other work which contains a notice placed by the copyright holder saying it may be distributed under the terms of this General Public License.  The ``Program'', below, refers to any such program or work, and a ``work based on the Program'' means either the Program or any derivative work under copyright law: that is to say, a work containing the Program or a portion of it, either verbatim or with modifications and/or translated into another language. (Hereinafter, translation is included without limitation in the term ``modification''.) Each licensee is addressed as ``you''.

        % Activities other than copying, distribution and modification are not covered by this License; they are outside its scope.  The act of running the Program is not restricted, and the output from the Program is covered only if its contents constitute a work based on the Program (independent of having been made by running the Program). Whether that is true depends on what the Program does.
    
    \item 凡是版权所有者在其程序和作品中声明其程序和作品可以在 GNU GPL 条款的约束下发布,这样的程序 或作品都受到本许可证约束。下面提到的``程序''指的是任何这样的程序或作品。而``程序的衍生作品''指的是这样的程序或者版权法认定下的衍生作品,也就是说包含此程序或程序的一部分的套件,可以是原封不动的,或经过修改的,和/或翻译成其他语言的(程序)。(在下文中,``修改''一词的涵义一律包含翻译作品。) 每个许可证接受人用``您''来称呼。

        本许可证条款不适用于复制、发布和修改以外的行为。这些行为超出这些条款的范围。执行本程序的行为不受条款的限制。而程序的输出只有在其内容构成本程序的衍生作品(并非只是因为该输出由本程序所产生)时,这一条款才适用。至于程序的输出内容是否构成本程序的衍生作品,则取决于程序具体的用途。
    % \item You may copy and distribute verbatim copies of the Program's source code as you receive it, in any medium, provided that you conspicuously and appropriately publish on each copy an appropriate copyright notice and disclaimer of warranty; keep intact all the notices that refer to this License and to the absence of any warranty; and give any other recipients of the Program a copy of this License along with the Program.

        % You may charge a fee for the physical act of transferring a copy, and you may at your option offer warranty protection in exchange for a fee.

    \item 只要您在每一程序副本上明显和恰当地宣告版权声明和无担保的声明,并原封不动保持此许可证的声明和无担保的声明,并将此许可证连同程序一起给其他每位程序接受者,您就可以用任何媒体复制和发布您收到的程序的源码。

        您可以据转让副本的实际行动收取一定费用。您也可以自由决定是否以提供担保来换取一定的费用。

    % \item You may modify your copy or copies of the Program or any portion of it, thus forming a work based on the Program, and copy and distribute such modifications or work under the terms of Section 1 above, provided that you also meet all of these conditions:
    \item 您可以修改程序的一个或几个副本或程序的任何部分,以此形成基于这些程序的衍生作品。只要您同时满足下面的所有条件,您就可以按前面第一款的要求复制和发布这一经过修改的程序或作品:
        %
        \begin{enumerate}
            % \item You must cause the modified files to carry prominent notices stating that you changed the files and the date of any change.
            \item 您必须在修改过的文件上附加明显的说明:叙明您修改过这些文件,以及修改的日期。
            % \item You must cause any work that you distribute or publish, that in whole or in part contains or is derived from the Program or any part thereof, to be licensed as a whole at no charge to all third parties under the terms of this License.
            \item 您必须让您发布或出版的作品,包括本程序的全部或一部分,或内含本程序的全部或部分所衍生的作品,允许第三方在此许可证条款下使用,并且不得因为此项授权行为而收费。
            % \item If the modified program normally reads commands interactively when run, you must cause it, when started running for such interactive use in the most ordinary way, to print or display an announcement including an appropriate copyright notice and a notice that there is no warranty (or else, saying that you provide a warranty) and that users may redistribute the program under these conditions, and telling the user how to view a copy of this License.  (Exception: if the Program itself is interactive but does not normally print such an announcement, your work based on the Program is not required to print an announcement.)
            \item 如果修改的程序在执行时以互动方式读取命令,您必须使它在开始进入最常使用的方式时列印或显示这样的声明:适当的版权声明和无担保的声明(或者您提供 担保的声明);使用者可以按此许可证条款重新发布程序的声明,并告诉使用者如何看到这一许可证 的副本。(例外的情况:如果原始程序以互动方式工作,但它通常并不列印这样的声明,那么您基于 此程序的作品也就不用列印声明)。
        \end{enumerate}
        %
        % These requirements apply to the modified work as a whole.  If identifiable sections of that work are not derived from the Program, and can be reasonably considered independent and separate works in themselves, then this License, and its terms, do not apply to those sections when you distribute them as separate works.  But when you distribute the same sections as part of a whole which is a work based on the Program, the distribution of the whole must be on the terms of this License, whose permissions for other licensees extend to the entire whole, and thus to each and every part regardless of who wrote it.

        这些要求适用于整个修改过的作品。如果能够确定作品的一部分并非是本程序的衍生产品,且可以合理地 单独考虑并将它与原作品分开的话,则当您将它作为独立的作品发布时,它不受此许可证和其条款的约 束。但是当您将这部分与基于本程序的作品一同发布时,则整个套件将受到本许可证条款约束,因为本许 可证对于其他许可证持有人的授权扩大到整个产品,也就是套件的每个部分,不管它是谁写的。

        % Thus, it is not the intent of this section to claim rights or contest your rights to work written entirely by you; rather, the intent is to exercise the right to control the distribution of derivative or collective works based on the Program.

        因此,本条款的意图不在于剥夺您拥有对完全由您自身完成作品的权利,而在于履行权利来控制基于本程 式的集体作品或衍生作品的发布。

        % In addition, mere aggregation of another work not based on the Program with the Program (or with a work based on the Program) on a volume of a storage or distribution medium does not bring the other work under the scope of this License.

        此外,将与本程序无关的作品和本程序(或本程序的衍生作品)一起放在贮存媒体或发布媒体的同一卷上,并不使其他作品置于此许可证的约束范围之内。
    % \item You may copy and distribute the Program (or a work based on it, under Section~2) in object code or executable form under the terms of Sections~1 and 2 above provided that you also do one of the following:
    \item 只要您遵守前面的第 1、2 款,并同时满足下列三条中的任一条,您就可以以目标码或可执行形式复制或发布程序(或符合第 2 款,本程序的衍生作品):
        %
        \begin{enumerate}
            % \item Accompany it with the complete corresponding machine-readable source code, which must be distributed under the terms of Sections~1 and 2 above on a medium customarily used for software interchange; or,
            \item 在通常用作软件交换的媒体上,和目标码一起附上机器可读的完整的本程序源码。这些源码的发布应符合上面第 1、2 款的要求。或者,
            % \item Accompany it with a written offer, valid for at least three years, to give any third party, for a charge no more than your cost of physically performing source distribution, a complete machine-readable copy of the corresponding source code, to be distributed under the terms of Sections~1 and 2 above on a medium customarily used for software interchange; or,
            \item 在通常用作软件交换的媒体上,和目标码一起,附上书面报价,提供替第三方复制源码的服务。该书面报价有效期不得少于 3 年,费用不得超过完成原程序发布的实际成本,源码的发布应符合上面的第 1、2 款的要求。或者,
            % \item Accompany it with the information you received as to the offer to distribute corresponding source code.  (This alternative is allowed only for noncommercial distribution and only if you received the program in object code or executable form with such an offer, in accord with Subsubsection~b above.)
            \item 和目标码一起,附上您收到的发布源码的报价信息。(这一条款只适用于非商业性发布,而且您只收到程序的目标码或可执行码,和按 b 款要求提供的报价。)
        \end{enumerate}
        %
        % The source code for a work means the preferred form of the work for making modifications to it.  For an executable work, complete source code means all the source code for all modules it contains, plus any associated interface definition files, plus the scripts used to control compilation and installation of the executable.  However, as a special exception, the source code distributed need not include anything that is normally distributed (in either source or binary form) with the major components (compiler, kernel, and so on) of the operating system on which the executable runs, unless that component itself accompanies the executable.

        作品的源码指的是对作品进行修改最优先择取的形式。对可执行的作品而言,完整的源码 套件包括:所有模组的所有原始程序,加上有关的介面的定义,加上控制可执行作品的安装 和编译的脚本。至于那些通常伴随着执行本程序所需的作业系统元件(如编译器、核心等)而发 布的软件(不论是源码或可执行码),则不在本许可证要求以程序源码形式伴随发布之列,除非它是本程 序的一部分。

        % If distribution of executable or object code is made by offering access to copy from a designated place, then offering equivalent access to copy the source code from the same place counts as distribution of the source code, even though third parties are not compelled to copy the source along with the object code.

        如果可执行码或目标码是以指定复制地点的方式来发布,那么在同一地点提供等价的源码复制服务也可以算作源码的发布,然而第三方并不需因此而负有必与目标码一起复制源码的义务。
    % \item You may not copy, modify, sublicense, or distribute the Program except as expressly provided under this License.  Any attempt otherwise to copy, modify, sublicense or distribute the Program is void, and will automatically terminate your rights under this License. However, parties who have received copies, or rights, from you under this License will not have their licenses terminated so long as such parties remain in full compliance.
    \item 除了本许可证明白声明的方式之外,您不能复制、修改、转发许可证和发布程序。任何试图用其他方式复 制、修改、转发许可证和发布程序是无效的,而且将自动结束许可证赋予您的权利。然而,对那些从您那 里按许可证条款得到副本和权利的人们,只要他们继续全面履行条款,许可证赋予他们的权利仍然有效。
    % \item You are not required to accept this License, since you have not signed it.  However, nothing else grants you permission to modify or distribute the Program or its derivative works.  These actions are prohibited by law if you do not accept this License.  Therefore, by modifying or distributing the Program (or any work based on the Program), you indicate your acceptance of this License to do so, and all its terms and conditions for copying, distributing or modifying the Program or works based on it.
    \item 您没有在许可证上签字,因而您没有必要一定接受此许可证。然而,没有任何其他东西赋予您修改和发布 程序及其衍生作品的权利。如果您不接受许可证,这些行为是法律禁止的。因此,如果您修改或发布程序 (或本程式的衍生作品),您就表明您接受这一许可证以及它的所有有关复制、发布和修改程序或基于程 序的作品的条款和条件。
    % \item Each time you redistribute the Program (or any work based on the Program), the recipient automatically receives a license from the original licensor to copy, distribute or modify the Program subject to these terms and conditions.  You may not impose any further restrictions on the recipients' exercise of the rights granted herein. You are not responsible for enforcing compliance by third parties to this License.
    \item 每当您重新发布程序(或任何程序的衍生作品)时,接受者自动从原始许可证颁发者那里接到受这些条款和条件支配的复制、发布或修改本程序的许可。您不可以增加任何条款来进一步限制本许可证赋予他们的 权利。您也没有强求第三方履行许可证条款的义务。
    % \item If, as a consequence of a court judgment or allegation of patent infringement or for any other reason (not limited to patent issues), conditions are imposed on you (whether by court order, agreement or otherwise) that contradict the conditions of this License, they do not excuse you from the conditions of this License.  If you cannot distribute so as to satisfy simultaneously your obligations under this License and any other pertinent obligations, then as a consequence you may not distribute the Program at all.  For example, if a patent license would not permit royalty-free redistribution of the Program by all those who receive copies directly or indirectly through you, then the only way you could satisfy both it and this License would be to refrain entirely from distribution of the Program.
    \item 如果因法院判决或因违反专利的指控或任何其他原因(不限于专利问题),使得强加于您的条件(不管是 法院判决、协议或其他)和许可证的条件有冲突时,他们也不能令您背离许可证的条款。在您不能同时满 足本许可证规定的义务及其他相关的义务来发布程序时,结果是您只能够根本不发布程序。例如,如果某 一专利许可证不允许所有直接或间接从您那里接受副本的人们,在不付专利费的情况下重新发布程序,唯 一能同时满足两方面要求的办法是停止发布程序。

        % If any portion of this section is held invalid or unenforceable under any particular circumstance, the balance of the section is intended to apply and the section as a whole is intended to apply in other circumstances.

        如果本条款的任一部分在特定的环境下无效或无法实施,本条其余部分仍应适用,并将这部分条款作为整体用于其他环境。

        % It is not the purpose of this section to induce you to infringe any patents or other property right claims or to contest validity of any such claims; this section has the sole purpose of protecting the integrity of the free software distribution system, which is implemented by public license practices.  Many people have made generous contributions to the wide range of software distributed through that system in reliance on consistent application of that system; it is up to the author/donor to decide if he or she is willing to distribute software through any other system and a licensee cannot impose that choice.

        本条款的目的不在于引诱您侵犯专利或其他财产权的主张,或争论这种主张的有效性。本条款的主要目的在于保护自由软体发布系统的完整性。它是通过公共许可证的应用来实现的。许多人已依赖同是出自此系统的应用程序,经由此系统发布大量自由软体而做出慷慨的供献。作者/捐献者有权决定他/她是否通过任何其他系统发布软体,许可证接受者不能强迫作者/捐献者做某种特定的选择。

        % This section is intended to make thoroughly clear what is believed to be a consequence of the rest of this License.

        我们相信许可证其他部分已涵盖本节所述状况,本节目的只在更明确说明许可证其余部分可能产生的结果。
    % \item If the distribution and/or use of the Program is restricted in certain countries either by patents or by copyrighted interfaces, the original copyright holder who places the Program under this License may add an explicit geographical distribution limitation excluding those countries, so that distribution is permitted only in or among countries not thus excluded.  In such case, this License incorporates the limitation as if written in the body of this License.
    \item 如果由于专利或者由于有版权的介面问题使程序在某些国家的发布和使用受到限制,则以本许可证发布程 序的原始作者可以增加发布地区的限制条款,将这些国家明确排除在外,并在这些国家以外的地区发布程 序。在这种情况下,这些限制条款如同写入本许可证一样,成为许可证的条款。
    % \item The Free Software Foundation may publish revised and/or new versions of the General Public License from time to time.  Such new versions will be similar in spirit to the present version, but may differ in detail to address new problems or concerns.
    \item 自由软体基金会可能随时出版通用公共许可证的修改版和/或新版。新版和当前的版本在精神上保持一致,但在细节上可能有所不同,以便处理新的问题与状况。

        % Each version is given a distinguishing version number.  If the Program specifies a version number of this License which applies to it and ``any later version'', you have the option of following the terms and conditions either of that version or of any later version published by the Free Software Foundation.  If the Program does not specify a version number of this License, you may choose any version ever published by the Free Software Foundation.

        每一版本都有不同的版本号。如果程序指定可适用的许可证版本号以及''任何更新的版 本'',您有权选择遵循指定的版本或自由软体基金会以后出版的新版本。如果程序未指定许可证版本,您 可选择自由软体基金会已经出版的任何版本。
    % \item If you wish to incorporate parts of the Program into other free programs whose distribution conditions are different, write to the author to ask for permission.  For software which is copyrighted by the Free Software Foundation, write to the Free Software Foundation; we sometimes make exceptions for this.  Our decision will be guided by the two goals of preserving the free status of all derivatives of our free software and of promoting the sharing and reuse of software generally.
    \item 如果您愿意将程序的一部分结合到其他自由程序中,而它们的发布条件不同,请写信给作者,要求准予使 用。如果是自由软件基金会加以版权保护的软件,请写信给自由软件基金会,我们有时会作为例外的情况 处理。我们的决定受两个主要目标的指导,这两个主要目标是:我们的自由软件的衍生作品继续保持自由 状态,以及从整体上促进软件的共享和重复利用。
\end{enumerate}


% \subsubsection{No Warranty}
\subsubsection{无担保声明}

\begin{enumerate}
        \addtocounter{enumi}{9}
    % \item Because the program is licensed free of charge, there is no warranty for the program, to the extent permitted by applicable law. Except when otherwise stated in writing the copyright holders and/or other parties provide the program ``as is'' without warranty of any kind, either expressed or implied, including, but not limited to, the implied warranties of merchantability and fitness for a particular purpose.  The entire risk as to the quality and performance of the program is with you. Should the program prove defective, you assume the cost of all necessary servicing, repair or correction.
    \item 由于准予免费使用本程序,因此在法律许可范围内,对本程序并不负担保责任。除非另有书面说明,版权 所有者和/或其他提供程序的人们``一样''不提供任何类型的担保,不论是明确的,还是隐含的,包括但 不限于可销售和适合特定用途的隐含保证。全部的风险,如程序的质量和性能问题都由您来承担。如果程 序出现缺陷,您应当承担所有必要的服务、修复和改正的费用。
    % \item In no event unless required by applicable law or agreed to in writing will any copyright holder, or any other party who may modify and/or redistribute the program as permitted above, be liable to you for damages, including any general, special, incidental or consequential damages arising out of the use or inability to use the program (including but not limited to loss of data or data being rendered inaccurate or losses sustained by you or third parties or a failure of the program to operate with any other programs), even if such holder or other party has been advised of the possibility of such damages.
    \item 非经法律要求或书面同意,在任何情况下,任何版权所有者或任何按许可证条款修改和/或发布程序的人 们都不对您的损失负有任何责任。包括由于使用或不能使用程序引起的任何一般的、特殊的、偶然发生的 或重大的损失(包括但不限于数据的损失,或者数据变得不精确,或者您或第三方的持续的损失,或者程 序不能和其他程序相兼容)。即使版权所有者和其他人已被告知这种损失的可能性也不例外。
\end{enumerate}


\providecommand{\LPPLsection}{\subsection}
\providecommand{\LPPLsubsection}{\subsubsection}
\providecommand{\LPPLsubsubsection}{\subsubsection}
\providecommand{\LPPLparagraph}{\paragraph}


% The file lppl.tex, some minor typographic changes:

%
% $Id$
%
% Copyright 1999 2002-2006 LaTeX3 Project
%    Everyone is allowed to distribute verbatim copies of this
%    license document, but modification of it is not allowed.
%
%
% If you wish to load it as part of a ``doc'' source, you have to
% ensure that a) % is a comment character and b) that short verb
% characters are being turned off, i.e.,
%
%   \DeleteShortVerb{\'}   % or whatever was made a shorthand
%   \MakePercentComment
%   \input{lppl}
%   \MakePercentIgnore
%   \MakeShortVerb{\'}     % turn it on again if necessary
%
%
% By default the license is produced with \section* as the highest
% heading level. If this is not appropriate for the document in which
% it is included define the commands listed below before loading this
% document, e.g., for inclusion as a separate chapter define:
%
%  \providecommand{\LPPLsection}{\chapter*}
%  \providecommand{\LPPLsubsection}{\section*}
%  \providecommand{\LPPLsubsubsection}{\subsection*}
%  \providecommand{\LPPLparagraph}{\subsubsection*}
%
%
% To allow cross-referencing the headings \label's have been attached
% to them, all starting with ``LPPL:''. As by default headings without
% numbers are produced, this will only allow page references.
% However, you can use the titleref package to produce textual
% references or you change the definitions of \LPPLsection, and
% friends to generated numbered headings.
%
%
% We want it to be possible that this file can be processed by
% (pdf)LaTeX on its own, or that this file can be included in another
% LaTeX document without any modification whatsoever.
% Hence the little test below.
%
%
\makeatletter
\ifx\@preamblecmds\@notprerr
  % In this case the preamble has already been processed so this file
  % is loaded as part of another document; just enclose everything in
  % a group
  \let\LPPLicense\bgroup
  \let\endLPPLicense\egroup
\else
  % In this case the preamble has not been processed yet so this file
  % is processed by itself.
  \documentclass{article}
  \let\LPPLicense\document
  \let\endLPPLicense\enddocument
\fi
\makeatother


\begin{LPPLicense}
    \providecommand{\LPPLsection}{\section*}
    \providecommand{\LPPLsubsection}{\subsection*}
    \providecommand{\LPPLsubsubsection}{\subsubsection*}
    \providecommand{\LPPLparagraph}{\paragraph*}
    \providecommand*{\LPPLfile}[1]{\texttt{#1}}
    \providecommand*{\LPPLdocfile}[1]{`\LPPLfile{#1.tex}'}
    \providecommand*{\LPPL}{\textsc{lppl}}

    % \LPPLsection{The \LaTeX\ Project Public License, Version 1.3c 2006-05-20}
    \LPPLsection{\LaTeX\ 项目公共许可证(LPPL), V1.3c 2006-05-20}
    \label{LPPL:LPPL}

%    \textbf{Copyright 1999, 2002--2006 \LaTeX3 Project}
%    \begin{quotation}
%        Everyone is allowed to distribute verbatim copies of this
%        license document, but modification of it is not allowed.
%    \end{quotation}

    % \LPPLsubsection{Preamble}
    \LPPLsubsection{导言}
    \label{LPPL:Preamble}

    % The \LaTeX\ Project Public License (\LPPL) is the primary license under which the \LaTeX\ kernel and the base \LaTeX\ packages are distributed.

    \LaTeX\ 项目公共许可证(LPPL)是分发\LaTeX\ 内核和基本的 \LaTeX\ 宏包时必须遵守的许可证。

    % You may use this license for any work of which you hold the copyright and which you wish to distribute.  This license may be particularly suitable if your work is \TeX-related (such as a \LaTeX\ package), but it is written in such a way that you can use it even if your work is unrelated to \TeX.

    您可以将该许可证用于您的维护版权和分发的工作。该许可证很适合您的 \TeX 相关的工作(如 \LaTeX\  宏包),但它是写于这样一种情况之下,即您的工作与 \TeX 毫不相干,您也可以使用该许可证。

    % The section `\textsc{whether and how to distribute works under this license}', below, gives instructions, examples, and recommendations for authors who are considering distributing their works under this license.

    本节的内容是``是否以及如何在这个许可证下做分发工作'',下面给出一些说明、例子和建议以方便那些正在考虑在这个许可证下分发他们作品的作者。

    % This license gives conditions under which a work may be distributed and modified, as well as conditions under which modified versions of that work may be distributed.

    这个许可证在某个作品可能被分发和修订,以及在那个作品的修订版的情况下可能被分发。

    % We, the \LaTeX3 Project, believe that the conditions below give you the freedom to make and distribute modified versions of your work that conform with whatever technical specifications you wish while maintaining the availability, integrity, and reliability of that work.  If you do not see how to achieve your goal while meeting these conditions, then read the document \LPPLdocfile{cfgguide} and \LPPLdocfile{modguide} in the base \LaTeX\ distribution for suggestions.

    \LaTeX3 项目下述的条件给您制作和发布您作品修订版本的自由,同时与您期望的技术规范一致,保持您作品的可用性、完整性、可靠性。如果您不知道如何实现您的目标,同时满足这些条件,请在分发的 \LaTeX\ 中阅读文档 \LPPLdocfile{cfgguide} 和 \LPPLdocfile{modguide} 来获得建议。


    % \LPPLsubsection{Definitions}
    \LPPLsubsection{定义}
    \label{LPPL:Definitions}

    % In this license document the following terms are used:

    本许可证文档使用了如下术语:

    \begin{description}
        % \item[Work] Any work being distributed under this License.
        \item[作品] 指在这个许可证许可下进行的任何分发工作。
        % \item[Derived Work] Any work that under any applicable law is derived from the Work.
        \item[衍生作品] 指衍生于作品的任何合法的其它作品。
        % \item[Modification] Any procedure that produces a Derived Work under any applicable law -- for example, the production of a file containing an original file associated with the Work or a significant portion of such a file, either verbatim or with modifications and/or translated into another language.
        \item[修正] 指产生一个合法衍生作品的任何过程,例如,一个文件的产生过程,包括与这个文件相关的原始文件或者像这样文件的一个重要部分无论是直接引用或修改和/或翻译成另一种语言。
        % \item[Modify] To apply any procedure that produces a Derived Work under any applicable law.
        \item[修改] 指实施产生合法衍生作品的任何过程。
        % \item[Distribution] Making copies of the Work available from one person to another, in whole or in part.  Distribution includes (but is not limited to) making any electronic components of the Work accessible by file transfer protocols such as \textsc{ftp} or \textsc{http} or by shared file systems such as Sun's Network File System (\textsc{nfs}).
        \item[分发] 指从一个人复制作品的全部或部分给另一个人。分发包括但不限于制作作品的电子元件,该电子元件适合文件传协议如 \textsc{ftp} 或 \textsc{http},或适合共享文件系统如升阳公司的网络文件系统(\textsc{nfs})。
        % \item[Compiled Work] A version of the Work that has been processed into a form where it is directly usable on a computer system. This processing may include using installation facilities provided by the Work, transformations of the Work, copying of components of the Work, or other activities.  Note that modification of any installation facilities provided by the Work constitutes modification of the Work.
        \item[编译作品] 指将作品处理成可以直接在计算机系统里可用的格式的过程。这种处理可能包 括作品提供的安装工具的使用、作品的转变、作品部分的复制,或其他处理。请注意,任何对由作品提供的安装工具的修改构成了对作品的修改。
        % \item[Current Maintainer] A person or persons nominated as such within the Work.  If there is no such explicit nomination then it is the `Copyright Holder' under any applicable law.
        \item[当前维护者] 指在作品里被提名的一个或者几个人。如果没有这样明确的提名则是合法的``版权持有人''。
        % \item[Base Interpreter] A program or process that is normally needed for running or interpreting a part or the whole of the Work.
        \item[基本解释程序] 指一个软件或程序,它对运行或解释作品的一部分或全部通常是必需的。

            % A Base Interpreter may depend on external components but these are not considered part of the Base Interpreter provided that each external component clearly identifies itself whenever it is used interactively.  Unless explicitly specified when applying the license to the Work, the only applicable Base Interpreter is a `\LaTeX-Format' or in the case of files belonging to the `\LaTeX-format' a program implementing the `\TeX{} language'.

            一个基本解释程序可能依赖外部组件,但这些外部组件不认为是基本解释程 序的一部分,规定无论何时在交互使用外部组件时都必须明确标识每一个外部组件。否则,将许可证应用于作品时,必须明确指明唯一可用的基本解释程序是``\LaTeX-格式''或程序执行``\TeX\ ''语言的文件属于``\LaTeX-格式''。
    \end{description}


    % \LPPLsubsection{Conditions on Distribution and Modification}
    \LPPLsubsection{分发和修改的条件}
    \label{LPPL:Conditions}

    \begin{enumerate}
        % \item Activities other than distribution and/or modification of the Work are not covered by this license; they are outside its scope. In particular, the act of running the Work is not restricted and no requirements are made concerning any offers of support for the Work.
        \item 除分发和/或修改作品的行为之外,其它行为没有涵盖在许可证之中。特别地,作品运转的行为不受限制,并且没有任何关于提供支持这项作品的要求。
        % \item\label{LPPL:item:distribute} You may distribute a complete, unmodified copy of the Work as you received it.  Distribution of only part of the Work is considered modification of the Work, and no right to distribute such a Derived Work may be assumed under the terms of this clause.
        \item\label{LPPL:item:distribute} 您可以分发一个完整的、未经修改的您收到的作品的副本。只有部分作品的分发是需要修改的,但没有权利分发这样的符合该条款中术语的衍生作品。
        % \item You may distribute a Compiled Work that has been generated from a complete, unmodified copy of the Work as distributed under Clause~\ref{LPPL:item:distribute} above, as long as that Compiled Work is distributed in such a way that the recipients may install the Compiled Work on their system exactly as it would have been installed if they generated a Compiled Work directly from the Work.
        \item 在符合上面第 \ref{LPPL:item:distribute} 条款时,您可以分发一件从一个完整的、未经修改的作品副本产生的编译作品。只要该编译作品是直接从作品中产生的,并按照这样的分发方式,收件人便可以在他们的系统中完全正确地安装编译作品。
        % \item\label{LPPL:item:currmaint} If you are the Current Maintainer of the Work, you may, without restriction, modify the Work, thus creating a Derived Work.  You may also distribute the Derived Work without restriction, including Compiled Works generated from the Derived Work.  Derived Works distributed in this manner by the Current Maintainer are considered to be updated versions of the Work.
        \item\label{LPPL:item:currmaint} 如果您是作品当前的维护者,则可以没有限制地修改作品,从而创建一个衍生作 品。您也可以没有限制地分发衍生作品,包括衍生作品中产生的编译作品。当前维护 者用这种方式分发的衍生作品被认为是作品的新版本。
        % \item If you are not the Current Maintainer of the Work, you may modify your copy of the Work, thus creating a Derived Work based on the Work, and compile this Derived Work, thus creating a Compiled Work based on the Derived Work.
        \item 如果您不是作品当前维护者,则可以修改您的作品副本,从而在作品的基础上创建一个衍生作品,并编译此衍生作品从而这个衍生作品的基础上创造一个编译作品。
        % \item\label{LPPL:item:conditions} If you are not the Current Maintainer of the Work, you may distribute a Derived Work provided the following conditions are met for every component of the Work unless that component clearly states in the copyright notice that it is exempt from that condition.  Only the Current Maintainer is allowed to add such statements of exemption to a component of the Work.
        \item\label{LPPL:item:conditions} 除了版权说明中明确声明的部分,只要对作品的每个部分符合下列条件时,即使您不是作品的当前维护者,您也可以分发衍生作品。只有当前维护者可以对作品的一个组成部分增加这样的声明。
            %
            \begin{enumerate}
                % \item If a component of this Derived Work can be a direct replacement for a component of the Work when that component is used with the Base Interpreter, then, wherever this component of the Work identifies itself to the user when used interactively with that Base Interpreter, the replacement component of this Derived Work clearly and unambiguously identifies itself as a modified version of this component to the user when used interactively with that Base Interpreter.
                \item 如果衍生作品的一个部分可以直接替换作品中一个基本解释程序使用的部分,然后,当交互式地使 用基本解释程序时,作品这个部分被用户识别的任何地方,衍生作品的替代部分清楚明确地被用户 识别为这个部分的一个修改版本。
                % \item Every component of the Derived Work contains prominent notices detailing the nature of the changes to that component, or a prominent reference to another file that is distributed as part of the Derived Work and that contains a complete and accurate log of the changes.
                \item 衍生作品的每个部分都包含针对该部分详述性质变化的突出声明,或对另一个 文件的突出参考条目,这个文件是作为衍生作品的一部分分发的,并且包含 一个完整准确的更改日志。
                % \item No information in the Derived Work implies that any persons, including (but not limited to) the authors of the original version of the Work, provide any support, including (but not limited to) the reporting and handling of errors, to recipients of the Derived Work unless those persons have stated explicitly that they do provide such support for the Derived Work.
                \item 衍生作品的信息中没有针对该衍生作品收件人的任何暗示,这些人包括(但不限于)作品的初始版本作者、提供任何支持包括(但不限于)错误报告和处理的人,除非这些人已经明 确表示为衍生作品提供了这种支持。
                % \item You distribute at least one of the following with the Derived Work:
                \item 您可以至少分发以下衍生作品中的一种:
                    %
                    \begin{enumerate}
                        % \item A complete, unmodified copy of the Work; if your distribution of a modified component is made by offering access to copy the modified component from a designated place, then offering equivalent access to copy the Work from the same or some similar place meets this condition, even though third parties are not compelled to copy the Work along with the modified component;
                        \item 一个完整的、未经修改的作品副本;如果修改部分的分发是通过提供权限从一个指定的地点复制修改的部分来实现,那么,会 提供相同的权限用于从相同或相似的符合该条件的地点复制作品,即使第三方不会随着修改部分被迫复制作品。
                        % \item Information that is sufficient to obtain a complete, unmodified copy of the Work.
                        \item 足以获得一个完整的、未经修改的作品副本的信息。
                    \end{enumerate}
            \end{enumerate}
        %
        % \item If you are not the Current Maintainer of the Work, you may distribute a Compiled Work generated from a Derived Work, as long as the Derived Work is distributed to all recipients of the Compiled Work, and as long as the conditions of Clause~\ref{LPPL:item:conditions}, above, are met with regard to the Derived Work.
        \item 如果您不是作品的当前维护者,只要衍生作品是分发给编译作品的全部收件人,并且满足上述第 \ref{LPPL:item:conditions} 条的关于衍生作品的条件,您就可以分发来自衍生作品的编译作品。
        % \item The conditions above are not intended to prohibit, and hence do not apply to, the modification, by any method, of any component so that it becomes identical to an updated version of that component of the Work as it is distributed by the Current Maintainer under Clause~\ref{LPPL:item:currmaint}, above.
        \item 上述条件的目的不是禁止,并且因此不适用以任何方式修改任何部分,以便使其成为作品的那个部分相同的更新版本,正如上述第 \ref{LPPL:item:currmaint} 条当前维护者分发的那样。
        % \item Distribution of the Work or any Derived Work in an alternative format, where the Work or that Derived Work (in whole or in part) is then produced by applying some process to that format, does not relax or nullify any sections of this license as they pertain to the results of applying that process.
        \item 以可选的方式分发作品或衍生作品,不是放宽或者废弃该许可 证的任何部分,因为它涉及应用过程的结果,在可选的方式中,作品 或衍生作品(全部或部分)由适应于该方式的应用过程产生。
        \item \null
            \begin{enumerate}
                % \item A Derived Work may be distributed under a different license provided that license itself honors the conditions listed in Clause~\ref{LPPL:item:conditions} above, in regard to the Work, though it does not have to honor the rest of the conditions in this license.
                \item 衍生作品可能在不同的许可证许可下进行分发,关于作品,如果那个许可证本身尊重上述第 \ref{LPPL:item:conditions} 条中的条件,虽然它并没有必须尊重本许可证的其他条件。
                % \item If a Derived Work is distributed under a different license, that Derived Work must provide sufficient documentation as part of itself to allow each recipient of that Derived Work to honor the restrictions in Clause~\ref{LPPL:item:conditions} above, concerning changes from the Work.
                \item 就作品的变化而言,如果派生作品在不同的许可证许可下进行分发,衍生作品必须提供足够的文件作为它的一部分来允许每一个收件人,兑现上述第 \ref{LPPL:item:conditions} 条中的限制。
            \end{enumerate}
            %
        % \item This license places no restrictions on works that are unrelated to the Work, nor does this license place any restrictions on aggregating such works with the Work by any means.
        \item 本许可证对与该作品无关的作品没有限制,也对以任何方式聚集这些作品的工作没有任何限制。
        % \item Nothing in this license is intended to, or may be used to, prevent complete compliance by all parties with all applicable laws.
        \item 本许可证没有任何企图或用于破坏适应的法律。
    \end{enumerate}


    % \LPPLsubsection{No Warranty}
    \LPPLsubsection{无担保声明}
    \label{LPPL:Warranty}

    % There is no warranty for the Work.  Except when otherwise stated in writing, the Copyright Holder provides the Work `as is', without warranty of any kind, either expressed or implied, including, but not limited to, the implied warranties of merchantability and fitness for a particular purpose.  The entire risk as to the quality and performance of the Work is with you.  Should the Work prove defective, you assume the cost of all necessary servicing, repair, or correction.

    作品没有保修。除非另有书面说明,版权持有人提供作品``按照原样'',没有任何形式的担保,明 示或暗示,包括但不限于,针对特定用途的适销性和适用性的暗示保证。作为作品的质量和性能的全部风险与 您同在。作品应证明有缺陷,否则您将承担所有必要的维修、修理、或改正的费用。

    % In no event unless required by applicable law or agreed to in writing will The Copyright Holder, or any author named in the components of the Work, or any other party who may distribute and/or modify the Work as permitted above, be liable to you for damages, including any general, special, incidental or consequential damages arising out of any use of the Work or out of inability to use the Work (including, but not limited to, loss of data, data being rendered inaccurate, or losses sustained by anyone as a result of any failure of the Work to operate with any other programs), even if the Copyright Holder or said author or said other party has been advised of the possibility of such damages.

    除非适用法律要求,或版权持有人书面同意,或该作品任何署名作者,分发和/或修改作品的其他人员,同 意上述或任何其他方,负责对您损害赔偿,否则将不赔偿。损害包括任何一般,特殊,附带或间接损害所产生 的任何无法使用作品的情况,(包括但不限于数据丢失,所呈现的数据不准确,或持续受损失人作为工作的任何 失败与任何其他程序)的结果,即使版权持有人或者作者,或者对方已被告知此类损害的可能性。


    % \LPPLsubsection{Maintenance of The Work}
    \LPPLsubsection{作品的维护}
    \label{LPPL:Maintenance}

    % The Work has the status `author-maintained' if the Copyright Holder explicitly and prominently states near the primary copyright notice in the Work that the Work can only be maintained by the Copyright Holder or simply that it is `author-maintained'.

    工作中有``作者维持''这么一个状态,如果版权所有者明确地、突出地说要注意作品的主要版权,那么作品只能通过版权所有者或者是``作者维持''这么一个简单的状态去维持。

    % The Work has the status `maintained' if there is a Current Maintainer who has indicated in the Work that they are willing to receive error reports for the Work (for example, by supplying a valid e-mail address). It is not required for the Current Maintainer to acknowledge or act upon these error reports.

    如果当前维护者表明他们愿意接受来自作品中的错误(例如,通过提供一个有效地电子邮箱),则作品处于``维持''状态。但并不要求当前维护者对这些错误有深刻理解或者采取措施。

    % The Work changes from status `maintained' to `unmaintained' if there is no Current Maintainer, or the person stated to be Current Maintainer of the work cannot be reached through the indicated means of communication for a period of six months, and there are no other significant signs of active maintenance.

    如果没有当前维护者,或者是有人声明要做作品的当前维护者,然后在 6 个月的时间内,他并没有通过指 定的手段使得他能胜任,且没有其他显著的有效维护的迹象,则作品从``维护''状态进入``未维护''状态。

    % You can become the Current Maintainer of the Work by agreement with any existing Current Maintainer to take over this role.

    通过与任何一位当前维护者达成一致,您就可以取代他的角色成为一名维护者。

    % If the Work is unmaintained, you can become the Current Maintainer of the Work through the following steps:

    如果作品处于未维护状态,您也可以成为作品的当前维护者,只需通过以下几个步骤:
    %
    \begin{enumerate}
        % \item Make a reasonable attempt to trace the Current Maintainer (and the Copyright Holder, if the two differ) through the means of an Internet or similar search.
        \item 通过互联网或类似的搜索手段做出一个合理的打算去跟随当前维护者(和版权持有人,如果两者不是同一人)。
        % \item If this search is successful, then enquire whether the Work is still maintained.
        \item 如果这个搜索是成功的,然后询问作品是否仍处于维护状态。
            %
            \begin{enumerate}
                % \item If it is being maintained, then ask the Current Maintainer to update their communication data within one month.
                \item 如果它正处于维护状态,然后向当前维护者要求更新他们一个月内的通讯数据.
                % \item\label{LPPL:item:intention} If the search is unsuccessful or no action to resume active maintenance is taken by the Current Maintainer, then announce within the pertinent community your intention to take over maintenance.  (If the Work is a \LaTeX{} work, this could be done, for example, by posting to \texttt{comp.text.tex}.)
                \item\label{LPPL:item:intention} 如果搜索不成功,或当前维护者没采取行动以恢复有效维护,那么相关的社区宣布您接管维护。(如果作品是一个 \LaTeX{} 作品,可以这样做,例如,张贴到 \texttt{comp.text.tex}。)
            \end{enumerate}
            %
        \item {}
            \begin{enumerate}
                % \item If the Current Maintainer is reachable and agrees to pass maintenance of the Work to you, then this takes effect immediately upon announcement.
                \item 如果当前维护者做到了,并同意作品的维护交给您,那么公布后将立即生效。
                % \item\label{LPPL:item:announce} If the Current Maintainer is not reachable and the Copyright Holder agrees that maintenance of the Work be passed to you, then this takes effect immediately upon announcement.
                \item\label{LPPL:item:announce} 如果当前维护者做不到,并且版权持有人同意把作品的维护交给您,那么公布后将立即生效。
            \end{enumerate}
            %
        % \item\label{LPPL:item:change} If you make an `intention announcement' as described in~\ref{LPPL:item:intention} above and after three months your intention is challenged neither by the Current Maintainer nor by the Copyright Holder nor by other people, then you may arrange for the Work to be changed so as to name you as the (new) Current Maintainer.
        \item\label{LPPL:item:change} 如果您做一个像在上面 \ref{LPPL:item:intention} 所述的那样的``意向性声明'',三个月后您的意向不被当前维护者反驳,也不被版权持有人或其它人反驳,那么,您可以改编作品,从而您成了(新的)当前维护者。
        % \item If the previously unreachable Current Maintainer becomes reachable once more within three months of a change completed under the terms of~\ref{LPPL:item:announce} or~\ref{LPPL:item:change}, then that Current Maintainer must become or remain the Current Maintainer upon request provided they then update their communication data within one month.
        \item 如果在 \ref{LPPL:item:announce} 或 \ref{LPPL:item:change} 的条件下 3 个月内完成了某个更改,以前的未达当前维护者再一次成为可达当前维护者,那么,那个当前维护 者必须成为或保持为上述的当前维护者,要求提供他们一个月内更新的通讯数据。
    \end{enumerate}
    %
    % A change in the Current Maintainer does not, of itself, alter the fact that the Work is distributed under the \LPPL\ license.
    %
    当前维护者的变化,本身并不改变在 \LPPL\ 许可证下这项工作的分发的事实。

    % If you become the Current Maintainer of the Work, you should immediately provide, within the Work, a prominent and unambiguous statement of your status as Current Maintainer.  You should also announce your new status to the same pertinent community as in~\ref{LPPL:item:intention} above.

    如果您成为作品的当前维护者,您应该立即提供,在作品中,突出自己的地位和毫不含糊的声明,您也应该向有关社区宣布新的状态,像上面在 \ref{LPPL:item:intention} 所述。


    % \LPPLsubsection{Whether and How to Distribute Works under This License}
    \LPPLsubsection{该许可证允许下能否和如何分发}
    \label{LPPL:Distribute}

    % This section contains important instructions, examples, and recommendations for authors who are considering distributing their works under this license.  These authors are addressed as `you' in this section.

    本节为那些正在考虑在这个许可证下分发他们作品的作者提供了重要指示、例子和建议。这些作者在本节中以``您''的形式出现。

    % \LPPLsubsubsection{Choosing This License or Another License}
    \LPPLsubsubsection{选择这个或其它许可证}
    \label{LPPL:Choosing}

    % If for any part of your work you want or need to use \emph{distribution} conditions that differ significantly from those in this license, then do not refer to this license anywhere in your work but, instead, distribute your work under a different license. You may use the text of this license as a model for your own license, but your license should not refer to the \LPPL\ or otherwise give the impression that your work is distributed under the \LPPL.

    如查您想或需要在您作品的任何一部分使用与该许可证显著不同的\emph{分发}条件,那么,请勿在您作品的任何地方提及该许可证,而应在不同的许可证下分发您的作品。您可以将该许 可证的文本作为您的许可证的模板,但您的许可证不应该提及 \LPPL\ ,否则会给人您的作品在 \LPPL\ 下分发的印象。

    % The document \LPPLdocfile{modguide} in the base \LaTeX\ distribution explains the motivation behind the conditions of this license.  It explains, for example, why distributing \LaTeX\ under the \textsc{gnu} General Public License (\textsc{gpl}) was considered inappropriate.  Even if your work is unrelated to \LaTeX, the discussion in \LPPLdocfile{modguide} may still be relevant, and authors intending to distribute their works under any license are encouraged to read it.

    基本 \LaTeX\ 分发中的 \LPPLdocfile{modguide} 文档解释了这个许可证条件后面的动机。它解释说,例如,为什么 gnu 通用公共许可证(\textsc{gpl})下分发 \LaTeX\ 被认为是不恰当的。即使您的作品与 \LaTeX\ 无关,而在 \LPPLdocfile{modguide} 里的讨论仍可能是相关的,鼓励那些想在某些许可证下分发自己作品的作 者去读一下。


    % \LPPLsubsubsection{A Recommendation on Modification Without Distribution}
    \LPPLsubsubsection{不进行分发时的修改的建议}
    \label{LPPL:WithoutDistribution}

    % It is wise never to modify a component of the Work, even for your own personal use, without also meeting the above conditions for distributing the modified component.  While you might intend that such modifications will never be distributed, often this will happen by accident -- you may forget that you have modified that component; or it may not occur to you when allowing others to access the modified version that you are thus distributing it and violating the conditions of this license in ways that could have legal implications and, worse, cause problems for the community. It is therefore usually in your best interest to keep your copy of the Work identical with the public one.  Many works provide ways to control the behavior of that work without altering any of its licensed components.

    不去修改作品的任何部分是明智的,即使是您自己用的,而且分发修改部分没有满足上述条件时也不要修 改。尽管您可能希望永远不会分发这样的修改,但这还是常常发生 – 您也许忘记曾经修改过,或者其他人进入 您修改过的版本,您这样分发并且违反了本许可证的条件,从而可能引起法律纠纷,更糟糕的是,引发社会问 题。因此,通常在最有利的情况下保持您的作品副本与公开发布的一样。许多作品在没有改变任何得到许可的 部分的情况下提供一些方式来控制那种行为。


    % \LPPLsubsubsection{How to Use This License}
    \LPPLsubsubsection{如何使用该许可证}
    \label{LPPL:HowTo}

    % To use this license, place in each of the components of your work both an explicit copyright notice including your name and the year the work was authored and/or last substantially modified.  Include also a statement that the distribution and/or modification of that component is constrained by the conditions in this license.

    要使用此许可证,可以在您作品的每一个组成部分里放置一个明确的版权声明,包括您的姓名和作品的撰写和/或最后大幅修改的年份。还包括一个受该许可证约束的分发和/或修改该组件的声明。

    % Here is an example of such a notice and statement:
    
    下面是一个这样的注意和声明的例子:
    %
\begin{verbatim}
  %% pig.dtx
  %% Copyright 2005 M. Y. Name
  %
  % This work may be distributed and/or modified under the
  % conditions of the LaTeX Project Public License, either version 1.3
  % of this license or (at your option) any later version.
  % The latest version of this license is in
  %   http://www.latex-project.org/lppl.txt
  % and version 1.3 or later is part of all distributions of LaTeX
  % version 2005/12/01 or later.
  %
  % This work has the LPPL maintenance status `maintained'.
  %
  % The Current Maintainer of this work is M. Y. Name.
  %
  % This work consists of the files pig.dtx and pig.ins
  % and the derived file pig.sty.
\end{verbatim}

    % Given such a notice and statement in a file, the conditions given in this license document would apply, with the `Work' referring to the three files `\LPPLfile{pig.dtx}', `\LPPLfile{pig.ins}', and `\LPPLfile{pig.sty}' (the last being generated from `\LPPLfile{pig.dtx}' using `\LPPLfile{pig.ins}'), the `Base Interpreter' referring to any `\LaTeX-Format', and both `Copyright Holder' and `Current Maintainer' referring to the person `M. Y. Name'.

    在一个文件中给出这样的一个注意和声明,将适应该许可证的条件,``Work''适应三个文件即``\LPPLfile{pig.dtx}''、``\LPPLfile{pig.dtx}''、``\LPPLfile{pig.sty}''(最后这个文件产生于用``\LPPLfile{pig.ins}''产生的``\LPPLfile{pig.dtx'}'),``基本解释程序''适应任何``\LaTeX-格式'',``版权持有人''和``当前维护者''适应``M.Y.Name''。

    % If you do not want the Maintenance section of \LPPL\ to apply to your Work, change `maintained' above into `author-maintained'. However, we recommend that you use `maintained' as the Maintenance section was added in order to ensure that your Work remains useful to the community even when you can no longer maintain and support it yourself.

    如果您不希望 \LPPL\ 的维护这一节适应于您的作品,请将上述的``维护''更改为``作者维护''。然而,我们建议您使用``维护'',因为添加维护这一节就是为了确保您的作品对社区有用,即使您自已都不再维护并支持您的作品。


    % \LPPLsubsubsection{Derived Works That Are Not Replacements}
    \LPPLsubsubsection{不可替代的衍生作品}
    \label{LPPL:NotReplacements}

    % Several clauses of the \LPPL\ specify means to provide reliability and stability for the user community. They therefore concern themselves with the case that a Derived Work is intended to be used as a (compatible or incompatible) replacement of the original Work. If this is not the case (e.g., if a few lines of code are reused for a completely different task), then clauses 6b and 6d shall not apply.

    \LPPL\ 所指定的若干条款为用户社区提供可靠性和稳定性。因此,他们关心这种情形,即衍生作品是用作初始作品的(合适的或不合适的)替代品。如果是这种情形并非如此(例如,如果一个完全不同的任务中重复使用几行代码),第 6b 和 6d 的条款不适应。

    % \LPPLsubsubsection{Important Recommendations}
    \LPPLsubsubsection{重要的建议}
    \label{LPPL:Recommendations}

    % \LPPLparagraph{Defining What Constitutes the Work}
    \LPPLparagraph{定义是什么构成了作品}

    % The \LPPL\ requires that distributions of the Work contain all the files of the Work.  It is therefore important that you provide a way for the licensee to determine which files constitute the Work.  This could, for example, be achieved by explicitly listing all the files of the Work near the copyright notice of each file or by using a line such as:

    \LPPL\ 要求作品的分发需包含作品的全部文件。因此,您提供给许可用以证判断哪些文件构成作品的方式是非常重要的,例如,通过明确列出每个文件的版权声明或在那里使用诸如这样的一行字:
    %
\begin{verbatim}
    % This work consists of all files listed in manifest.txt.
\end{verbatim}
    %
    % in that place.  In the absence of an unequivocal list it might be impossible for the licensee to determine what is considered by you to comprise the Work and, in such a case, the licensee would be entitled to make reasonable conjectures as to which files comprise the Work.

    在没有一个明确的清单的情况下,许可证不可能判定哪些文件组成了作品,在这种情况下,许可证将有权作出合理的猜测哪些文件组成了作品。
\end{LPPLicense}


% \subsection{GNU Free Documentation License, Version 1.2, November 2002}
\subsection{GNU自由文档许可证, V1.2, 2002年11月} % Translation from http://www.thebigfly.com/gnu/FDLv1.2/
\label{label_fdl}

%  \textbf{Copyright  2000,2001,2002  Free Software Foundation, Inc.}\par
%  51 Franklin St, Fifth Floor, Boston, MA  02110-1301  USA
%  \begin{quotation}
%    Everyone is allowed to distribute verbatim copies of this
%    license document, but modification of it is not allowed.
%  \end{quotation}

% \subsubsection{Preamble}
\subsubsection{导言}

% The purpose of this License is to make a manual, textbook, or other functional and useful document ``free'' in the sense of freedom: to assure everyone the effective freedom to copy and redistribute it, with or without modifying it, either commercially or noncommercially. Secondarily, this License preserves for the author and publisher a way to get credit for their work, while not being considered responsible for modifications made by others.

本授权的目的在于作为一种手册、教科书或其它的具有功能性的有用文件获得自由:确保每一个人都具有复制和重新发布它的有效自由,而不论是否作出修改,也不论其是否具有商业行为。其次,本授权保存了作者以及出版者由于他们的工作而 得到 名誉的方式,同时也不被认为应该对其他人所作出的修改而担负责任。

% This License is a kind of ``copyleft'', which means that derivative works of the document must themselves be free in the same sense.  It complements the GNU General Public License, which is a copyleft license designed for free software.

本授权是一种``公共版权'',这表示文件的衍生作品本身必须具有相同的自由涵义。它补足了 GNU 公共通用授权 —— 一种为了自由软件而设计的``公共版权''授权。

% We have designed this License in order to use it for manuals for free software, because free software needs free documentation: a free program should come with manuals providing the same freedoms that the software does.  But this License is not limited to software manuals; it can be used for any textual work, regardless of subject matter or whether it is published as a printed book.  We recommend this License principally for works whose purpose is instruction or reference.

我们设计了本授权是为了将它使用到自由软件的使用手册上,因为自由软件需要自由的文档:一种自由的程序应该提供与此软件具有相同的自由的使用手册。但是本授权并不被限制在软件使用手册的应用上;它可以被用于任何以文字作基础的作品,而不论其主题内容,或者它是否是一个被出版的印刷书籍。我们建议本授权主要应用在以使用说明或提供参考作为目的的作品上。


% \subsubsection{Applicability and definitions}
\subsubsection{效力与定义}

% This License applies to any manual or other work, in any medium, that contains a notice placed by the copyright holder saying it can be distributed under the terms of this License.  Such a notice grants a world-wide, royalty-free license, unlimited in duration, to use that work under the conditions stated herein.  The \textbf{``Document''}, below, refers to any such manual or work. Any member of the public is a licensee, and is addressed as \textbf{``you''}. You accept the license if you copy, modify or distribute the work in a way requiring permission under copyright law.

本授权的效力在于任何媒体中的任何的使用手册或其它作品,只要其中包含由版权所有人所指定的声明,说明它可以在本授权的条款下被发布。这样的一份声明提供了全球范围内的,免版税的和没有期限的许可,在此所陈述条件下使用那个作品。以下所称的``文件'',指的是任何像这样的使用手册或作品。公众中的任何成员都是被许可人,并且称作为``你''。如果你以一种需要在版权法下取得允许的方式进行复制、修改或发布作品,你就接受了这项许可。

% A \textbf{``Modified Version''} of the Document means any work containing the Document or a portion of it, either copied verbatim, or with modifications and/or translated into another language.

``修改版本''指的是任何包含文件或是它的其中一部份,不论是逐字的复制或是经过修正,或翻译成其它语言的任何作品。

% A \textbf{``Secondary Section''} is a named appendix or a front-matter section of the Document that deals exclusively with the relationship of the publishers or authors of the Document to the Document's overall subject (or to related matters) and contains nothing that could fall directly within that overall subject.  (Thus, if the Document is in part a textbook of mathematics, a Secondary Section may not explain any mathematics.)  The relationship could be a matter of historical connection with the subject or with related matters, or of legal, commercial, philosophical, ethical or political position regarding them.

``次要章节''是一个具名的附录,或是文件的本文之前内容的章节,专门用来处理文件的出版者或作者,与文件整体主题(或其它相关内容)的关系,并且不包含任何可以直接落入那个整体主题的内容。(因此,如果文件的部分内容是作为数学教科书,那么其次要章节就可以不用来解释任何数学。)它的关系可以是与主题相关的历史连接,或是与其相关的法律、商业、哲学、伦理道德或政治立场。

% The \textbf{``Invariant Sections''} are certain Secondary Sections whose titles are designated, as being those of Invariant Sections, in the notice that says that the Document is released under this License.  If a section does not fit the above definition of Secondary then it is not allowed to be designated as Invariant.  The Document may contain zero Invariant Sections.  If the Document does not identify any Invariant Sections then there are none.

``不变章节''是标题已被指定的某些次要章节,在一个声明了是以本授权加以发行的文件中,依此作为不变章节。如果一个章节并不符合上述有关于次要的定义时,则它并不允许被指定为不变。文件可以不包含不变章节。如果文件并没有指出任何不变章节,那么就是没有。

% The \textbf{``Cover Texts''} are certain short passages of text that are listed, as Front-Cover Texts or Back-Cover Texts, in the notice that says that the Document is released under this License.  A Front-Cover Text may be at most 5 words, and a Back-Cover Text may be at most 25 words.

封面文字''是某些被加以列出的简短文字段落,在一个声明了是以本授权加以发行的文件中,依此作为前封面文字或后封面文字。前封面文字最多可以包含 5 个单词,后封面文字最多可以包含 25 个单词。

% A \textbf{``Transparent''} copy of the Document means a machine-readable copy, represented in a format whose specification is available to the general public, that is suitable for revising the document straightforwardly with generic text editors or (for images composed of pixels) generic paint programs or (for drawings) some widely available drawing editor, and that is suitable for input to text formatters or for automatic translation to a variety of formats suitable for input to text formatters.  A copy made in an otherwise Transparent file format whose markup, or absence of markup, has been arranged to thwart or discourage subsequent modification by readers is not Transparent. An image format is not Transparent if used for any substantial amount of text.  A copy that is not ``Transparent'' is called \textbf{``Opaque''}.

文件的''透明''拷贝指的是一份机器可读的拷贝,它以一种一般公众可以取得其规格说明的格式来表现,适合于直接用一般文字编辑器、一般点阵图像程序用于由图元像素构成的影像或一些可以广泛取得的绘图程序用于由向量绘制的图形直接地进行修订;并且适合于输入到文字格式化程式,或是可以自动地转换到适合于输入到文字格式化程序的各种格式。一份以透明以外的档案格式所构成的拷贝,其标记或缺少标记,若是被安排成用来挫折或是打消读者进行其后续的修改,则此拷贝并非透明。一种影像格式,如果仅仅是用来充斥文本的资料量时,就不是透明的。一个不是透明的拷贝被称为混浊。

% Examples of suitable formats for Transparent copies include plain ASCII without markup, Texinfo input format, LaTeX input format, SGML or XML using a publicly available DTD, and standard-conforming simple HTML, PostScript or PDF designed for human modification.  Examples of transparent image formats include PNG, XCF and JPG.  Opaque formats include proprietary formats that can be read and edited only by proprietary word processors, SGML or XML for which the DTD and/or processing tools are not generally available, and the machine-generated HTML, PostScript or PDF produced by some word processors for output purposes only.

透明拷贝适合格式的例子包括有:没有标记的纯 ASCII、Texinfo 输入格式、LaTeX 输入格式、使用可以公开取得其 DTD 的 SGML 或 XML、合乎标准的简单 HTML、PostScript 或 PDF。透明影像格式的例子有 PNG、XCF 和 JPG 。混浊格式包括只能够以私人文书处理器阅读以及编辑的私人格式、DTD 以及或处理工具不能够一般地加以取得的 SGML 或 XML、以及由某些文书处理器只是为了输出的目的而做出的,由机器制作的 HTML、PostScript 或 PDF 。

% The \textbf{``Title Page''} means, for a printed book, the title page itself, plus such following pages as are needed to hold, legibly, the material this License requires to appear in the title page.  For works in formats which do not have any title page as such, ``Title Page'' means the text near the most prominent appearance of the work's title, preceding the beginning of the body of the text.

``标题页''对一本印刷书籍来说,指的是标题页本身,以及所需要用来容纳本授权必须出现在标题页的易读内容的,如此的接续数页。对于并没有任何如此页面的作品的某些格式,``标题页''指的是本文主体开始之前作品标题最显着位置的文字。

% A section \textbf{``Entitled XYZ''} means a named subunit of the Document whose title either is precisely XYZ or contains XYZ in parentheses following text that translates XYZ in another language.  (Here XYZ stands for a specific section name mentioned below, such as \textbf{``Acknowledgements''}, \textbf{``Dedications''}, \textbf{``Endorsements''}, or \textbf{``History''}.) To \textbf{``Preserve the Title''} of such a section when you modify the Document means that it remains a section ``Entitled XYZ'' according to this definition.

一个标题为\textbf{``XYZ''}的章节指的是文件的一个具名的次要单元,其标题精确地为 \textbf{``XYZ''} 或是将 \textbf{``XYZ''} 包含在跟着翻译为其它语言的 \textbf{``XYZ''} 文字后面的括号内 -- 这里 \textbf{``XYZ''} 代表的是名称于下提及的特定章节,像是感谢、贡献、背书或历史。当你修改文件时,给像这样子的章节保存其标题指的是,它保持为一个根据这个定义的标题为 \textbf{``XYZ''}

% The Document may include Warranty Disclaimers next to the notice which states that this License applies to the Document.  These Warranty Disclaimers are considered to be included by reference in this License, but only as regards disclaiming warranties: any other implication that these Warranty Disclaimers may have is void and has no effect on the meaning of this License.

文件可以在用来陈述本授权效力及于文件的声明后,包括担保放弃。这些担保放弃被考虑为以提及的方式,包括在本授权中,但是只被看作为放弃担保之用:任何这些担保放弃可能会有的其它暗示都是无效的,并且也对本授权的含义没有影响。

以下是有关复制、发布及修改的明确条款及条件。


% \subsubsection{Verbatim Copying}
\subsubsection{逐字的复制}

% You may copy and distribute the Document in any medium, either commercially or noncommercially, provided that this License, the copyright notices, and the license notice saying this License applies to the Document are reproduced in all copies, and that you add no other conditions whatsoever to those of this License.  You may not use technical measures to obstruct or control the reading or further copying of the copies you make or distribute.  However, you may accept compensation in exchange for copies.  If you distribute a large enough number of copies you must also follow the conditions in section~3.

你可以复制或发布文件于任何媒体,而不论其是否具有商业买卖行为,其条件为具有本授权、版权声明和许可声明,说明本授权效力于文件的所有重制拷贝,并且你没有增加任何其它条件到本授权的条件中。你不可以使用技术手段,来妨碍或控制你所制作或发布的拷贝阅读或进一步的发布。然而,你可以接受补偿以作为拷贝的交换。如果你发布了数量足够大的拷贝,你也必须遵循第三条的条件。

% You may also lend copies, under the same conditions stated above, and you may publicly display copies.

你也可以在上述的相同条件下借出拷贝,并且你可以公开地陈列拷贝。

% \subsubsection{Copying in Quantity}

\subsubsection{大量地复制}

% If you publish printed copies (or copies in media that commonly have printed covers) of the Document, numbering more than 100, and the Document's license notice requires Cover Texts, you must enclose the copies in covers that carry, clearly and legibly, all these Cover Texts: Front-Cover Texts on the front cover, and Back-Cover Texts on the back cover.  Both covers must also clearly and legibly identify you as the publisher of these copies.  The front cover must present the full title with all words of the title equally prominent and visible.  You may add other material on the covers in addition. Copying with changes limited to the covers, as long as they preserve the title of the Document and satisfy these conditions, can be treated as verbatim copying in other respects.

如果你出版文件的印刷拷贝或者通常具有印刷封面的媒体的拷贝,数量上超过一百个单位,而且文件许可声明要求有封面文字,那么你必须将这些拷贝附上清楚且易读的文字:前封面文字于前封面上、后封面文字于后封面上。这两种封面必须清楚易读地辨认出,你是这些拷贝的出版者。前封面文字必须展示完整的标题,而标题的文字应当同等地显著可见。你可以增加额外的内容于封面上。仅在封面作出改变的复制,只要它们保存了文件的标题,并且满足了这些条件,可以在其它方面被看作为逐字的复制。

% If the required texts for either cover are too voluminous to fit legibly, you should put the first ones listed (as many as fit reasonably) on the actual cover, and continue the rest onto adjacent pages.

如果对于任意一个封面所需要的文字,数量过于庞大以至于不能符合易读的原则,你应该在实际封面的最前面列出所能符合易读原则的内容,然后将剩下的接续在相邻的页面。

% If you publish or distribute Opaque copies of the Document numbering more than 100, you must either include a machine-readable Transparent copy along with each Opaque copy, or state in or with each Opaque copy a computer-network location from which the general network-using public has access to download using public-standard network protocols a complete Transparent copy of the Document, free of added material. If you use the latter option, you must take reasonably prudent steps, when you begin distribution of Opaque copies in quantity, to ensure that this Transparent copy will remain thus accessible at the stated location until at least one year after the last time you distribute an Opaque copy (directly or through your agents or retailers) of that edition to the public.

如果你出版或发布数量超过一百个单位文件的混浊拷贝,你必须与此混浊拷贝一起包含一份机器可读的透明拷贝,或是与一份混浊拷贝一起或其陈述一个电脑网络位址,使一般的网络使用公众具有存取权,可以使用公开标准的网络协定,下载一份文件的完全透明拷贝,此拷贝中并且没有增加额外的内容。如果你使用后面的选项,当你开始大量地发布混浊拷贝时,你必须采取合理的审慎步骤,以保证这个透明拷贝将会在发布的一开始就保持可供存取,直到你最后一次发布那个发行版的一份混浊拷贝给公众后,至少一年为止。以保证这个透明拷贝,将会在所陈述的位址保持如此的可存取性,直到你最后一次发布那个发行版直接或经由你的代理商或零售商的一份混浊拷贝给公众后,至少一年为止。

% It is requested, but not required, that you contact the authors of the Document well before redistributing any large number of copies, to give them a chance to provide you with an updated version of the Document.

你被要求,但不是必须,在重新发布任何大数量的拷贝之前与文件的作者联络,给予他们提供你一份文件的更新版本的机会。


% \subsubsection{Modifications}
\subsubsection{修改}

% You may copy and distribute a Modified Version of the Document under the conditions of sections 2 and 3 above, provided that you release the Modified Version under precisely this License, with the Modified Version filling the role of the Document, thus licensing distribution and modification of the Modified Version to whoever possesses a copy of it.  In addition, you must do these things in the Modified Version:

你可以在上述第二条和第三条的条件下,复制和发布文件的修改版本,其条件为你要精确地在本授权下发布修改版本,且修改版本补足了文件的角色,从而允许修改版本的发布和修改权利给任何拥有它拷贝的人。另外,你必须在修改版本中做这些事:
%
\begin{itemize}
    % \item[A.] Use in the Title Page (and on the covers, if any) a title distinct from that of the Document, and from those of previous versions (which should, if there were any, be listed in the History section of the Document).  You may use the same title as a previous version if the original publisher of that version gives permission.
    \item[A.] 在标题页或在封面上使用,如果有与先前版本不同的文件,应该被列在文件的历史章节不同的标题。如果版本的原始出版者允许,你可以使用与某一个先前版本相同的标题。
    % \item[B.] List on the Title Page, as authors, one or more persons or entities responsible for authorship of the modifications in the Modified Version, together with at least five of the principal authors of the Document (all of its principal authors, if it has fewer than five), unless they release you from this requirement.
    \item[B.] 在修改版本的标题页上列出担负作者权的一个或多个人或实体作为作者,并且列出至少五位文件的主要作者。如果少于五位,则列出全部的主要作者,除非他们免除了你这个要求。
    % \item[C.] State on the Title page the name of the publisher of the Modified Version, as the publisher.
    \item[C.] 在标题页陈述修改版本的出版者的名称作为出版者。 
    % \item[D.] Preserve all the copyright notices of the Document.
    \item[D.] 保存文件的所有版权声明。
    % \item[E.] Add an appropriate copyright notice for your modifications adjacent to the other copyright notices.
    \item[E.] 为你的修改增加一个与其它版权声明相邻的适当的版权声明。 
    % \item[F.] Include, immediately after the copyright notices, a license notice giving the public permission to use the Modified Version under the terms of this License, in the form shown in the Addendum below.
    \item[F.] 在版权声明后面,以授权附录所显示的形式,包括一个给予公众在本授权条款下使用修改版本的许可声明。
    % \item[G.] Preserve in that license notice the full lists of Invariant Sections and required Cover Texts given in the Document's license notice.
    \item[G.] 在那个许可声明中保存恒常章节和文件许可声明中必要封面文字的全部列表。 
    % \item[H.] Include an unaltered copy of this License.
    \item[H.] 包括一个未被改变的本授权的副本。
    % \item[I.] Preserve the section Entitled ``History'', Preserve its Title, and add to it an item stating at least the title, year, new authors, and publisher of the Modified Version as given on the Title Page.  If there is no section Entitled ``History'' in the Document, create one stating the title, year, authors, and publisher of the Document as given on its Title Page, then add an item describing the Modified Version as stated in the previous sentence.
    \item[I.] 保存标题为``历史''的章节和其标题,并且增加一项至少陈述如同在标题页中所给的修改版本的标题、年份、新作者和出版者。如果在文件中没有标题为``历史''的章节,则制作出一个陈述如同在它的标题页中所给的文件的标题、年份、新作者和出版者,然后增加一项描述修改版本如前面句子所陈述的情形。
    % \item[J.] Preserve the network location, if any, given in the Document for public access to a Transparent copy of the Document, and likewise the network locations given in the Document for previous versions it was based on.  These may be placed in the ``History'' section. You may omit a network location for a work that was published at least four years before the Document itself, or if the original publisher of the version it refers to gives permission.
    \item[J.] 如果有的话,保存在文件中为了给公众存取文件的透明拷贝,而给予的网络位址,以及同样地在文件中为了它所根据的先前版本,而给予的网络位址。这些可以被放置在历史章节。你可以省略一个在文件本身之前,已经至少出版了四年的作品的网络位址,或是如果它所参照的那个版本的原始出版者给予允许的情形下也可以省略它。
    % \item[K.] For any section Entitled ``Acknowledgements'' or ``Dedications'', Preserve the Title of the section, and preserve in the section all the substance and tone of each of the contributor acknowledgements and/or dedications given therein.
    \item[K.] 在任何标题为感谢或贡献的章节,保存章节的标题,并且在那章节保存到那时候为止,每一个贡献者的感谢以及或贡献的所有声色。
    % \item[L.] Preserve all the Invariant Sections of the Document, unaltered in their text and in their titles.  Section numbers or the equivalent are not considered part of the section titles.
    \item[L.] 保存文件的所有恒常章节,于其文字以及标题皆不得变更。章节号码或其同等物并不被认为是章节标题的一部份。
    % \item[M.] Delete any section Entitled ``Endorsements''.  Such a section may not be included in the Modified Version.
    \item[M.] 删除任何标题为背书的章节。这样子的章节不可以被包括在修改版本中。
    % \item[N.] Do not retitle any existing section to be Entitled ``Endorsements'' or to conflict in title with any Invariant Section.
    \item[N.] 不要重新命名任何现存的章节,而使其标题为背书,或造成与任何恒常章节相冲突的标题。
    % \item[O.] Preserve any Warranty Disclaimers.
    \item[O.] 保存任何的担保放弃。
\end{itemize}

% If the Modified Version includes new front-matter sections or appendices that qualify as Secondary Sections and contain no material copied from the Document, you may at your option designate some or all of these sections as invariant. To do this, add their titles to the list of Invariant Sections in the Modified Version's license notice. These titles must be distinct from any other section titles.

如果修改版本包括新的本文之前内容的章节,或合乎作为次要章节的附录,并且没有包含复制自文件的内容,则你具有选择可以指定一些或全部这些章节为恒常的。要这样做,将它们的标题增加到在修改版本许可声明中的恒常章节列表中。这些标题必须可以和任何其它章节标题加以区别。

% You may add a section Entitled ``Endorsements'', provided it contains nothing but endorsements of your Modified Version by various parties--for example, statements of peer review or that the text has been approved by an organization as the authoritative definition of a standard.

你可以增加一个标题为``背书''的章节,其条件为它仅只包含由许多团体所提供的你的修改版本的背书 -- 举例来说,同侪评审的说明,或本文已经被一个机构认可为一个标准的权威定义。

% You may add a passage of up to five words as a Front-Cover Text, and a passage of up to 25 words as a Back-Cover Text, to the end of the list of Cover Texts in the Modified Version.  Only one passage of Front-Cover Text and one of Back-Cover Text may be added by (or through arrangements made by) any one entity.  If the Document already includes a cover text for the same cover, previously added by you or by arrangement made by the same entity you are acting on behalf of, you may not add another; but you may replace the old one, on explicit permission from the previous publisher that added the old one.

你可以增加一个作为前封面文字的最多五个字的段落,以及一个作为后封面文字的最多二十五个字的段落,到修改版本的封面文字列表的后面。前封面文字和后封面文字都只能有一个段落,可以经由任何一个实体,或经由任何一个实体所作出的安排而被加入。如果文件已经在同样的封面包括了封面文字 -- 先前由你或由你所代表的相同实体所作出的安排而加入,则你不可以增加另外一个;但是你可以在先前出版者的明确允许下替换掉旧的。

% The author(s) and publisher(s) of the Document do not by this License give permission to use their names for publicity for or to assert or imply endorsement of any Modified Version.

文件的作者和出版者并不由此授权,而给予允许使用他们的名字以为了或经由声称或暗示任何修改版本背书为自己所应得的方式而获得名声的权利。

% \subsubsection{Combining Documents}
\subsubsection{组合文件}

% You may combine the Document with other documents released under this License, under the terms defined in section 4 above for modified versions, provided that you include in the combination all of the Invariant Sections of all of the original documents, unmodified, and list them all as Invariant Sections of your combined work in its license notice, and that you preserve all their Warranty Disclaimers.

你可以在上述第四条的条款中对于修改版本的定义之下,将文件与其它在本授权下发行的文件组合起来,其条件是你要在组合品中,包括所有原始文件的所有恒常章节,不做修改,同时在组合作品的许可声明中将它们全部列为恒常章节,并且你要保存它们所有的担保放弃。

% The combined work need only contain one copy of this License, and multiple identical Invariant Sections may be replaced with a single copy.  If there are multiple Invariant Sections with the same name but different contents, make the title of each such section unique by adding at the end of it, in parentheses, the name of the original author or publisher of that section if known, or else a unique number. Make the same adjustment to the section titles in the list of Invariant Sections in the license notice of the combined work. 

组合作品只需包含本授权的一份副本,并且重复的恒常章节可以仅以单一个拷贝来取代。如果名称重复但内容不同的恒常章节,则将任此章节的标题,以在它的后面增加的方式加以独特化,如果已知的话,于括号中指出那个章节的原始作者或出版者的名称,或是指定一个独特的号码。在此组合作品许可声明中恒常章节的列表中,对其章节标题也作出相同的调整。

% In the combination, you must combine any sections Entitled ``History'' in the various original documents, forming one section Entitled ``History''; likewise combine any sections Entitled ``Acknowledgements'', and any sections Entitled ``Dedications''.  You must delete all sections Entitled ``Endorsements''.

在组合品中,你必须组合在不同原始文件中,标题为历史的任何章节,形成一个标题为历史的章节;同样组合任意标题为感谢或贡献的章节。你必须删除标题为背书的所有章节。

% \subsubsection{Collection of Documents}
\subsubsection{文件的收集}

% You may make a collection consisting of the Document and other documents released under this License, and replace the individual copies of this License in the various documents with a single copy that is included in the collection, provided that you follow the rules of this License for verbatim copying of each of the documents in all other respects.

你可以制作含有文件以及其它以本授权发行文件的收集品,并且将本授权对不同文件中的个别副本,以单一个包括在收集品的副本取代,其条件是你要遵循在其它方面,给予一个文件逐字复制的允许本授权的规则。

% You may extract a single document from such a collection, and distribute it individually under this License, provided you insert a copy of this License into the extracted document, and follow this License in all other respects regarding verbatim copying of that document.

你可以从这样的一个收集品中抽取出一份单一的文件,并且在本授权下将它单独地发布,其条件是你要在抽取出的文件中插入本授权的一份副本,并且在关于那份文件的逐字的复制的所有其它方面,遵循本授权。


% \subsubsection{Aggregating with independent Works}
\subsubsection{独立作品的聚集}

% A compilation of the Document or its derivatives with other separate and independent documents or works, in or on a volume of a storage or distribution medium, is called an ``aggregate'' if the copyright resulting from the compilation is not used to limit the legal rights of the compilation's users beyond what the individual works permit. When the Document is included in an aggregate, this License does not apply to the other works in the aggregate which are not themselves derivative works of the Document.

一个文件的编辑物,其中或附加于储存物或发布媒体的一册的,具有其它分别且独立的文件或作品的衍生品,如果经由编辑而产生的版权,并没有用来限制此编辑物使用者的法律权力,而超过了个别的作品所允许的,则被称为一个聚集品。当文件中包括一个聚集品,本授权的效力并不仅在于此聚集品中的,于其本身并非文件的衍生作品的其它作品。

% If the Cover Text requirement of section 3 is applicable to these copies of the Document, then if the Document is less than one half of the entire aggregate, the Document's Cover Texts may be placed on covers that bracket the Document within the aggregate, or the electronic equivalent of covers if the Document is in electronic form. Otherwise they must appear on printed covers that bracket the whole aggregate.

如果第三条的封面文字要求效力于这些文件的拷贝,并且文件的篇幅少于整个聚集品的一半,则文件的封面文字可以被放在只围绕着文件,并于聚集品内部的封面或是电子的封面同等物上,如果文件是以电子的形式出现的话。否则它们必须出现在绕着整个聚集品的印刷封面上。


% \subsubsection{Translation}
\subsubsection{翻译}

% Translation is considered a kind of modification, so you may distribute translations of the Document under the terms of section 4. Replacing Invariant Sections with translations requires special permission from their copyright holders, but you may include translations of some or all Invariant Sections in addition to the original versions of these Invariant Sections.  You may include a translation of this License, and all the license notices in the Document, and any Warranty Disclaimers, provided that you also include the original English version of this License and the original versions of those notices and disclaimers.  In case of a disagreement between the translation and the original version of this License or a notice or disclaimer, the original version will prevail.

翻译被认为一种修改,因此你可以在第四条的条款下发布文件的翻译。用翻译更换恒常章节需要取得版权所有者的特别允许,但是你可以包括部份或所有恒常章节的翻译,使其附加到这些恒常章节的原始版本之中。你可以包括本授权、文件中的所有许可声明和任何的担保放弃的翻译,其条件为你也必须包括本授权的原始英文版本,以及这些声明与放弃的原始版本。如果发生翻译与本授权、声明或放弃的原始版本有任何的不同意时,将以原始版本为准。

% If a section in the Document is Entitled ``Acknowledgements'', ``Dedications'', or ``History'', the requirement (section 4) to Preserve its Title (section 1) will typically require changing the actual title.

如果在文件中的章节被标题为``感谢''、``贡献''或``历史'',则保存标题第一条的必要条件第四条,典型上将会需要去更动实际的标题。

% \subsubsection{Termination}
\subsubsection{终止}

% You may not copy, modify, sublicense, or distribute the Document except as expressly provided for under this License.  Any other attempt to copy, modify, sublicense or distribute the Document is void, and will automatically terminate your rights under this License.  However, parties who have received copies, or rights, from you under this License will not have their licenses terminated so long as such parties remain in full compliance.

你不可以复制、修改、在本授权下再设定额外条件的次授权或发布文件,除非明白地表示是在本授权所规范的条件下进行。任何其它的复制、修改、在本授权下再设定额外条件的次授权、或发布文件的意图都是无效的,并且将会自动地终止你在本授权下所被保障的权利。然而,你在本授权下收到拷贝及权利的团体,只要团体完全遵守本授权的条件,则他们所获得的许可将不会被终止。


% \subsubsection{Future Revisions of this License}
\subsubsection{本授权的未来改版}

% The Free Software Foundation may publish new, revised versions of the GNU Free Documentation License from time to time.  Such new versions will be similar in spirit to the present version, but may differ in detail to address new problems or concerns.  See http://www.gnu.org/copyleft/.

自由软件基金会可能偶尔会出版自由文件授权的新修订过的版本。这种新版本在精神上将会类似于现在的版本,但在细节上可能会有不同,以对应新的问题或相关的事。请见 \url{http://www.gnu.org/copyleft/}。

% Each version of the License is given a distinguishing version number. If the Document specifies that a particular numbered version of this License ``or any later version'' applies to it, you have the option of following the terms and conditions either of that specified version or of any later version that has been published (not as a draft) by the Free Software Foundation.  If the Document does not specify a version number of this License, you may choose any version ever published (not as a draft) by the Free Software Foundation.

本授权的任何版本都被指定一个可供区别的版本号码。如果文件指定一个效力于它的特定号码版本的本授权或任何以后的版本,你就具有选择遵循指定的版本,或任何已经由自由软件基金会出版的后来版本并且不是草稿的条款和条件。如果文件并没有指定一个本授权的版本号码,你就可以选择任何一个曾由自由软件基金会所出版不是草稿的版本。


% \subsubsection{Addendum: How to use this License for your documents}
\subsubsection{授权附录:如何使用本授权用于你的文件}

% To use this License in a document you have written, include a copy of the License in the document and put the following copyright and license notices just after the title page:

为使用本授权成为你撰写成的一份文件,必须在文件中包括本授权的一份复本,以及标题页的后面包括许可声明:

\bigskip
\begin{quote}
    Copyright \copyright \textsc{year your name}.
    Permission is granted to copy, distribute and/or modify this document under the terms of the GNU Free Documentation License, Version 1.2 or any later version published by the Free Software Foundation; with no Invariant Sections, no Front-Cover Texts, and no Back-Cover Texts. A copy of the license is included in the section entitled ``GNU Free Documentation License''.
\end{quote}
\bigskip

% If you have Invariant Sections, Front-Cover Texts and Back-Cover Texts, replace the ``with \dots\ Texts''. line with this:

如果你有恒常章节、前封面文字和后封面文字,请将with... Texts这一行以这些文本取代:

\bigskip
\begin{quote}
    with the Invariant Sections being \textsc{list their titles}, with the Front-Cover Texts being \textsc{list}, and with the Back-Cover Texts being \textsc{list}.
\end{quote}
\bigskip

% If you have Invariant Sections without Cover Texts, or some other combination of the three, merge those two alternatives to suit the situation.

如果你有不具封面文字的恒常章节,或一些其它这三者的组合,将可选择的二项合并以符合实际情形。

% If your document contains nontrivial examples of program code, we recommend releasing these examples in parallel under your choice of free software license, such as the GNU General Public License, to permit their use in free software.

如果你的文件中包含有并非微不足道的程序码范例,我们建议这些范例平行地在你的自由软件授权选择下,比如以 GNU General Public License 的自由软件授权来发布,从而允许它们作为自由软件而使用。

\clearpage

%%% Local Variables:
%%% mode: latex
%%% TeX-master: "beameruserguide"
%%% End:
