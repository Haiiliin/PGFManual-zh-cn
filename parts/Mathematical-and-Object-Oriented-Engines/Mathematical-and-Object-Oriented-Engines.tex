\part{Mathematical and Object-Oriented Engines}

{\Large \emph{by Mark Wibrow and Till Tantau}}


\bigskip
\noindent
\pgfname\ comes with two useful engines: One for doing mathematics, one for
doing object-oriented programming. Both engines can be used independently of
the main \pgfname.

The job of the mathematical engine is to support mathematical operations like
addition, subtraction, multiplication and division, using both integers and
non-integers, but also functions such as square-roots, sine, cosine, and
generate pseudo-random numbers. Mostly, you will use the mathematical
facilities of \pgfname\ indirectly, namely when you write a coordinate like
|(5cm*3,6cm/4)|, but the mathematical engine can also be used independently of
\pgfname\ and \tikzname.

The job of the object-oriented engine is to support simple object-oriented
programming in \TeX. It allows the definition of \emph{classes} (without
inheritance), \emph{methods}, \emph{attributes} and \emph{objects}.

\vskip1cm
\begin{codeexample}[graphic=white]
\pgfmathsetseed{1}
\foreach \col in {black,red,green,blue}
{
  \begin{tikzpicture}[x=10pt,y=10pt,ultra thick,baseline,line cap=round]
    \coordinate (current point) at (0,0);
    \coordinate (old velocity) at (0,0);
    \coordinate (new velocity) at (rand,rand);

    \foreach \i in {0,1,...,100}
    {
      \draw[\col!\i] (current point)
      .. controls ++([scale=-1]old velocity) and
                  ++(new velocity) .. ++(rand,rand)
         coordinate (current point);
      \coordinate (old velocity) at (new velocity);
      \coordinate (new velocity) at (rand,rand);
    }
  \end{tikzpicture}
}
\end{codeexample}

% Copyright 2018 by Mark Wibrow and Till Tantau
%
% This file may be distributed and/or modified
%
% 1. under the LaTeX Project Public License and/or
% 2. under the GNU Free Documentation License.
%
% See the file doc/generic/pgf/licenses/LICENSE for more details.


\section{Design Principles}

\pgfname{} needs to perform many computations while typesetting a picture. For
this, \pgfname\ relies on a mathematical engine, which can also be used
independently of \pgfname, but which is distributed as part of the \pgfname\
package nevertheless. Basically, the engine provides a parsing mechanism
similar to the \calcname{} package so that expressions like |2*3cm+5cm| can be
parsed; but the \pgfname\ engine is more powerful and can be extended and
enhanced.

\pgfname{} provides enhanced functionality, which permits the parsing of
mathematical operations involving integers and non-integers with or without
units. Furthermore, various functions, including trigonometric functions and
random number generators can also be parsed (see
Section~\ref{pgfmath-parsing}). The \calcname{} macros |\setlength| and friends
have \pgfname{} versions which can parse these operations and functions (see
Section~\ref{pgfmath-registers}). Additionally, each operation and function has
an independent \pgfname{} command associated with it (see
Section~\ref{pgfmath-commands}), and can be accessed outside the parser.

The mathematical engine of \pgfname\ is implicitly used whenever you specify a
number or dimension in a higher-level macro. For instance, you can write
|\pgfpoint{2cm+4cm/2}{3cm*sin(30)}| or suchlike. However, the mathematical
engine can also be used independently of the \pgfname\ core, that is, you can
also just load it to get access to a mathematical parser.


\subsection{Loading the Mathematical Engine}

The mathematical engine of \pgfname\ is loaded automatically by \pgfname, but
if you wish to use the mathematical engine but you do not need \pgfname\
itself, you can load the following package:

\begin{package}{pgfmath}
    This command will load the mathematical engine of \pgfname, but not
    \pgfname{} itself. It defines commands like |\pgfmathparse|.
\end{package}


\subsection{Layers of the Mathematical Engine}

Like \pgfname\ itself, the mathematical engine is also structured into
different layers:
%
\begin{enumerate}
    \item The top layer, which you will typically use directly, provides the
        command |\pgfmathparse|. This command parses a mathematical expression
        and evaluates it.

        Additionally, the top layer also defines some additional functions
        similar to the macros of the |calc| package for setting dimensions and
        counters. These macros are just wrappers around the |\pgfmathparse|
        macro.

    \item The calculation layer provides macros for performing one specific
        computation like computing a reciprocal or a multiplication. The parser
        uses these macros for the actual computation.
    \item The implementation layer provides the actual implementations of the
        computations. These can be changed (and possibly be made more
        efficient) without affecting the higher layers.
\end{enumerate}


\subsection{Efficiency and Accuracy of the Mathematical Engine}

Currently, the mathematical algorithms are all implemented in \TeX. This poses
some intriguing programming challenges as \TeX{} is a language for typesetting,
rather than for general mathematics, and as with any programming language,
there is a trade-off between accuracy and efficiency. If you find the level of
accuracy insufficient for your purposes, you will have to replace the
algorithms in the implementation layer.

All the fancy mathematical ``bells-and-whistles'' that the parser provides,
come with an additional processing cost, and in some instances, such as simply
setting a length to |1cm|, with no other operations involved, the additional
processing time is undesirable. To overcome this, the following feature is
implemented: when no mathematical operations are required, an expression can be
preceded by |+|. This will bypass the parsing process and the assignment will
be orders of magnitude faster. This feature \emph{only} works with the macros
for setting registers described in Section~\ref{pgfmath-registers}.
%
\begin{codeexample}[code only]
\pgfmathsetlength\mydimen{1cm}  % parsed     : slower.
\pgfmathsetlength\mydimen{+1cm} % not parsed : much faster.
\end{codeexample}

% Copyright 2019 by Mark Wibrow
%
% This file may be distributed and/or modified
%
% 1. under the LaTeX Project Public License and/or
% 2. under the GNU Free Documentation License.
%
% See the file doc/generic/pgf/licenses/LICENSE for more details.


\section{Mathematical Expressions}
\label{pgfmath-syntax}

The easiest way of using \pgfname's mathematical engine is to provide a
mathematical expression given in familiar infix notation, for example,
|1cm+4*2cm/5.5| or |2*3+3*sin(30)|. This expression can be parsed by the
mathematical engine and the result can be placed in a dimension register, a
counter, or a macro.

It should be noted that all calculations must not exceed $\pm16383.99999$ at
\emph{any} point, because the underlying computations rely on \TeX{}
dimensions. This means that many of the underlying computations are necessarily
approximate and, in addition, not very fast. \TeX{} is, after all, a
typesetting language and not ideally suited to relatively advanced mathematical
operations. However, it is possible to change the computations as described in
Section~\ref{pgfmath-reimplement}.

In the present section, the high-level macros for parsing an expression are
explained first, then the syntax for expression is explained.


\subsection{Parsing Expressions}
\label{pgfmath-registers}
\label{pgfmath-parsing}

\subsubsection{Commands}

The \todosp{why 2 labels?}basic command for invoking the parser of \pgfname's
mathematical engine is the following:

\begin{command}{\pgfmathparse\marg{expression}}
    This macro parses \meta{expression} and returns the result without units in
    the macro |\pgfmathresult|.

    \example |\pgfmathparse{2pt+3.5pt}| will set |\pgfmathresult| to the text
    |5.5|.

    In the following, the special properties of this command are explained. The
    exact syntax of mathematical expressions is explained in Sections
    \ref{pgfmath-operators} and~\ref{pgfmath-functions}.
    %
    \begin{itemize}
        \item The result stored in the macro |\pgfmathresult| is a decimal
            \emph{without units}. This is true regardless of whether the
            \meta{expression} contains any unit specification. All numbers with
            units are converted to points first. See
            Section~\ref{pgfmath-units} for details on units.
        \item The parser will recognize \TeX{} registers and box dimensions, so
            |\mydimen|, |0.5\mydimen|, |\wd\mybox|, |0.5\dp\mybox|,
            |\mycount\mydimen| and so on can be parsed.
        \item The $\varepsilon$-TeX\ extensions |\dimexpr|, |\numexpr|,
            |\glueexpr|, and |\muexpr| are recognized and evaluated. The values
            they result in will be used in the further evaluation, as if you
            had put |\the| before them.
        \item Parenthesis can be used to change the order of the evaluation.
        \item Various functions are recognized, so it is possible to parse
            |sin(.5*pi r)*60|, which means ``the sine of $0.5$ times $\pi$
            radians, multiplied by 60''. The argument of functions can be any
            expression.
        \item Scientific notation in the form |1.234e+4| is recognized (but the
            restriction on the range of values still applies). The exponent
            symbol can be upper or lower case (i.e., |E| or |e|).
        \item An integer with a zero-prefix (excluding, of course zero itself),
            is interpreted as an octal number and is automatically converted to
            base 10.
        \item An integer with prefix |0x| or |0X| is interpreted as a
            hexadecimal number and is automatically converted to base 10.
            Alphabetic digits can be in uppercase or lowercase.
        \item An integer with prefix |0b| or |0B| is interpreted as a binary
            number and is automatically converted to base 10.
        \item An expression (or part of an expression) surrounded with double
            quotes (i.e., the character |"|) will not be evaluated. Obviously
            this should be used with great care.
    \end{itemize}
\end{command}

\begin{command}{\pgfmathqparse\marg{expression}}
    This macro is similar to |\pgfmathparse|: it parses \meta{expression} and
    returns the result in the macro |\pgfmathresult|. It differs in two
    respects. Firstly, |\pgfmathqparse| does not parse functions, scientific
    notation, the prefixes for binary octal, or hexadecimal numbers, nor does
    it accept the special use of |"|, |?| or |:| characters. Secondly, numbers
    in \meta{expression} \emph{must} specify a \TeX{} unit (except in such
    instances as |0.5\pgf@x|), which greatly simplifies the problem of parsing
    real numbers. As a result of these restrictions |\pgfmathqparse| is about
    twice as fast as |\pgfmathparse|. Note that the result will still be a
    number without units.
\end{command}

\begin{command}{\pgfmathpostparse}
    At the end of the parse this command is executed, allowing some custom
    action to be performed on the result of the parse. When this command is
    executed, the macro |\pgfmathresult| will hold the result of the parse (as
    always, without units). The result of the custom action should be used to
    redefine |\pgfmathresult| appropriately. By default, this command is
    equivalent to |\relax|. This differs from previous versions, where, if the
    parsed expression contained no units, the result of the parse was scaled
    according to the value in |\pgfmathresultunitscale| (which by default was
    |1|).

    This scaling can be  turned on again using:
    |\let\pgfmathpostparse=\pgfmathscaleresult|. Note, however that by scaling
    the result, the base conversion functions will not work, and the |"|
    character should not be used to quote parts of an expression.
\end{command}

Instead of the |\pgfmathparse| macro you can also use wrapper commands, whose
usage is very similar to their cousins in the \calcname{} package. The only
difference is that the expressions can be any expression that is handled by
|\pgfmathparse|. For all of the following commands, if \meta{expression} starts
with |+|, no parsing is done and a simple assignment or increment is done using
normal \TeX\ assignments or increments. This will be orders of magnitude faster
than calling the parser.

The effect of the following commands is always local to the current \TeX\
scope.

\begin{command}{\pgfmathsetlength\marg{register}\marg{expression}}
    Basically, this command sets the length of the \TeX{} \meta{register} to
    the value specified by \meta{expression}. However, there is some fine
    print:

    First, in case \meta{expression} starts with a |+|, a simple \TeX\
    assignment is done. In particular, \meta{register} can be a glue register
    and \meta{expression} be something like |+1pt plus 1fil| and the
    \meta{register} will be assigned the expected value.

    Second, when the \meta{expression} does not start with |+|, it is first
    parsed using |\pgfmathparse|, resulting in a (dimensionless) value
    |\pgfmathresult|. Now, if the parser encountered the unit |mu| somewhere in
    the expression, it assumes that \meta{register} is a |\muskip| register and
    will try to assign to \meta{register} the value |\pgfmathresult| followed
    by |mu|. Otherwise, in case |mu| was not encountered, it is assumed that
    \meta{register} is a dimension register or a glue register and we assign
    |\pgfmathresult| followed by |pt| to it.

    The net effect of the above is that you can write things like
    %
\begin{codeexample}[]
  \muskipdef\mymuskip=0
  \pgfmathsetlength{\mymuskip}{1mu+3*4mu} \the\mymuskip
\end{codeexample}
    %
\begin{codeexample}[]
  \dimendef\mydimen=0
  \pgfmathsetlength{\mydimen}{1pt+3*4pt}  \the\mydimen
\end{codeexample}
    %
\begin{codeexample}[]
  \skipdef\myskip=0
  \pgfmathsetlength{\myskip}{1pt+3*4pt}  \the\myskip
\end{codeexample}

    One thing that will \emph{not} work is
    |\pgfmathsetlength{\myskip}{1pt plus 1fil}| since the parser does not
    support fill's. You can, however, use the |+| notation in this case:
    %
\begin{codeexample}[]
  \skipdef\myskip=0
  \pgfmathsetlength{\myskip}{+1pt plus 1fil}  \the\myskip
\end{codeexample}
    %
\end{command}

\begin{command}{\pgfmathaddtolength\marg{register}\marg{expression}}
    Adds the value of \meta{expression} to the \TeX{} \meta{register}. All of
    the special consideration mentioned for |\pgfmathsetlength| also apply here
    in the same way.
\end{command}

\begin{command}{\pgfmathsetcount\marg{count register}\marg{expression}}
    Sets the value of the \TeX{} \meta{count register}, to the \emph{truncated}
    value specified by \meta{expression}.
\end{command}

\begin{command}{\pgfmathaddtocount\marg{count register}\marg{expression}}
    Adds the \emph{truncated} value  of \meta{expression} to the \TeX{}
    \meta{count register}.
\end{command}

\begin{command}{\pgfmathsetcounter\marg{counter}\marg{expression}}
    Sets the value of the \LaTeX{} \meta{counter} to the \emph{truncated} value
    specified by \meta{expression}.
\end{command}

\begin{command}{\pgfmathaddtocounter\marg{counter}\marg{expression}}
    Adds the \emph{truncated} value  of \meta{expression} to \meta{counter}.
\end{command}

\begin{command}{\pgfmathsetmacro\marg{macro}\marg{expression}}
    Defines \meta{macro} as the  value of \meta{expression}. The result is a
    decimal without units.
\end{command}

\begin{command}{\pgfmathsetlengthmacro\marg{macro}\marg{expression}}
    Defines \meta{macro} as the value of \meta{expression} \LaTeX{} \emph{in
    points}.
\end{command}

\begin{command}{\pgfmathtruncatemacro\marg{macro}\marg{expression}}
    Defines \meta{macro} as the truncated value of \meta{expression}.
\end{command}


\subsubsection{Considerations Concerning Units}
\label{pgfmath-units}

As was explained earlier, the parser commands like |\pgfmathparse| will always
return a result without units in it and all dimensions that have a unit like
|10pt| or |1in| will first be converted to \TeX\ points (|pt|) and, then, the
unit is dropped.

Sometimes it is useful, nevertheless, to find out whether an expression or not.
For this, you can use the following commands:

{\let\ifpgfmathunitsdeclared\relax
  \begin{command}{\ifpgfmathunitsdeclared}
    After a call  of |\pgfmathparse| this if will be true exactly if
    some unit was encountered in the expression. It is always set
    globally in each call.

    Note that \emph{any} ``mentioning'' of a unit inside an
    expression will set this \TeX-if to true. In particular, even an
    expressionlike |2pt/1pt|, which arguably should be considered
    ``scalar'' or ``unit-free'' will still have this \TeX-if set to
    true. However, see the |scalar| function for a way to change
    this.
  \end{command}
}

\begin{math-function}{scalar(\mvar{value})}
\mathcommand
    This function is the identity function on its input, but it will reset the
    \TeX-if |\ifpgfmathunitsdeclared|. Thus, it can be used to indicate that
    the given \meta{value} should be considered as a ``scalar'' even when it
    contains units; but note that it will work even when the \meta{value} is a
    string or something else. The only effect of this function is to clear the
    unit declaration.
    %
\begin{codeexample}[]
\pgfmathparse{scalar(1pt/2pt)} \pgfmathresult\
\ifpgfmathunitsdeclared with \else without \fi unit
\end{codeexample}

    Note, however, that this command (currently) really just clears the \TeX-if
    as the input is scanned from left-to-right. Thus, even if there is a use of
    a unit before the |scalar| function is used, the \TeX-if will be cleared:
    %
\begin{codeexample}[]
\pgfmathparse{1pt+scalar(1pt)} \pgfmathresult\
\ifpgfmathunitsdeclared with \else without \fi unit
\end{codeexample}

    The other way round, a use of a unit after the |scalar| function will set
    the units once more.
    %
\begin{codeexample}[]
\pgfmathparse{scalar(1pt)+1pt} \pgfmathresult\
\ifpgfmathunitsdeclared with \else without \fi unit
\end{codeexample}

    For these reasons, you should use the function only on the outermost level
    of an expression.

    A typical use of this function is the following:
    %
\begin{codeexample}[preamble={\usetikzlibrary{calc,quotes}}]
\tikz{
  \coordinate["$A$"]       (A) at (2,2);
  \coordinate["$B$" below] (B) at (0,0);
  \coordinate["$C$" below] (C) at (3,0);
  \draw (A) -- (B) -- (C) -- cycle;
  \path
    let \p1 =($(A)-(B)$), \p2 =($(A)-(C)$),
        \n1 = {veclen(\x1,\y1)}, \n2 = {veclen(\x2,\y2)}
    in  coordinate ["$D$" below] (D) at ($ (B)!scalar(\n1/(\n1+\n2))!(C) $);
  \draw (A) -- (D);
}
\end{codeexample}
    %
\end{math-function}

A special kind of units are \TeX's ``math units'' (|mu|). It will be treated as
if |pt| had been used, but you can check whether an expression contained a math
unit using the following:
%
{\let\ifpgfmathmathunitsdeclared\relax
  \begin{command}{\ifpgfmathmathunitsdeclared}
    This \TeX-if is similar to |\ifpgfmathunitsdeclared|, but it
    is only set when the unit |mu| is encountered at least
    once. In this case, |\ifpgfmathunitsdeclared| will \emph{also}
    be set to true. The |scalar| function has no effect on this \TeX-if.
  \end{command}
}


\subsection{Syntax for Mathematical Expressions: Operators}

The syntax for the expressions recognized by |\pgfmathparse| and friends is
rather straightforward. Let us start with the operators.

\label{pgfmath-operators}

The following operators (presented in the context in which they are used) are
recognized:
%
\begin{math-operator}{+}{infix}{add}
    Adds \mvar{x} to \mvar{y}.
\end{math-operator}

\begin{math-operator}{-}{infix}{subtract}
    Subtracts \mvar{y} from \mvar{x}.
\end{math-operator}

\begin{math-operator}{-}{prefix}{neg}
    Reverses the sign of \mvar{x}.
\end{math-operator}

\begin{math-operator}{*}{infix}{multiply}
    Multiplies \mvar{x} by \mvar{y}.
\end{math-operator}

\begin{math-operator}{/}{infix}{divide}
    Divides \mvar{x} by \mvar{y}. An error will result if \mvar{y} is 0, or if
    the result of the division is too big for the mathematical engine. Please
    remember when using this command that accurate (and reasonably quick)
    division of real numbers that are not integers is particularly tricky in
    \TeX.
\end{math-operator}

\begin{math-operator}{\char`\^}{infix}{pow}
    Raises \mvar{x} to the power \mvar{y}.
\end{math-operator}

\begin{math-operator}{\protect\exclamationmarktext}{postfix}{factorial}
    Calculates the factorial of \mvar{x}.
\end{math-operator}

\begin{math-operator}{r}{postfix}{deg}
    Converts \mvar{x} to degrees (\mvar{x} is assumed to be in radians). This
    operator has the same precedence as multiplication.
\end{math-operator}

\begin{math-operators}{?}{:}{conditional}{ifthenelse}
    |?| and |:| are special operators which can be used as a shorthand for |if|
    \mvar{x} |then| \mvar{y} |else| \mvar{z} inside the parser. The expression
    \mvar{x} is taken to be true if it evaluates to any non-zero value.
\end{math-operators}

\begin{math-operator}{==}{infix}{equal}
    Returns |1| if \mvar{x}$=$\mvar{y}, |0| otherwise.
\end{math-operator}

\begin{math-operator}{>}{infix}{greater}
    Returns |1| if \mvar{x}$>$\mvar{y}, |0| otherwise.
\end{math-operator}

\begin{math-operator}{<}{infix}{less}
    Returns |1| if \mvar{x}$<$\mvar{y}, |0| otherwise.
\end{math-operator}

\begin{math-operator}{\protect\exclamationmarktext=}{infix}{notequal}
    Returns |1| if \mvar{x}$\neq$\mvar{y}, |0| otherwise.
\end{math-operator}

\begin{math-operator}{>=}{infix}{notless}
    Returns |1| if \mvar{x}$\geq$\mvar{y}, |0| otherwise.
\end{math-operator}

\begin{math-operator}{<=}{infix}{notgreater}
    Returns |1| if \mvar{x}$\leq$\mvar{y}, |0| otherwise.
\end{math-operator}

\begin{math-operator}{{\char`\&}{\char`\&}}{infix}{and}
    Returns |1| if both \mvar{x} and \mvar{y} evaluate to some non-zero value.
    Both arguments are evaluated.
\end{math-operator}

{
 \catcode`\|=12
\begin{math-operator}[no index]{||}{infix}{or}
        \index{*pgfmanualvbarvbarr@\protect\texttt{\protect\pgfmanualvbarvbar} math operator}%
        \index{Math operators!*pgfmanualvbarvbar@\protect\texttt{\protect\pgfmanualvbarvbar}}%
    Returns {\tt 1} if either \mvar{x} or \mvar{y} evaluate to some non-zero
    value.
\end{math-operator}
}

\begin{math-operator}{\protect\exclamationmarktext}{prefix}{not}
    Returns |1| if \mvar{x} evaluates to zero, |0| otherwise.
\end{math-operator}

\begin{math-operators}{(}{)}{group}{}
    These operators act in the usual way, that is, to control the order in
    which operators are executed, for example, |(1+2)*3|. This includes the
    grouping of arguments for functions, for example, |sin(30*10)| or
    |mod(72,3)| (the comma character is also treated as an operator).

    Parentheses for functions with one argument are not always necessary,
    |sin 30| (note the space) is the same as |sin(30)|. However, functions have
    the highest precedence so, |sin 30*10| is the same as |sin(30)*10|.
\end{math-operators}

\begin{math-operators}{\char`\{}{\char`\}}{array}{}
    These operators are used to process array-like structures (within an
    expression these characters do not act like \TeX{} grouping tokens). The
    \meta{array specification} consists of comma separated elements, for
    example, |{1, 2, 3, 4, 5}|. Each element in the array will be evaluated as
    it is parsed, so expressions can be used. In addition, an element of an
    array can be an array itself, allowing multiple dimension arrays to be
    simulated: |{1, {2,3}, {4,5}, 6}|. When storing an array in a macro, do not
    forget the surrounding braces: |\def\myarray{{1,2,3}}| not
    |\def\myarray{1,2,3}|.
    %
\begin{codeexample}[]
\def\myarray{{1,"two",2+1,"IV","cinq","sechs",sin(\i*5)*14}}
\foreach \i in  {0,...,6}{\pgfmathparse{\myarray[\i]}\pgfmathresult, }
\end{codeexample}
    %
\end{math-operators}

\chardef\lbrack=`\[
\chardef\rbrack=`\]
\begin{math-operators}{\lbrack}{\rbrack}{array access}{array}
    |[| and |]| are two operators used in one particular circumstance: to
    access an array (specified using the |{| and |}| operators) using the index
    \mvar{x}. Indexing starts from zero, so, if the index is greater than, or
    equal to, the number of values in the array, an error will occur, and zero
    will be returned.
    %
\begin{codeexample}[]
\def\myarray{{7,-3,4,-9,11}}
\pgfmathparse{\myarray[3]} \pgfmathresult
\end{codeexample}

    If the array is defined to have multiple dimensions, then the array access
    operators can be immediately repeated.
    %
\begin{codeexample}[]
\def\print#1{\pgfmathparse{#1}\pgfmathresult}
\def\identitymatrix{{{1,0,0},{0,1,0},{0,0,1}}}
\tikz[x=0.5cm,y=0.5cm]\foreach \i in {0,1,2} \foreach \j in {0,1,2}
  \node at (\j,-\i) [anchor=base] {\print{\identitymatrix[\i][\j]}};
\end{codeexample}
    %
\end{math-operators}

\begin{math-operators}{\char`\"}{\char`\"}{group}{}
    These operators are used to quote \mvar{x}. However, as every expression is
    expanded with |\edef| before it is parsed, macros (e.g., font commands like
    |\tt| or |\Huge|) may need to be ``protected'' from this expansion (e.g.,
    |\noexpand\Huge|). Ideally, you should avoid such macros anyway. Obviously,
    these operators should be used with great care as further calculations are
    unlikely to be possible with the result.
    %
\begin{codeexample}[]
\def\x{5}
\foreach \y in {0,10}{
  \pgfmathparse{\x > \y ? "\noexpand\Large Bigger" : "\noexpand\tiny smaller"}
  \x\ is \pgfmathresult\ than \y.
}
\end{codeexample}
    %
\end{math-operators}


\subsection{Syntax for Mathematical Expressions: Functions}
\label{pgfmath-functions}

The following functions are recognized:

\medskip
\def\mathlink#1{\hyperlink{math:#1}{\tt#1}}
\begin{tikzpicture}
\foreach \f [count=\i from 0] in
{abs,acos,add,and,array,asin,atan,atan2,bin,ceil,cos,
 cosec,cosh,cot,deg,depth,div,divide,e,equal,factorial, false,
 floor,frac,gcd,greater,height,hex,Hex,int,ifthenelse,iseven,isodd,isprime,
 less,ln,log10,log2,max,min,mod,Mod,multiply,
 neg,not,notequal,notgreater,notless,
 oct,or,pi,pow,rad,rand,random,real,rnd,round,
 scalar,sec,sign,sin,sinh,sqrt,subtract,tan,tanh,true, veclen,width}
\node [anchor=base west] at ({int(\i/12)*2.5cm},{-mod(\i,12)*1.1*\baselineskip}) {\mathlink{\f}};
\end{tikzpicture}
\bigskip

Each function has a \pgfname{} command associated with it (which is also shown
with the function below). In general, the command is simply the name of the
function prefixed with |\pgfmath|, for example, |\pgfmathadd|, but there are
some notable exceptions.


\subsubsection{Basic arithmetic functions}
\label{pgfmath-functions-basic}

\begin{math-function}{add(\mvar{x},\mvar{y})}
\mathcommand
    Adds $x$ and $y$.
    %
\begin{codeexample}[]
\pgfmathparse{add(75,6)} \pgfmathresult
\end{codeexample}
    %
\end{math-function}

\begin{math-function}{subtract(\mvar{x},\mvar{y})}
\mathcommand
    Subtract $y$ from $x$.
    %
\begin{codeexample}[]
\pgfmathparse{subtract(75,6)} \pgfmathresult
\end{codeexample}
    %
\end{math-function}

\begin{math-function}{neg(\mvar{x})}
\mathcommand
    This returns $-\mvar{x}$.
    %
\begin{codeexample}[]
\pgfmathparse{neg(50)} \pgfmathresult
\end{codeexample}
    %
\end{math-function}

\begin{math-function}{multiply(\mvar{x},\mvar{y})}
\mathcommand
    Multiply $x$ by $y$.
    %
\begin{codeexample}[]
\pgfmathparse{multiply(75,6)} \pgfmathresult
\end{codeexample}
    %
\end{math-function}

\begin{math-function}{divide(\mvar{x},\mvar{y})}
\mathcommand
    Divide $x$ by $y$.
    %
\begin{codeexample}[]
\pgfmathparse{divide(75,6)} \pgfmathresult
\end{codeexample}
    %
\end{math-function}

\begin{math-function}{div(\mvar{x},\mvar{y})}
\mathcommand
    Divide $x$ by $y$ and return the integer part of the result.
    %
\begin{codeexample}[]
\pgfmathparse{div(75,9)} \pgfmathresult
\end{codeexample}
    %
\end{math-function}

\begin{math-function}{factorial(\mvar{x})}
\mathcommand
    Return \mvar{x}!.
    %
\begin{codeexample}[]
\pgfmathparse{factorial(5)} \pgfmathresult
\end{codeexample}
    %
\end{math-function}

\begin{math-function}{sqrt(\mvar{x})}
\mathcommand
    Calculates $\sqrt{\textrm{\mvar{x}}}$.
    %
\begin{codeexample}[]
\pgfmathparse{sqrt(10)} \pgfmathresult
\end{codeexample}

\begin{codeexample}[]
\pgfmathparse{sqrt(8765.432)} \pgfmathresult
\end{codeexample}
    %
\end{math-function}

\begin{math-function}{pow(\mvar{x},\mvar{y})}
\mathcommand
    Raises \mvar{x} to the power \mvar{y}. For greatest accuracy, \mvar{y}
    should be an integer. If \mvar{y} is not an integer, the actual calculation
    will be an approximation of $e^{y \ln(x)}$.
    %
\begin{codeexample}[]
\pgfmathparse{pow(2,7)} \pgfmathresult
\end{codeexample}
    %
\end{math-function}

\begin{math-function}{e}
\mathcommand
    Returns the value 2.718281828.
    %
{
\catcode`\^=7
\begin{codeexample}[]
\pgfmathparse{(e^2-e^-2)/2} \pgfmathresult
\end{codeexample}
}
\end{math-function}

\begin{math-function}{exp(\mvar{x})}
\mathcommand
{ \catcode`\^=7

    Maclaurin series for $e^x$.
}
\begin{codeexample}[]
\pgfmathparse{exp(1)} \pgfmathresult
\end{codeexample}

\begin{codeexample}[]
\pgfmathparse{exp(2.34)} \pgfmathresult
\end{codeexample}
    %
\end{math-function}

\begin{math-function}{ln(\mvar{x})}
\mathcommand
{ \catcode`\^=7
    An approximation for $\ln(\textrm{\mvar{x}})$. This uses an algorithm of
    Rouben Rostamian, and coefficients suggested by Alain Matthes.
}
\begin{codeexample}[]
\pgfmathparse{ln(10)} \pgfmathresult
\end{codeexample}

\begin{codeexample}[]
\pgfmathparse{ln(exp(5))} \pgfmathresult
\end{codeexample}
    %
\end{math-function}

\begin{math-function}{log10(\mvar{x})}
\mathcommand[logten(\mvar{x})]
    An approximation for $\log_{10}(\textrm{\mvar{x}})$.
    %
\begin{codeexample}[]
\pgfmathparse{log10(100)} \pgfmathresult
\end{codeexample}
    %
\end{math-function}

\begin{math-function}{log2(\mvar{x})}
\mathcommand[logtwo(\mvar{x})]
    An approximation for $\log_2(\textrm{\mvar{x}})$.
    %
\begin{codeexample}[]
\pgfmathparse{log2(128)} \pgfmathresult
\end{codeexample}
    %
\end{math-function}

\begin{math-function}{abs(\mvar{x})}
\mathcommand
    Evaluates the absolute value of $x$.
    %
\begin{codeexample}[]
\pgfmathparse{abs(-5)} \pgfmathresult
\end{codeexample}

\begin{codeexample}[]
\pgfmathparse{-abs(4*-3)} \pgfmathresult
\end{codeexample}
    %
\end{math-function}

\begin{math-function}{mod(\mvar{x},\mvar{y})}
\mathcommand
    This evaluates \mvar{x} modulo \mvar{y}, using truncated division. The sign
    of the result is the same as the sign of
    $\frac{\textrm{\mvar{x}}}{\textrm{\mvar{y}}}$.
    %
\begin{codeexample}[]
\pgfmathparse{mod(20,6)} \pgfmathresult
\end{codeexample}

\begin{codeexample}[]
\pgfmathparse{mod(-100,30)} \pgfmathresult
\end{codeexample}
    %
\end{math-function}

\begin{math-function}{Mod(\mvar{x},\mvar{y})}
\mathcommand
    This evaluates \mvar{x} modulo \mvar{y}, using floored division. The sign
    of the result is never negative.
    %
\begin{codeexample}[]
\pgfmathparse{Mod(-100,30)} \pgfmathresult
\end{codeexample}
    %
\end{math-function}

\begin{math-function}{sign(\mvar{x})}
\mathcommand
    Returns the sign of $x$.
    %
\begin{codeexample}[]
\pgfmathparse{sign(-5)} \pgfmathresult
\end{codeexample}

\begin{codeexample}[]
\pgfmathparse{sign(0)} \pgfmathresult
\end{codeexample}

\begin{codeexample}[]
\pgfmathparse{sign(5)} \pgfmathresult
\end{codeexample}
    %
\end{math-function}


\subsubsection{Rounding functions}
\label{pgfmath-functions-rounding}

\begin{math-function}{round(\mvar{x})}
\mathcommand
    Rounds \mvar{x} to the nearest integer. It uses ``asymmetric half-up''
    rounding. So |1.5| is rounded to |2|, but |-1.5| is rounded to |-2|
    (\emph{not} |-1|).
    %
\begin{codeexample}[]
\pgfmathparse{round(32.5/17)} \pgfmathresult
\end{codeexample}

\begin{codeexample}[]
\pgfmathparse{round(398/12)} \pgfmathresult
\end{codeexample}
    %
\end{math-function}

\begin{math-function}{floor(\mvar{x})}
\mathcommand
    Rounds \mvar{x} down to the nearest integer.
    %
\begin{codeexample}[]
\pgfmathparse{floor(32.5/17)} \pgfmathresult
\end{codeexample}

\begin{codeexample}[]
\pgfmathparse{floor(398/12)} \pgfmathresult
\end{codeexample}

\begin{codeexample}[]
\pgfmathparse{floor(-398/12)} \pgfmathresult
\end{codeexample}
    %
\end{math-function}

\begin{math-function}{ceil(\mvar{x})}
\mathcommand
    Rounds \mvar{x} up to the nearest integer.
    %
\begin{codeexample}[]
\pgfmathparse{ceil(32.5/17)} \pgfmathresult
\end{codeexample}

\begin{codeexample}[]
\pgfmathparse{ceil(398/12)} \pgfmathresult
\end{codeexample}

\begin{codeexample}[]
\pgfmathparse{ceil(-398/12)} \pgfmathresult
\end{codeexample}
    %
\end{math-function}

\begin{math-function}{int(\mvar{x})}
\mathcommand
    Returns the integer part of \mvar{x}.
    %
\begin{codeexample}[]
\pgfmathparse{int(32.5/17)} \pgfmathresult
\end{codeexample}
    %
\end{math-function}

\begin{math-function}{frac(\mvar{x})}
\mathcommand
    Returns the fractional part of \mvar{x}.
    %
\begin{codeexample}[]
\pgfmathparse{frac(32.5/17)} \pgfmathresult
\end{codeexample}
    %
\end{math-function}

\begin{math-function}{real(\mvar{x})}
\mathcommand
    Ensures \mvar{x} contains a decimal point.
    %
\begin{codeexample}[]
\pgfmathparse{real(4)} \pgfmathresult
\end{codeexample}
    %
\end{math-function}


\subsubsection{Integer arithmetics functions}
\label{pgfmath-functions-integerarithmetics}

\begin{math-function}{gcd(\mvar{x},\mvar{y})}
\mathcommand
    Computes the greatest common divider of \mvar{x} and \mvar{y}.
    %
\begin{codeexample}[]
\pgfmathparse{gcd(42,56)} \pgfmathresult
\end{codeexample}
    %
\end{math-function}

\begin{math-function}{isodd(\mvar{x})}
\mathcommand
    Returns |1| if the integer part of \mvar{x} is odd. Otherwise, returns |0|.
    %
\begin{codeexample}[]
\pgfmathparse{isodd(2)} \pgfmathresult,
\pgfmathparse{isodd(3)} \pgfmathresult
\end{codeexample}
    %
\end{math-function}

\begin{math-function}{iseven(\mvar{x})}
\mathcommand
    Returns |1| if the integer part of \mvar{x} is even. Otherwise, returns |0|.
    %
\begin{codeexample}[]
\pgfmathparse{iseven(2)} \pgfmathresult,
\pgfmathparse{iseven(3)} \pgfmathresult
\end{codeexample}
    %
\end{math-function}

\begin{math-function}{isprime(\mvar{x})}
\mathcommand
    Returns |1| if the integer part of \mvar{x} is prime. Otherwise, returns |0|.
    %
\begin{codeexample}[]
\pgfmathparse{isprime(1)} \pgfmathresult,
\pgfmathparse{isprime(2)} \pgfmathresult,
\pgfmathparse{isprime(31)} \pgfmathresult,
\pgfmathparse{isprime(64)} \pgfmathresult
\end{codeexample}
    %
\end{math-function}


\subsubsection{Trigonometric functions}
\label{pgfmath-functions-trigonometric}

\begin{math-function}{pi}
\mathcommand
    Returns the value $\pi=3.141592654$.
    %
\begin{codeexample}[]
\pgfmathparse{pi} \pgfmathresult
\end{codeexample}

\begin{codeexample}[]
\pgfmathparse{pi r} \pgfmathresult
\end{codeexample}
    %
\end{math-function}

\begin{math-function}{rad(\mvar{x})}
\mathcommand
    Convert \mvar{x} to radians. \mvar{x} is assumed to be in degrees.
    %
\begin{codeexample}[]
\pgfmathparse{rad(90)} \pgfmathresult
\end{codeexample}
    %
\end{math-function}

\begin{math-function}{deg(\mvar{x})}
\mathcommand
    Convert \mvar{x} to degrees. \mvar{x} is assumed to be in radians.
    %
\begin{codeexample}[]
\pgfmathparse{deg(3*pi/2)} \pgfmathresult
\end{codeexample}
    %
\end{math-function}

\begin{math-function}{sin(\mvar{x})}
\mathcommand
    %
    Sine of \mvar{x}. By employing the |r| operator, \mvar{x} can be in
    radians.
    %
\begin{codeexample}[]
\pgfmathparse{sin(60)} \pgfmathresult
\end{codeexample}

\begin{codeexample}[]
\pgfmathparse{sin(pi/3 r)} \pgfmathresult
\end{codeexample}
    %
\end{math-function}

\begin{math-function}{cos(\mvar{x})}
\mathcommand
    Cosine of \mvar{x}. By employing the |r| operator, \mvar{x} can be in
    radians.
    %
\begin{codeexample}[]
\pgfmathparse{cos(60)} \pgfmathresult
\end{codeexample}

\begin{codeexample}[]
\pgfmathparse{cos(pi/3 r)} \pgfmathresult
\end{codeexample}
    %
\end{math-function}

\begin{math-function}{tan(\mvar{x})}
\mathcommand
    Tangent of \mvar{x}. By employing the |r| operator, \mvar{x} can be in
    radians.
    %
\begin{codeexample}[]
\pgfmathparse{tan(45)} \pgfmathresult
\end{codeexample}

\begin{codeexample}[]
\pgfmathparse{tan(2*pi/8 r)} \pgfmathresult
\end{codeexample}
    %
\end{math-function}

\begin{math-function}{sec(\mvar{x})}
\mathcommand
    Secant of \mvar{x}. By employing the |r| operator, \mvar{x} can be in
    radians.
    %
\begin{codeexample}[]
\pgfmathparse{sec(45)} \pgfmathresult
\end{codeexample}
    %
\end{math-function}

\begin{math-function}{cosec(\mvar{x})}
\mathcommand
    Cosecant of \mvar{x}. By employing the |r| operator, \mvar{x} can be in
    radians.
    %
\begin{codeexample}[]
\pgfmathparse{cosec(30)} \pgfmathresult
\end{codeexample}
    %
\end{math-function}

\begin{math-function}{cot(\mvar{x})}
\mathcommand
    Cotangent of \mvar{x}. By employing the |r| operator, \mvar{x} can be in
    radians.
    %
\begin{codeexample}[]
\pgfmathparse{cot(15)} \pgfmathresult
\end{codeexample}
    %
\end{math-function}

\begin{math-function}{asin(\mvar{x})}
\mathcommand
    Arcsine of \mvar{x}. The result is in degrees and in the range $\pm90^\circ$.
    %
\begin{codeexample}[]
\pgfmathparse{asin(0.7071)} \pgfmathresult
\end{codeexample}
    %
\end{math-function}

\begin{math-function}{acos(\mvar{x})}
\mathcommand
    Arccosine of \mvar{x} in degrees. The result is in the range $[0^\circ,180^\circ]$.
    %
\begin{codeexample}[]
\pgfmathparse{acos(0.5)} \pgfmathresult
\end{codeexample}
    %
\end{math-function}

\begin{math-function}{atan(\mvar{x})}
\mathcommand
    Arctangent of $x$ in degrees.
    %
\begin{codeexample}[]
\pgfmathparse{atan(1)} \pgfmathresult
\end{codeexample}
    %
\end{math-function}

\begin{math-function}{atan2(\mvar{y},\mvar{x})}
\mathcommand[atantwo(\mvar{y},\mvar{x})]
    Arctangent of $y\div x$ in degrees. This also takes into account the
    quadrants.
    %
\begin{codeexample}[]
\pgfmathparse{atan2(-4,3)} \pgfmathresult
\end{codeexample}
    %
\end{math-function}

\begin{key}{/pgf/trig format=\mchoice{deg,rad} (initially deg)}
    Allows to define whether trigonometric math functions (i.e.\ all in this
    subsection) operate with degrees or with radians.
    %
\begin{codeexample}[]
\pgfmathparse{cos(45)} \pgfmathresult
\end{codeexample}
\begin{codeexample}[]
\pgfkeys{/pgf/trig format=rad}
\pgfmathparse{cos(pi/2)} \pgfmathresult
\end{codeexample}

    The initial configuration |trig format=deg| is the base of \pgfname: almost
    all of it is based on degrees.

    Specifying |trig format=rad| is most useful for data visualization where
    the angles are typically given in radians. However, it is applied to all
    trigonometric functions for which the option applies, including any drawing
    instructions which operate on angles.
    %
\begin{codeexample}[]
\begin{tikzpicture}
    \draw[-stealth]
        (0:1) -- (45:1) -- (90:1) -- (135:1) -- (180:1);

    \draw[-stealth,trig format=rad,red]
        (pi:1) -- (5/4*pi:1) -- (6/4*pi:1) -- (7/4*pi:1) -- (2*pi:1);
\end{tikzpicture}
\end{codeexample}

    \paragraph{Warning:}
    At the time of this writing, this feature is ``experimental''. Please
    handle it with care: there may be path instructions or libraries in
    \pgfname\ which rely on |trig format=deg|. The intended usage of
    |trig format=rad| is for local scopes -- and as option for data
    visualization.
\end{key}


\subsubsection{Comparison and logical functions}
\label{pgfmath-functions-comparison}

\begin{math-function}{equal(\mvar{x},\mvar{y})}
\mathcommand
    This returns |1| if $\mvar{x}=\mvar{y}$ and |0| otherwise.
    %
\begin{codeexample}[]
\pgfmathparse{equal(20,20)} \pgfmathresult
\end{codeexample}
    %
\end{math-function}

\begin{math-function}{greater(\mvar{x},\mvar{y})}
\mathcommand
    This returns |1| if $\mvar{x}>\mvar{y}$ and |0| otherwise.
    %
\begin{codeexample}[]
\pgfmathparse{greater(20,25)} \pgfmathresult
\end{codeexample}
    %
\end{math-function}

\begin{math-function}{less(\mvar{x},\mvar{y})}
\mathcommand
    This returns |1| if $\mvar{x}<\mvar{y}$ and |0| otherwise.
    %
\begin{codeexample}[]
\pgfmathparse{greater(20,25)} \pgfmathresult
\end{codeexample}
    %
\end{math-function}

\begin{math-function}{notequal(\mvar{x},\mvar{y})}
\mathcommand
    This returns |0| if $\mvar{x}=\mvar{y}$ and |1| otherwise.
    %
\begin{codeexample}[]
\pgfmathparse{notequal(20,25)} \pgfmathresult
\end{codeexample}
    %
\end{math-function}

\begin{math-function}{notgreater(\mvar{x},\mvar{y})}
\mathcommand
    This returns |1| if $\mvar{x}\leq\mvar{y}$ and |0| otherwise.
    %
\begin{codeexample}[]
\pgfmathparse{notgreater(20,25)} \pgfmathresult
\end{codeexample}
    %
\end{math-function}

\begin{math-function}{notless(\mvar{x},\mvar{y})}
\mathcommand
    This returns |1| if $\mvar{x}\geq\mvar{y}$ and |0| otherwise.
    %
\begin{codeexample}[]
\pgfmathparse{notless(20,25)} \pgfmathresult
\end{codeexample}
    %
\end{math-function}

\begin{math-function}{and(\mvar{x},\mvar{y})}
\mathcommand
    This returns |1| if \mvar{x} and \mvar{y} both evaluate to non-zero values.
    Otherwise |0| is returned.
    %
\begin{codeexample}[]
\pgfmathparse{and(5>4,6>7)} \pgfmathresult
\end{codeexample}
    %
\end{math-function}

\begin{math-function}{or(\mvar{x},\mvar{y})}
\mathcommand
    This returns |1| if either \mvar{x} or \mvar{y} evaluate to non-zero
    values. Otherwise |0| is returned.
    %
\begin{codeexample}[]
\pgfmathparse{or(5>4,6>7)} \pgfmathresult
\end{codeexample}
    %
\end{math-function}

\begin{math-function}{not(\mvar{x})}
\mathcommand
    This returns |1| if $\mvar{x}=0$, otherwise |0|.
    %
\begin{codeexample}[]
\pgfmathparse{not(true)} \pgfmathresult
\end{codeexample}
    %
\end{math-function}

\begin{math-function}{ifthenelse(\mvar{x},\mvar{y},\mvar{z})}
\mathcommand
    This returns \mvar{y} if \mvar{x} evaluates to some non-zero value,
    otherwise \mvar{z} is returned.
    %
\begin{codeexample}[]
\pgfmathparse{ifthenelse(5==4,"yes","no")} \pgfmathresult
\end{codeexample}
    %
\end{math-function}

\begin{math-function}{true}
\mathcommand
    This evaluates to |1|.
    %
\begin{codeexample}[]
\pgfmathparse{true ? "yes" : "no"} \pgfmathresult
\end{codeexample}
    %
\end{math-function}

\begin{math-function}{false}
\mathcommand
    This evaluates to |0|.
    %
\begin{codeexample}[]
\pgfmathparse{false ? "yes" : "no"} \pgfmathresult
\end{codeexample}
    %
\end{math-function}


\subsubsection{Pseudo-random functions}
\label{pgfmath-functions-random}

\begin{math-function}{rnd}
\mathcommand
    Generates a pseudo-random number between $0$ and $1$ with a uniform
    distribution.
    %
\begin{codeexample}[pre={\pgfmathsetseed{1}}]
\foreach \x in {1,...,10}{\pgfmathparse{rnd}\pgfmathresult, }
\end{codeexample}
    %
\end{math-function}

\begin{math-function}{rand}
\mathcommand
    Generates a pseudo-random number between $-1$ and $1$ with a uniform
    distribution.
    %
\begin{codeexample}[pre={\pgfmathsetseed{1}}]
\foreach \x in {1,...,10}{\pgfmathparse{rand}\pgfmathresult, }
\end{codeexample}
    %
\end{math-function}

\begin{math-function}{random(\opt{\mvar{x},\mvar{y}})}
\mathcommand
    This function takes zero, one or two arguments. If there are zero
    arguments, a uniform random number between $0$ and $1$ is generated. If
    there is one argument \mvar{x}, a random integer between $1$ and \mvar{x}
    is generated. Finally, if there are two arguments, a random integer between
    \mvar{x} and \mvar{y} is generated. If there are no arguments, the
    \pgfname{} command should be called as follows: |\pgfmathrandom{}|.
    %
\begin{codeexample}[pre={\pgfmathsetseed{1}}]
\foreach \x in {1,...,10}{\pgfmathparse{random()}\pgfmathresult, }
\end{codeexample}

\begin{codeexample}[pre={\pgfmathsetseed{1}}]
\foreach \x in {1,...,10}{\pgfmathparse{random(100)}\pgfmathresult, }
\end{codeexample}

\begin{codeexample}[pre={\pgfmathsetseed{1}}]
\foreach \x in {1,...,10}{\pgfmathparse{random(232,762)}\pgfmathresult, }
\end{codeexample}
    %
\end{math-function}


\subsubsection{Base conversion functions}
\label{pgfmath-functions-base}

\begin{math-function}{hex(\mvar{x})}
\mathcommand
    Convert \mvar{x}{} (assumed to be an integer in base 10) to a hexadecimal
    representation, using lower case alphabetic digits. No further calculation
    will be possible with the result.
    %
\begin{codeexample}[]
\pgfmathparse{hex(65535)} \pgfmathresult
\end{codeexample}
    %
\end{math-function}

\begin{math-function}{Hex(\mvar{x})}
\mathcommand
    Convert \mvar{x}{} (assumed to be an integer in base 10) to a hexadecimal
    representation, using upper case alphabetic digits. No further calculation
    will be possible with the result.
    %
\begin{codeexample}[]
\pgfmathparse{Hex(65535)} \pgfmathresult
\end{codeexample}
    %
\end{math-function}

\begin{math-function}{oct(\mvar{x})}
\mathcommand
    Convert \mvar{x}{} (assumed to be an integer in base 10) to an octal
    representation. No further calculation should be attempted with the result,
    as the parser can only process numbers converted to base 10.
    %
\begin{codeexample}[]
\pgfmathparse{oct(63)} \pgfmathresult
\end{codeexample}
    %
\end{math-function}

\begin{math-function}{bin(\mvar{x})}
\mathcommand
    Convert \mvar{x}{} (assumed to be an integer in base 10) to a binary
    representation. No further calculation should be attempted with the result,
    as the parser can only process numbers converted to base 10.
    %
\begin{codeexample}[]
\pgfmathparse{bin(185)} \pgfmathresult
\end{codeexample}
    %
\end{math-function}


\subsubsection{Miscellaneous functions}
\label{pgfmath-functions-misc}

\begin{math-function}{min(\mvar{x$_1$},\mvar{x$_2$},\ldots,\mvar{x$_n$})}
\mathcommand[min({\mvar{x$_1$},\mvar{x$_2$},\ldots},{\ldots,\mvar{x$_{n-1}$},\mvar{x$_n$}})]
    Return the minimum value from \mvar{x$_1$}\ldots\mvar{x$_n$}. For
    historical reasons, the command |\pgfmathmin| takes two arguments, but each
    of these can contain an arbitrary number of comma separated values.
    %
\begin{codeexample}[]
\pgfmathparse{min(3,4,-2,250,-8,100)} \pgfmathresult
\end{codeexample}
    %
\end{math-function}

\begin{math-function}{max(\mvar{x$_1$},\mvar{x$_2$},\ldots,\mvar{x$_n$})}
\mathcommand[max({\mvar{x$_1$},\mvar{x$_2$},\ldots},{\ldots,\mvar{x$_{n-1}$},\mvar{x$_n$}})]
    Return the maximum value from \mvar{x$_1$}\ldots\mvar{x$_n$}. Again, for
    historical reasons, the command |\pgfmathmax| takes two arguments, but each
    of these can contain an arbitrary number of comma separated values.
    %
\begin{codeexample}[]
\pgfmathparse{max(3,4,-2,250,-8,100)} \pgfmathresult
\end{codeexample}
    %
\end{math-function}

\begin{math-function}{veclen(\mvar{x},\mvar{y})}
\mathcommand
    Calculates $\sqrt{\left(\textrm{\mvar{x}}^2+\textrm{\mvar{y}}^2\right)}$.
    This uses a polynomial approximation, based on ideas of Rouben Rostamian
    %
\begin{codeexample}[]
\pgfmathparse{veclen(12,5)} \pgfmathresult
\end{codeexample}
    %
\end{math-function}

\begin{math-function}{array(\mvar{x},\mvar{y})}
\mathcommand
    This accesses the array \mvar{x} at the index \mvar{y}. The array must
    begin and end with braces (e.g., |{1,2,3,4}|) and array indexing starts at
    |0|.
    %
\begin{codeexample}[]
\pgfmathparse{array({9,13,17,21},2)} \pgfmathresult
\end{codeexample}
    %
\end{math-function}

The following hyperbolic functions were adapted from code suggested by Martin
Heller:

\begin{math-function}{sinh(\mvar{x})}
\mathcommand
    The hyperbolic sine of \mvar{x}
    %
\begin{codeexample}[]
\pgfmathparse{sinh(0.5)} \pgfmathresult
\end{codeexample}
    %
\end{math-function}

\begin{math-function}{cosh(\mvar{x})}
\mathcommand
    The hyperbolic cosine of \mvar{x}
    %
\begin{codeexample}[]
\pgfmathparse{cosh(0.5)} \pgfmathresult
\end{codeexample}
    %
\end{math-function}

\begin{math-function}{tanh(\mvar{x})}
\mathcommand
    The hyperbolic tangent of \mvar{x}
    %

\begin{codeexample}[]
\pgfmathparse{tanh(0.5)} \pgfmathresult
\end{codeexample}
    %
\end{math-function}

\begin{math-function}{width("\mvar{x}")}
\mathcommand
    Return the width of a \TeX{} (horizontal) box containing \mvar{x}. The
    quote characters are necessary to prevent \mvar{x}{} from being parsed. It
    is important to remember that any expression is expanded with |\edef|
    before being parsed, so any macros (e.g., font commands like |\tt| or
    |\Huge|) will need to be ``protected'' (e.g., |\noexpand\Huge| is usually
    sufficient).
    %
\begin{codeexample}[]
\pgfmathparse{width("Some Lovely Text")} \pgfmathresult
\end{codeexample}

    Note that results of this method are provided in points.
\end{math-function}

\begin{math-function}{height("\mvar{x}")}
\mathcommand
    Return the height of a box containing \mvar{x}.
    %
\begin{codeexample}[]
\pgfmathparse{height("Some Lovely Text")} \pgfmathresult
\end{codeexample}
    %
\end{math-function}

\begin{math-function}{depth("\mvar{x}")}
\mathcommand
    Returns the depth of a box containing \mvar{x}.
    %
\begin{codeexample}[]
\pgfmathparse{depth("Some Lovely Text")} \pgfmathresult
\end{codeexample}
    %
\end{math-function}

% Copyright 2018 by Mark Wibrow
%
% This file may be distributed and/or modified
%
% 1. under the LaTeX Project Public License and/or
% 2. under the GNU Free Documentation License.
%
% See the file doc/generic/pgf/licenses/LICENSE for more details.


\section{Additional Mathematical Commands}
\label{pgfmath-commands}

Instead of parsing and evaluating complex expressions, you can also use the
mathematical engine to evaluate a single mathematical operation. The macros
used for many of these computations are listed above in
Section~\ref{pgfmath-functions}. \pgfname{} also provides some additional
commands which are shown below:


\subsection{Basic arithmetic functions}
\label{pgfmath-commands-basic}

In addition to the commands described in Section~\ref{pgfmath-functions-basic},
the following command is provided:

\begin{command}{\pgfmathreciprocal\marg{x}}
    Defines |\pgfmathresult| as $1\div\meta{x}$. This provides greatest
    accuracy when \mvar{x} is small.
\end{command}


\subsection{Comparison and logical functions}

In addition to the commands described in
Section~\ref{pgfmath-functions-comparison}, the following command was provided
by Christian Feuers\"anger:

\begin{command}{\pgfmathapproxequalto\marg{x}\marg{y}}
    Defines |\pgfmathresult| 1.0 if $ \rvert \meta{x} - \meta{y} \lvert <
    0.0001$, but 0.0 otherwise. As a side-effect, the global boolean
    |\ifpgfmathcomparison| will be set accordingly.
\end{command}


\subsection{Pseudo-Random Numbers}
\label{pgfmath-random}

In addition to the commands described in
Section~\ref{pgfmath-functions-random}, the following commands are provided:

\begin{command}{\pgfmathgeneratepseudorandomnumber}
    Defines |\pgfmathresult| as a pseudo-random integer between 1 and
    $2^{31}-1$. This uses a linear congruency generator, based on ideas of
    Erich Janka.
\end{command}

\begin{command}{\pgfmathrandominteger\marg{macro}\marg{minimum}\marg{maximum}}
    This defines \meta{macro} as a pseudo-randomly generated integer from the
    range \meta{minimum} to \meta{maximum} (inclusive).
    %
\begin{codeexample}[]
\begin{pgfpicture}
   \foreach \x in {1,...,50}{
      \pgfmathrandominteger{\a}{1}{50}
      \pgfmathrandominteger{\b}{1}{50}
      \pgfpathcircle{\pgfpoint{+\a pt}{+\b pt}}{+2pt}
      \color{blue!40!white}
      \pgfsetstrokecolor{blue!80!black}
      \pgfusepath{stroke, fill}
   }
\end{pgfpicture}
\end{codeexample}
    %
\end{command}

\begin{command}{\pgfmathdeclarerandomlist\marg{list name}\{\marg{item-1}\marg{item 2}...\}}
    This creates a list of items with the name \meta{list name}.
\end{command}

\begin{command}{\pgfmathrandomitem\marg{macro}\marg{list name}}
    Select an item from a random list \meta{list name}. The
    selected item is placed in \meta{macro}.
\end{command}

\begin{codeexample}[]
\begin{pgfpicture}
   \pgfmathdeclarerandomlist{color}{{red}{blue}{green}{yellow}{white}}
   \foreach \a in {1,...,50}{
      \pgfmathrandominteger{\x}{1}{85}
      \pgfmathrandominteger{\y}{1}{85}
      \pgfmathrandominteger{\r}{5}{10}
      \pgfmathrandomitem{\c}{color}
      \pgfpathcircle{\pgfpoint{+\x pt}{+\y pt}}{+\r pt}
      \color{\c!40!white}
      \pgfsetstrokecolor{\c!80!black}
      \pgfusepath{stroke, fill}
   }
\end{pgfpicture}
\end{codeexample}

\begin{command}{\pgfmathsetseed\marg{integer}}
    Explicitly sets the seed for the pseudo-random number generator. By default
    it is set to the value of |\time|$\times$|\year|.
\end{command}


\subsection{Base Conversion}
\label{pgfmath-bases}

\pgfname{} provides limited support for conversion between
\emph{representations} of numbers. Currently the numbers must be positive
integers in the range $0$ to $2^{31}-1$, and the bases in the range $2$ to
$36$. All digits representing numbers greater than 9 (in base ten), are
alphabetic, but may be upper or lower case.

In addition to the commands described in Section~\ref{pgfmath-functions-base},
the following commands are provided:

\begin{command}{\pgfmathbasetodec\marg{macro}\marg{number}\marg{base}}
    Defines \meta{macro} as the result of converting \meta{number} from base
    \meta{base} to base 10. Alphabetic digits can be upper or lower case.

\medskip{\def\medskip{}

\begin{codeexample}[]
\pgfmathbasetodec\mynumber{107f}{16} \mynumber
\end{codeexample}

    \noindent Note that, as usual in \TeX, the braces around an argument can be
    omitted if the argument is just a single token (a macro name is a single
    token).
    %
\begin{codeexample}[]
\pgfmathbasetodec\mynumber{33FC}{20} \mynumber
\end{codeexample}

}\medskip
    %
\end{command}

\begin{command}{\pgfmathdectobase\marg{macro}\marg{number}\marg{base}}
    Defines \meta{macro} as the result of converting \meta{number} from base 10
    to base \meta{base}. Any resulting alphabetic digits are in \emph{lower
    case}.
    %
\begin{codeexample}[]
\pgfmathdectobase\mynumber{65535}{16} \mynumber
\end{codeexample}
    %
\end{command}

\begin{command}{\pgfmathdectoBase\marg{macro}\marg{number}\marg{base}}
    Defines \meta{macro} as the result of converting \meta{number} from base 10
    to base \meta{base}. Any resulting alphabetic digits are in \emph{upper
    case}.
    %
\begin{codeexample}[]
\pgfmathdectoBase\mynumber{65535}{16} \mynumber
\end{codeexample}
    %
\end{command}

\begin{command}{\pgfmathbasetobase\marg{macro}\marg{number}\marg{base-1}\marg{base-2}}
    Defines \meta{macro} as the result of converting \meta{number} from base
    \meta{base-1} to base \meta{base-2}. Alphabetic digits in \meta{number} can
    be upper or lower case, but any resulting alphabetic digits are in
    \emph{lower case}.
    %
\begin{codeexample}[]
\pgfmathbasetobase\mynumber{11011011}{2}{16} \mynumber
\end{codeexample}
    %
\end{command}

\begin{command}{\pgfmathbasetoBase\marg{macro}\marg{number}\marg{base-1}\marg{base-2}}
    Defines \meta{macro} as the result of converting \meta{number} from base
    \meta{base-1} to base \meta{base-2}. Alphabetic digits in \meta{number} can
    be upper or lower case, but any resulting alphabetic digits are in
    \emph{upper case}.
    %
\begin{codeexample}[]
\pgfmathbasetoBase\mynumber{121212}{3}{12} \mynumber
\end{codeexample}
    %
\end{command}

\begin{command}{\pgfmathsetbasenumberlength\marg{integer}}
    Sets the number of digits in the result of a base conversion to
    \meta{integer}. If the result of a conversion has less digits than this
    number, it is prefixed with zeros.
    %
\begin{codeexample}[]
\pgfmathsetbasenumberlength{8}
\pgfmathdectobase\mynumber{15}{2} \mynumber
\end{codeexample}
    %
\end{command}

\begin{command}{\pgfmathtodigitlist\marg{macro}\marg{number}}
    This command converts \meta{number} into a comma-separated list of digits
    and stores the result in \meta{macro}. The \marg{number} is \emph{not}
    parsed before processing.
    %
\begin{codeexample}[]
\pgfmathsetbasenumberlength{8}
\begin{tikzpicture}[x=0.25cm, y=0.25cm]
  \foreach \n [count=\y] in {0, 60, 102, 102, 126, 102, 102, 102, 0}{
    \pgfmathdectobase{\binary}{\n}{2}
    \pgfmathtodigitlist{\digitlist}{\binary}
    \foreach \digit [count=\x, evaluate={\c=\digit*50+15;}] in \digitlist
      \fill [fill=black!\c] (\x, -\y) rectangle ++(1,1);
  }
\end{tikzpicture}
\end{codeexample}
    %
\end{command}


\subsection{Angle Computations}

Unlike the rest of the math engine, which is a ``standalone'' package, the
following commands only work in conjunction with the core of \pgfname.

\begin{command}{\pgfmathanglebetweenpoints\marg{p}\marg{q}}
    Returns the angle of a line from \meta{p} to \meta{q} relative to a line
    going straight right from \meta{p}.
    %
\begin{codeexample}[]
\pgfmathanglebetweenpoints{\pgfpoint{1cm}{3cm}}{\pgfpoint{2cm}{4cm}}
\pgfmathresult
\end{codeexample}
    %
\end{command}

\begin{command}{\pgfmathanglebetweenlines\marg{$p_1$}\marg{$q_1$}\marg{$p_2$}\marg{$q_2$}}
    Returns the clockwise angle between a line going through $p_1$ and $q_1$
    and a line going through $p_2$ and $q_2$.
    %
\begin{codeexample}[]
\pgfmathanglebetweenlines{\pgfpoint{1cm}{3cm}}{\pgfpoint{2cm}{4cm}}
                         {\pgfpoint{0cm}{1cm}}{\pgfpoint{1cm}{0cm}}
\pgfmathresult
\end{codeexample}
    %
\end{command}

% Copyright 2019 by Mark Wibrow
%
% This file may be distributed and/or modified
%
% 1. under the LaTeX Project Public License and/or
% 2. under the GNU Free Documentation License.
%
% See the file doc/generic/pgf/licenses/LICENSE for more details.


\section{Customizing the Mathematical Engine}
\label{pgfmath-reimplement}

Perhaps you have a desire for some function that \pgfname\ does not provide.
Perhaps you are not happy with the accuracy or efficiency of some of the
algorithms that are implemented in \pgfname. In these cases you will want to
add a function to the parser or replace the current implementations of the
algorithms with your own code.

The mathematical engine was designed with such customization in mind. It is
possible to add new functions, or modify the code for existing functions. Note,
however, that whilst adding new operators is possible, it can be a rather
tricky business and is only recommended for adventurous users.

To add a new function to the math engine the following command can be used:

\begin{command}{\pgfmathdeclarefunction\opt{|*|}\marg{function name}\marg{number of arguments}\marg{code}}
    This will set up the parser to recognize a function called \meta{name}. The
    name of the function can consist of, uppercase or lowercase letters,
    numbers or the underscore |_|. In line with many programming languages, a
    function name cannot begin with a number or contain any spaces. The
    function may not have been declared earlier, unless the optional star (|*|)
    is provided, which forces an ``overwriting'' of the function by the new
    function. Note that you \emph{should never change the arity of standard
    functions} and you should normally use |\pgfmathredeclarefunction|,
    which does not allow you to do anything wrong here.

    The \meta{number of arguments} can be any positive integer, zero, or the
    value |...|, which indicates a variable number of arguments. \pgfname{}
    treats constants, such as |pi| and |e|, as functions with zero arguments.
    Functions with more than nine arguments or with a variable number of
    arguments are a ``bit special'' and are discussed below.

    The effect of \meta{code} should be to set the macro |\pgfmathresult| to
    the correct value (namely to the result of the computation without units).
    Furthermore, the function should have no other side effects, that is, it
    should not change any global values. As an example, consider the creation
    of a new function |double|, which takes one argument, and returns the value
    of that argument times two.
    %
\begin{codeexample}[]
\makeatletter
\pgfmathdeclarefunction{double}{1}{
  \begingroup
    \pgf@x=#1pt\relax
    \multiply\pgf@x by2\relax
    \pgfmathreturn\pgf@x
  \endgroup
}
\makeatother
\pgfmathparse{double(44.3)}\pgfmathresult
\end{codeexample}

    The macro |\pgfmathreturn|\meta{tokens} must be
    directly followed by an |\endgroup| and will save the result of the
    computation, by defining |\pgfmathresult| as the expansion of
    \meta{tokens} (without units) outside the group, so \meta{tokens}
    must be something that can be assigned to a dimension register.

    Alternatively, the |\pgfmathsmuggle|\meta{macro} can be used. This must
    also be directly followed by an |\endgroup| and will simply ``smuggle'' the
    definition of \meta{macro} outside the \TeX-group.

    By performing computations within a \TeX-group, \pgfname{} registers such
    as |\pgf@x|, |\pgf@y| and |\c@pgf@counta|, |\c@pgfcountb|, and so forth,
    can be used at will.

    Beyond setting up the parser, this command also defines two macros which
    provide access to the function independently of the parser:
    %
    \begin{itemize}
        \item |\pgfmath|\meta{function name}

            This macro will provide a ``public'' interface for the function
            \meta{function name} allowing the function to be called
            independently of the parser. All arguments passed to this macro are
            evaluated using |\pgfmathparse| and then passed on to the following
            macro:
        \item |\pgfmath|\meta{function name}|@|

            This macro is the ``private'' implementation of the function's
            algorithm (but note that, for speed, the parser calls this macro
            rather than the ``public'' one). Arguments passed to this macro are
            expected to be numbers without units. It is defined using
            \meta{code}, but need not be self-contained.
    \end{itemize}

    For functions that are declared with less than ten arguments, the public
    macro is defined in the same way as normal \TeX{} macros using, for
    example, |\def\pgfmathNoArgs{|\meta{code}|}| for a function with no
    arguments, or |\def\pgfmathThreeArgs#1#2#3{|\meta{code}|}| for a function
    with three arguments. The private macro is defined in the same way, and
    each argument can therefore be accessed in \meta{code} using |#1|, |#2| and
    so on.

    For functions with more than nine arguments, or functions with a variable
    number of arguments, these macros are only defined as taking \emph{one}
    argument. The public macro expects its arguments to be comma separated, for
    example, |\pgfmathVariableArgs{1.1,3.5,-1.5,2.6}|. Each argument is parsed
    and passed on to the private macro as follows:
    |\pgfmathVariableArgs@{{1.1}{3.5}{-1.5}{2.6}}|. This means that some
    ``extra work'' will be required to access each argument (although it is a
    fairly simple task).

    Note that there are two exceptions to this arrangement: the public versions
    of the |min| and |max| functions still take two arguments for compatibility
    with older versions, but each of these arguments can take several comma
    separated values.
\end{command}

To redefine a function use the following command:

\begin{command}{\pgfmathredeclarefunction\marg{function name}\marg{code}}
    This command redefines the |\pgfmath|\meta{function name}|@| macro with the
    new \meta{code}. See the description of the
    |\pgfmathdeclarefunction| for details. You cannot change the number of
    arguments for an existing function.
    %
\begin{codeexample}[]
\makeatletter
\pgfmathdeclarefunction{foo}{1}{
  \begingroup
    \pgf@x=#1pt\relax
    \multiply\pgf@x by2\relax
    \pgfmathreturn\pgf@x
  \endgroup
}
\pgfmathparse{foo(42)}\pgfmathresult
\pgfmathredeclarefunction{foo}{
  \begingroup
    \pgf@x=#1pt\relax
    \multiply\pgf@x by3\relax
    \pgfmathreturn\pgf@x
  \endgroup
}
\pgfmathparse{foo(42)}\pgfmathresult
\makeatother
\end{codeexample}
    %
\end{command}

    \pgfname{} uses the last known definition of a function within the
    prevailing scope, so it is possible for a function to be redefined locally.
    You should also remember that any |.sty| or |.tex| file containing any
    re-implementations should be loaded after |pgfmath|.

    In addition to the above commands, the following key is provided to quickly
    create simple ad hoc functions which can greatly improve the readability of
    code, and is particularly useful in \tikzname{}:

\begin{key}{/pgf/declare function=\meta{function definitions}}
    This key allows simple functions to be created locally. Its use is perhaps
    best illustrated by an example:
    %
\begin{codeexample}[]
\begin{tikzpicture}
  \draw [help lines] (0,0) grid (3,2);
  \draw [blue, thick, x=0.0085cm, y=1cm,
    declare function={
      sines(\t,\a,\b)=1 + 0.5*(sin(\t)+sin(\t*\a)+sin(\t*\b));
    }]
    plot [domain=0:360, samples=144, smooth] (\x,{sines(\x,3,5)});
\end{tikzpicture}
\end{codeexample}

    Each definition in \meta{function definitions} takes the form
    \meta{name}|(|\meta{arguments}|)=|\meta{definition}|;| (note the semicolon
    at the end, this is very important). If multiple functions are being
    defined, the semicolon is used to separate them (\emph{not} a comma). The
    function \meta{name} can be any name that is not already a function name in
    the current scope. The list of \meta{arguments} are commands such as |\x|,
    or |\y| (it is not possible to declare functions that take variable numbers
    of arguments using this key). If the function takes no arguments, then the
    parentheses need not be used. The \meta{definition} should be an expression
    that can be parsed by the mathematical engine and should use the commands
    specified in \meta{arguments}.

    When specifying multiple functions, functions that appear later on in
    \meta{function definitions} can refer to earlier functions:
    %
\begin{codeexample}[pre={\pgfmathsetseed{1}}]
\begin{tikzpicture}[
  declare function={
    excitation(\t,\w) = sin(\t*\w);
    noise             = rnd - 0.5;
    source(\t)        = excitation(\t,20) + noise;
    filter(\t)        = 1 - abs(sin(mod(\t, 90)));
    speech(\t)        = 1 + source(\t)*filter(\t);
  }
]
  \draw [help lines] (0,0) grid (3,2);
  \draw [blue, thick, x=0.0085cm, y=1cm] (0,1) --
    plot [domain=0:360, samples=144, smooth] (\x,{speech(\x)});
\end{tikzpicture}
\end{codeexample}
    %
\end{key}

\begin{key}{/pgf/declare function/execute at begin function=\meta{tokens}}
    These \meta{tokens} are inserted just before |\pgfmathdeclarefunction|
    scans the body of the function definition.  This is a rather low-level
    option, so you should read the implementation to figure out where the
    \meta{tokens} are inserted.
\end{key}

\begin{key}{/pgf/declare function/execute at end function=\meta{tokens}}
    These \meta{tokens} are inserted just after |\pgfmathdeclarefunction| has
    finished scanning the body of the function definition.  This is a rather
    low-level option, so you should read the implementation to figure out where
    the \meta{tokens} are inserted.
\end{key}

\begin{key}{/pgf/declare function/ignore spaces=\meta{boolean}}
    Uses the two previously described keys |/pgf/declare function/execute at begin function| and
    |/pgf/declare function/execute at end function| to install catcodes such
    that spaces inside the body of the function definition of
    |\pgfmathdeclarefunction| are ignored.  The usual \TeX\ tokenization rules
    apply, so if the body of the function had already been tokenized by other
    means this will become ineffective.  If you want to use a space you can use
    |~| in the function body which has its catcode set to 10 (space).
\end{key}


\section{Number Printing}
\label{pgfmath-numberprinting}

{\emph{An extension by Christian Feuersänger}}

\medskip
\noindent
\pgfname\ supports number printing in different styles and rounds to arbitrary
precision.

\begin{command}{\pgfmathprintnumber\marg{x}}
    Generates pretty-printed output for the (real) number \meta{x}. The input
    number \meta{x} is parsed using |\pgfmathfloatparsenumber| which allows
    arbitrary precision.

    Numbers are typeset in math mode using the current set of number printing
    options, see below. Optional arguments can also be provided using
    |\pgfmathprintnumber[|\meta{options}|]|\meta{x}.
\end{command}

\begin{command}{\pgfmathprintnumberto\marg{x}\marg{macro}}
    Returns the resulting number into \meta{macro} instead of typesetting it
    directly.
\end{command}

\begin{key}{/pgf/number format/fixed}
    Configures |\pgfmathprintnumber| to round the number to a fixed number of
    digits after the period, discarding any trailing zeros.
    %
\begin{codeexample}[]
\pgfkeys{/pgf/number format/.cd,fixed,precision=2}
\pgfmathprintnumber{4.568}\hspace{1em}
\pgfmathprintnumber{5e-04}\hspace{1em}
\pgfmathprintnumber{0.1}\hspace{1em}
\pgfmathprintnumber{24415.98123}\hspace{1em}
\pgfmathprintnumber{123456.12345}
\end{codeexample}

    See section~\ref{sec:number:styles} for how to change the appearance.
\end{key}

\begin{key}{/pgf/number format/fixed zerofill=\marg{boolean}  (default true)}
    Enables or disables zero filling for any number drawn in fixed point
    format.
    %
\begin{codeexample}[]
\pgfkeys{/pgf/number format/.cd,fixed,fixed zerofill,precision=2}
\pgfmathprintnumber{4.568}\hspace{1em}
\pgfmathprintnumber{5e-04}\hspace{1em}
\pgfmathprintnumber{0.1}\hspace{1em}
\pgfmathprintnumber{24415.98123}\hspace{1em}
\pgfmathprintnumber{123456.12345}
\end{codeexample}
    %
    This key affects numbers drawn with |fixed| or |std| styles (the latter
    only if no scientific format is chosen).
    %
\begin{codeexample}[]
\pgfkeys{/pgf/number format/.cd,std,fixed zerofill,precision=2}
\pgfmathprintnumber{4.568}\hspace{1em}
\pgfmathprintnumber{5e-05}\hspace{1em}
\pgfmathprintnumber{1}\hspace{1em}
\pgfmathprintnumber{123456.12345}
\end{codeexample}

    See section~\ref{sec:number:styles} for how to change the appearance.
\end{key}

\begin{key}{/pgf/number format/sci}
    Configures |\pgfmathprintnumber| to display numbers in scientific format,
    that means sign, mantissa and exponent (basis~$10$). The mantissa is
    rounded to the desired |precision| (or |sci precision|, see below).
    %
\begin{codeexample}[]
\pgfkeys{/pgf/number format/.cd,sci,precision=2}
\pgfmathprintnumber{4.568}\hspace{1em}
\pgfmathprintnumber{5e-04}\hspace{1em}
\pgfmathprintnumber{0.1}\hspace{1em}
\pgfmathprintnumber{24415.98123}\hspace{1em}
\pgfmathprintnumber{123456.12345}
\end{codeexample}

    See section~\ref{sec:number:styles} for how to change the exponential
    display style.
\end{key}

\begin{key}{/pgf/number format/sci zerofill=\marg{boolean}  (default true)}
    Enables or disables zero filling for any number drawn in scientific format.
    %
\begin{codeexample}[]
\pgfkeys{/pgf/number format/.cd,sci,sci zerofill,precision=2}
\pgfmathprintnumber{4.568}\hspace{1em}
\pgfmathprintnumber{5e-04}\hspace{1em}
\pgfmathprintnumber{0.1}\hspace{1em}
\pgfmathprintnumber{24415.98123}\hspace{1em}
\pgfmathprintnumber{123456.12345}
\end{codeexample}
    %
    As with |fixed zerofill|, this option does only affect numbers drawn in
    |sci| format (or |std| if the scientific format is chosen).

    See section~\ref{sec:number:styles} for how to change the exponential
    display style.
\end{key}

\begin{stylekey}{/pgf/number format/zerofill=\marg{boolean} (default true)}
    Sets both |fixed zerofill| and |sci zerofill| at once.
\end{stylekey}

\begin{keylist}{/pgf/number format/std,%
    /pgf/number format/std=\meta{lower e},
    /pgf/number format/std=\meta{lower e}:\meta{upper e}%
}
    Configures |\pgfmathprintnumber| to a standard algorithm. It chooses either
    |fixed| or |sci|, depending on the order of magnitude. Let $n=s \cdot m
    \cdot 10^e$ be the input number and $p$ the current precision. If $-p/2 \le
    e \le 4$, the number is displayed using |fixed| format. Otherwise, it is
    displayed using |sci| format.
    %
\begin{codeexample}[]
\pgfkeys{/pgf/number format/.cd,std,precision=2}
\pgfmathprintnumber{4.568}\hspace{1em}
\pgfmathprintnumber{5e-04}\hspace{1em}
\pgfmathprintnumber{0.1}\hspace{1em}
\pgfmathprintnumber{24415.98123}\hspace{1em}
\pgfmathprintnumber{123456.12345}
\end{codeexample}
    %
    The parameters can be customized using the optional integer argument(s): if
    $\text{\meta{lower e}} \le e \le \text{\meta{upper e}}$, the number is
    displayed in |fixed| format, otherwise in |sci| format. Note that
    \meta{lower e} should be negative for useful results. The precision used
    for the scientific format can be adjusted with |sci precision| if
    necessary.
\end{keylist}

\begin{keylist}{/pgf/number format/relative*=\meta{exponent base 10}}
    Configures |\pgfmathprintnumber| to format numbers relative to an order of
    magnitude, $10^r$, where $r$ is an integer number.

    This key addresses different use-cases.

    \paragraph{First use-case:}

    provide a unified format for a \emph{sequence} of numbers. Consider the
    following test:
    %
\begin{codeexample}[]
\pgfkeys{/pgf/number format/relative*={1}}
\pgfmathprintnumber{6.42e-16}\hspace{1em}
\pgfmathprintnumber{1.2}\hspace{1em}
\pgfmathprintnumber{6}\hspace{1em}
\pgfmathprintnumber{20.6}\hspace{1em}
\pgfmathprintnumber{87}
\end{codeexample}
    %
    \noindent With any other style, the |6.42e-16| would have been formatted as
    an isolated number. Here, it is rounded to |0| because when viewed relative
    to $10^1$ (the exponent $1$ is the argument for |relative|), it has no
    significant digits.
    %
\begin{codeexample}[]
\pgfkeys{/pgf/number format/relative*={2}}
\pgfmathprintnumber{123.345}\hspace{1em}
\pgfmathprintnumber{0.0012}\hspace{1em}
\pgfmathprintnumber{0.0014}\hspace{1em}
\end{codeexample}
    %
    \noindent The example above applies the initial |precision=2| to |123.345|
    -- relative to $100$. Two significant digits of |123.345| relative to $100$
    are |123|. Note that the ``$2$ significant digits of |123.345|'' translates
    to ``round |1.2345| to $2$ digits'', which would yield |1.2300|. Similarly,
    the other two numbers are |0| compared to $100$ using the given
    |precision|.
    %
\begin{codeexample}[]
\pgfkeys{/pgf/number format/relative*={-3}}
\pgfmathprintnumber{123.345}\hspace{1em}
\pgfmathprintnumber{0.0012}\hspace{1em}
\pgfmathprintnumber{0.0014}\hspace{1em}
\end{codeexample}

    \paragraph{Second use-case:}

    improve rounding in the presence of \emph{inaccurate} numbers. Let us
    suppose that some limited-precision arithmetics resulted in the result
    |123456999| (like the |fpu| of \pgfname). You know that its precision is
    about five or six significant digits. And you want to provide a fixed point
    output. In this case, the trailing digits |....999| are a numerical
    artifact due to the limited precision. Use |relative*=3,precision=0| to
    eliminate the artifacts:
    %
\begin{codeexample}[]
\pgfkeys{/pgf/number format/.cd,relative*={3},precision=0}
\pgfmathprintnumber{123456999}\hspace{1em}
\pgfmathprintnumber{123456999.12}
\end{codeexample}
    %
    \noindent Here, |precision=0| means that we inspect |123456.999| and round
    that number to $0$ digits. Finally, we move the period back to its initial
    position. Adding |relative style=fixed| results in fixed point output
    format:
    %
\begin{codeexample}[]
\pgfkeys{/pgf/number format/.cd,relative*={3},precision=0,relative style=fixed}
\pgfmathprintnumber{123456999}\hspace{1em}
\pgfmathprintnumber{123456999.12}
\end{codeexample}
    %
    \noindent Note that there is another alternative for this use-case which is
    discussed later: the |fixed relative| style.
    %
\begin{codeexample}[]
\pgfkeys{/pgf/number format/.cd,fixed relative,precision=6}
\pgfmathprintnumber{123456999}\hspace{1em}
\pgfmathprintnumber{123456999.12}
\end{codeexample}

    You might wonder why there is an asterisk in the key's name. The short
    answer is: there is also a \declareandlabel{/pgf/number format/relative}
    number printer which does unexpected things. The key |relative*| repairs
    this. Existing code will still use the old behavior.

    Technically, the key works as follows: as already explained above,
    |relative*=3| key applied to |123456999.12| moves the period by three
    positions and analyzes |123456.99912|. Mathematically speaking, we are
    given a number $x = \pm m \cdot 10^e$ and we attempt to apply
    |relative*=|$r$. The method then rounds $x / 10^r$ to |precision| digits.
    Afterwards, it multiplies the result by $10^r$ and typesets it.
\end{keylist}

\begin{stylekey}{/pgf/number format/every relative}
    A style which configures how the |relative| method finally displays its
    results.

    The initial configuration is
    %
\begin{codeexample}[code only]
\pgfkeys{/pgf/number format/every relative/.style=std}
\end{codeexample}

    Note that rounding is turned off when the resulting style is being
    evaluated (since |relative| already rounded the number).

    Although supported, I discourage from using |fixed zerofill| or
    |sci zerofill| in this context -- it may lead to a suggestion of higher
    precision than is actually used (because |fixed zerofill| might simply add
    |.00| although there was a different information before |relative| rounded
    the result).
\end{stylekey}

\begin{key}{/pgf/number format/relative style=\marg{options}}
    The same as |every relative/.append style=|\marg{options}.
\end{key}

\begin{keylist}{/pgf/number format/fixed relative}
    Configures |\pgfmathprintnumber| to format numbers in a similar way to the
    |fixed| style, but the |precision| is interpreted relatively to the
    number's exponent.

    The motivation is to get the same rounding effect as for |sci|, but to
    display the number in the |fixed| style:
    %
\begin{codeexample}[]
\pgfkeys{/pgf/number format/.cd,fixed relative,precision=3}
\pgfmathprintnumber{1000.0123}\hspace{1em}
\pgfmathprintnumber{100.0567}\hspace{1em}
\pgfmathprintnumber{0.000010003452}\hspace{1em}
\pgfmathprintnumber{0.010073452}\hspace{1em}
\pgfmathprintnumber{1.23567}\hspace{1em}
\pgfmathprintnumber{1003.75}\hspace{1em}
\pgfmathprintnumber{1006.75}\hspace{1em}
\end{codeexample}

    The effect of |fixed relative| is that the number is rounded to
    \emph{exactly} the first \meta{precision} non-zero digits, no matter how
    many leading zeros the number might have.

    Use |fixed relative| if you want |fixed| and if you know that only the
    first $n$ digits are correct. Use |sci| if you need a scientific display
    style and only the first $n$ digits are correct.

    Note that |fixed relative| ignores the |fixed zerofill| flag.

    See also the |relative*| key. Note that the |relative=|\marg{exponent} key
    explicitly moves the period to some designated position before it attempts
    to round the number. Afterwards, it ``rounds from the right'', i.e.\ it
    rounds to that explicitly chosen digit position. In contrast to that,
    |fixed relative| ``rounds from the left'': it takes the \emph{first}
    non-zero digit, temporarily places the period after this digit, and rounds
    that number. The rounding style |fixed| leaves the period where it is, and
    rounds everything behind that digit. The |sci| style is similar to 
    |fixed relative|.
\end{keylist}

\begin{key}{/pgf/number format/int detect}
    Configures |\pgfmathprintnumber| to detect integers automatically. If the
    input number is an integer, no period is displayed at all. If not, the
    scientific format is chosen.
    %
\begin{codeexample}[]
\pgfkeys{/pgf/number format/.cd,int detect,precision=2}
\pgfmathprintnumber{15}\hspace{1em}
\pgfmathprintnumber{20}\hspace{1em}
\pgfmathprintnumber{20.4}\hspace{1em}
\pgfmathprintnumber{0.01}\hspace{1em}
\pgfmathprintnumber{0}
\end{codeexample}
    %
\end{key}

\begin{command}{\pgfmathifisint\marg{number constant}\marg{true code}\marg{false code}}
    A command which does the same check as |int detect|, but it invokes
    \meta{true code} if the \meta{number constant} actually is an integer and
    the \meta{false code} if not.

    As a side-effect, |\pgfretval| will contain the parsed number, either in
    integer format or as parsed floating point number.

    The argument \meta{number constant} will be parsed with
    |\pgfmathfloatparsenumber|.
    %
\begin{codeexample}[]
15 \pgfmathifisint{15}{is an int: \pgfretval.}{is no int}\hspace{1em}
15.5 \pgfmathifisint{15.5}{is an int: \pgfretval.}{is no int}
\end{codeexample}
    %
\end{command}

\begin{key}{/pgf/number format/int trunc}
    Truncates every number to integers (discards any digit after the period).
\begin{codeexample}[]
\pgfkeys{/pgf/number format/.cd,int trunc}
\pgfmathprintnumber{4.568}\hspace{1em}
\pgfmathprintnumber{5e-04}\hspace{1em}
\pgfmathprintnumber{0.1}\hspace{1em}
\pgfmathprintnumber{24415.98123}\hspace{1em}
\pgfmathprintnumber{123456.12345}
\end{codeexample}
    %
\end{key}

\begin{key}{/pgf/number format/frac}
    Displays numbers as fractionals.
    %
\begin{codeexample}[width=3cm,preamble={\usetikzlibrary{fpu}}]
\pgfkeys{/pgf/number format/frac}
\pgfmathprintnumber{0.333333333333333}\hspace{1em}
\pgfmathprintnumber{0.5}\hspace{1em}
\pgfmathprintnumber{2.133333333333325e-01}\hspace{1em}
\pgfmathprintnumber{0.12}\hspace{1em}
\pgfmathprintnumber{2.666666666666646e-02}\hspace{1em}
\pgfmathprintnumber{-1.333333333333334e-02}\hspace{1em}
\pgfmathprintnumber{7.200000000000000e-01}\hspace{1em}
\pgfmathprintnumber{6.666666666666667e-02}\hspace{1em}
\pgfmathprintnumber{1.333333333333333e-01}\hspace{1em}
\pgfmathprintnumber{-1.333333333333333e-02}\hspace{1em}
\pgfmathprintnumber{3.3333333}\hspace{1em}
\pgfmathprintnumber{1.2345}\hspace{1em}
\pgfmathprintnumber{1}\hspace{1em}
\pgfmathprintnumber{-6}
\end{codeexample}

    \begin{key}{/pgf/number format/frac TeX=\marg{\textbackslash macro} (initially \texttt{\textbackslash frac})}
        Allows to use a different implementation for |\frac| inside of the
        |frac| display type.
    \end{key}

    \begin{key}{/pgf/number format/frac denom=\meta{int} (initially empty)}
        Allows to provide a custom denominator for |frac|.
        %
\begin{codeexample}[width=3cm,preamble={\usetikzlibrary{fpu}}]
\pgfkeys{/pgf/number format/.cd,frac, frac denom=10}
\pgfmathprintnumber{0.1}\hspace{1em}
\pgfmathprintnumber{0.5}\hspace{1em}
\pgfmathprintnumber{1.2}\hspace{1em}
\pgfmathprintnumber{-0.6}\hspace{1em}
\pgfmathprintnumber{-1.4}\hspace{1em}
\end{codeexample}
    \end{key}
    %
    \begin{key}{/pgf/number format/frac whole=\mchoice{true,false} (initially true)}
        Configures whether complete integer parts shall be placed in front of
        the fractional part. In this case, the fractional part will be less
        then $1$. Use |frac whole=false| to avoid whole number parts.
        %
\begin{codeexample}[width=3cm,preamble={\usetikzlibrary{fpu}}]
\pgfkeys{/pgf/number format/.cd,frac, frac whole=false}
\pgfmathprintnumber{20.1}\hspace{1em}
\pgfmathprintnumber{5.5}\hspace{1em}
\pgfmathprintnumber{1.2}\hspace{1em}
\pgfmathprintnumber{-5.6}\hspace{1em}
\pgfmathprintnumber{-1.4}\hspace{1em}
\end{codeexample}
    \end{key}
    %
    \begin{key}{/pgf/number format/frac shift=\marg{integer} (initially 4)}
        In case you experience problems because of stability problems, try
        experimenting with a different |frac shift|. Higher shift values $k$
        yield higher sensitivity to inaccurate data or inaccurate arithmetics.

        Technically, the following happens. If $r < 1$ is the fractional part
        of the mantissa, then a scale $i = 1/r \cdot 10^k$ is computed where
        $k$ is the shift; fractional parts of $i$ are neglected. The value
        $1/r$ is computed internally, its error is amplified.

        If you still experience stability problems, use |\usepackage{fp}| in
        your preamble. The |frac| style will then automatically employ the
        higher absolute precision of |fp| for the computation of $1/r$.
    \end{key}
\end{key}

\begin{key}{/pgf/number format/precision=\marg{number}}
    Sets the desired rounding precision for any display operation. For
    scientific format, this affects the mantissa.
\end{key}

\begin{key}{/pgf/number format/sci precision=\meta{number or empty} (initially empty)}
    Sets the desired rounding precision only for |sci| styles.

    Use |sci precision={}| to restore the initial configuration (which uses the
    argument provided to |precision| for all number styles).
\end{key}

\begin{key}{/pgf/number format/read comma as period=\mchoice{true,false} (initially false)}
    This is one of the few keys which allows to customize the number parser. If
    this switch is turned on, a comma is read just as a period.
    %
\begin{codeexample}[]
\pgfkeys{/pgf/number format/read comma as period}
\pgfmathprintnumber{1234,56}
\end{codeexample}
    %
    This is typically undesired as it can cause side-effects with math parsing
    instructions. However, it is supported to format input numbers or input
    tables. Consider |use comma| to typeset the result with a comma as well.
    %
\begin{codeexample}[]
\pgfkeys{/pgf/number format/.cd,
    read comma as period,
    use comma}
\pgfmathprintnumber{1234,56}
\end{codeexample}
    %
\end{key}


\subsection{Changing display styles}%
\label{sec:number:styles}

You can change the way how numbers are displayed. For example, if you use the
`\texttt{fixed}' style, the input number is rounded to the desired precision
and the current fixed point display style is used to typeset the number. The
same is applied to any other format: first, rounding routines are used to get
the correct digits, afterwards a display style generates proper \TeX-code.

\begin{key}{/pgf/number format/set decimal separator=\marg{text}}
    Assigns \marg{text} as decimal separator for any fixed point numbers
    (including the mantissa in sci format).

    Use |\pgfkeysgetvalue{/pgf/number format/set decimal separator}\value| to
    get the current separator into |\value|.
\end{key}

\begin{stylekey}{/pgf/number format/dec sep=\marg{text}}
    Just another name for |set decimal separator|.
\end{stylekey}

\begin{key}{/pgf/number format/set thousands separator=\marg{text}}
    Assigns \marg{text} as thousands separator for any fixed point numbers
    (including the mantissa in sci format).
    %
\begin{codeexample}[]
\pgfkeys{/pgf/number format/.cd,
    fixed,
    fixed zerofill,
    precision=2,
    set thousands separator={}}
\pgfmathprintnumber{1234.56}
\end{codeexample}
    %
\begin{codeexample}[]
\pgfkeys{/pgf/number format/.cd,
    fixed,
    fixed zerofill,
    precision=2,
    set thousands separator={}}
\pgfmathprintnumber{1234567890}
\end{codeexample}

\begin{codeexample}[]
\pgfkeys{/pgf/number format/.cd,
    fixed,
    fixed zerofill,
    precision=2,
    set thousands separator={.}}
\pgfmathprintnumber{1234567890}
\end{codeexample}
    %
\begin{codeexample}[]
\pgfkeys{/pgf/number format/.cd,
    fixed,
    fixed zerofill,
    precision=2,
    set thousands separator={,}}
\pgfmathprintnumber{1234567890}
\end{codeexample}
    %
\begin{codeexample}[]
\pgfkeys{/pgf/number format/.cd,
    fixed,
    fixed zerofill,
    precision=2,
    set thousands separator={{{,}}}}
\pgfmathprintnumber{1234567890}
\end{codeexample}
    %
    The last example employs commas and disables the default comma-spacing.

    Use |\pgfkeysgetvalue{/pgf/number format/set thousands separator}\value| to
    get the current separator into |\value|.
\end{key}

\begin{stylekey}{/pgf/number format/1000 sep=\marg{text}}
    Just another name for |set thousands separator|.
\end{stylekey}

\begin{key}{/pgf/number format/1000 sep in fractionals=\marg{boolean} (initially false)}
    Configures whether the fractional part should also be grouped into groups
    of three digits.

    The value |true| will active the |1000 sep| for both, integer and
    fractional parts. The value |false| will active |1000 sep| only for the
    integer part.
    %
\begin{codeexample}[]
\pgfkeys{/pgf/number format/.cd,
    fixed,
    precision=999,
    set thousands separator={\,},
    1000 sep in fractionals,
    }
\pgfmathprintnumber{1234.1234567}
\end{codeexample}
    %
\begin{codeexample}[]
\pgfkeys{/pgf/number format/.cd,
    fixed,fixed zerofill,
    precision=9,
    set thousands separator={\,},
    1000 sep in fractionals,
    }
\pgfmathprintnumber{1234.1234567}
\end{codeexample}
    %
\end{key}

\begin{key}{/pgf/number format/min exponent for 1000 sep=\marg{number} (initially 0)}
    Defines the smallest exponent in scientific notation which is required to
    draw thousand separators. The exponent is the number of digits minus one,
    so $\meta{number}=4$ will use thousand separators starting with $1e4 =
    10000$.
    %
\begin{codeexample}[]
\pgfkeys{/pgf/number format/.cd,
    int detect,
    1000 sep={\,},
    min exponent for 1000 sep=0}
\pgfmathprintnumber{5000}; \pgfmathprintnumber{1000000}
\end{codeexample}

\begin{codeexample}[]
\pgfkeys{/pgf/number format/.cd,
    int detect,
    1000 sep={\,},
    min exponent for 1000 sep=4}
\pgfmathprintnumber{1000}; \pgfmathprintnumber{5000}
\end{codeexample}
    %
\begin{codeexample}[]
\pgfkeys{/pgf/number format/.cd,
    int detect,
    1000 sep={\,},
    min exponent for 1000 sep=4}
\pgfmathprintnumber{10000}; \pgfmathprintnumber{1000000}
\end{codeexample}
    %
    \noindent A value of |0| disables this feature (negative values are
    ignored).
\end{key}

\begin{key}{/pgf/number format/use period}
    A predefined style which installs periods ``\texttt{.}'' as decimal
    separators and commas ``\texttt{,}'' as thousands separators. This style is
    the default.
    %
\begin{codeexample}[]
\pgfkeys{/pgf/number format/.cd,fixed,precision=2,use period}
\pgfmathprintnumber{12.3456}
\end{codeexample}
    %
\begin{codeexample}[]
\pgfkeys{/pgf/number format/.cd,fixed,precision=2,use period}
\pgfmathprintnumber{1234.56}
\end{codeexample}
    %
\end{key}

\begin{key}{/pgf/number format/use comma}
    A predefined style which installs commas ``\texttt{,}'' as decimal
    separators and periods ``\texttt{.}'' as thousands separators.
    %
\begin{codeexample}[]
\pgfkeys{/pgf/number format/.cd,fixed,precision=2,use comma}
\pgfmathprintnumber{12.3456}
\end{codeexample}
    %
\begin{codeexample}[]
\pgfkeys{/pgf/number format/.cd,fixed,precision=2,use comma}
\pgfmathprintnumber{1234.56}
\end{codeexample}
    %
\end{key}

\begin{key}{/pgf/number format/skip 0.=\marg{boolean} (initially false)}
    Configures whether numbers like $0.1$ shall be typeset as $.1$ or not.
    %
\begin{codeexample}[]
\pgfkeys{/pgf/number format/.cd,
    fixed,
    fixed zerofill,precision=2,
    skip 0.}
\pgfmathprintnumber{0.56}
\end{codeexample}
    %
\begin{codeexample}[]
\pgfkeys{/pgf/number format/.cd,
    fixed,
    fixed zerofill,precision=2,
    skip 0.=false}
\pgfmathprintnumber{0.56}
\end{codeexample}
    %
\end{key}

\begin{key}{/pgf/number format/showpos=\marg{boolean} (initially false)}
    Enables or disables the display of plus signs for non-negative numbers.
    %
\begin{codeexample}[]
\pgfkeys{/pgf/number format/showpos}
\pgfmathprintnumber{12.345}
\end{codeexample}

\begin{codeexample}[]
\pgfkeys{/pgf/number format/showpos=false}
\pgfmathprintnumber{12.345}
\end{codeexample}

\begin{codeexample}[]
\pgfkeys{/pgf/number format/.cd,showpos,sci}
\pgfmathprintnumber{12.345}
\end{codeexample}
    %
\end{key}

\begin{stylekey}{/pgf/number format/print sign=\marg{boolean}}
    A style which is simply an alias for |showpos=|\marg{boolean}.
\end{stylekey}

\begin{key}{/pgf/number format/sci 10e}
    Uses $m \cdot 10^e$ for any number displayed in scientific format.
    %
\begin{codeexample}[]
\pgfkeys{/pgf/number format/.cd,sci,sci 10e}
\pgfmathprintnumber{12.345}
\end{codeexample}
    %
\end{key}

\begin{key}{/pgf/number format/sci 10\textasciicircum e}
    The same as `|sci 10e|'.
\end{key}

\begin{key}{/pgf/number format/sci e}
    Uses the `$1e{+}0$' format which is generated by common scientific tools
    for any number displayed in scientific format.
    %
\begin{codeexample}[]
\pgfkeys{/pgf/number format/.cd,sci,sci e}
\pgfmathprintnumber{12.345}
\end{codeexample}
    %
\end{key}

\begin{key}{/pgf/number format/sci E}
    The same with an uppercase `\texttt{E}'.
    %
\begin{codeexample}[]
\pgfkeys{/pgf/number format/.cd,sci,sci E}
\pgfmathprintnumber{12.345}
\end{codeexample}
    %
\end{key}

\begin{key}{/pgf/number format/sci subscript}
    Typesets the exponent as subscript for any number displayed in scientific
    format. This style requires very little space.
    %
\begin{codeexample}[]
\pgfkeys{/pgf/number format/.cd,sci,sci subscript}
\pgfmathprintnumber{12.345}
\end{codeexample}
    %
\end{key}

\begin{key}{/pgf/number format/sci superscript}
    Typesets the exponent as superscript for any number displayed in scientific
    format. This style requires very little space.
    %
\begin{codeexample}[]
\pgfkeys{/pgf/number format/.cd,sci,sci superscript}
\pgfmathprintnumber{12.345}
\end{codeexample}
    %
\end{key}

\begin{key}{/pgf/number format/sci generic=\marg{keys}}
    Allows to define an own number style for the scientific format. Here,
    \meta{keys} can be one of the following choices (omit the long key prefix):

    \begin{key}{/pgf/number format/sci generic/mantissa sep=\marg{text} (initially empty)}
        Provides the separator between a mantissa and the exponent. It might be
        |\cdot|, for example,
    \end{key}
    %
    \begin{key}{/pgf/number format/sci generic/exponent=\marg{text} (initially empty)}
        Provides text to format the exponent. The actual exponent is available
        as argument |#1| (see below).
    \end{key}
    %
\begin{codeexample}[]
\pgfkeys{
    /pgf/number format/.cd,
    sci,
    sci generic={mantissa sep=\times,exponent={10^{#1}}}}
\pgfmathprintnumber{12.345};
\pgfmathprintnumber{0.00012345}
\end{codeexample}
    %
    The \meta{keys} can depend on three parameters, namely on |#1| which is the
    exponent, |#2| containing the flags entity of the floating point number and
    |#3| is the (unprocessed and unformatted) mantissa.

    Note that |sci generic| is \emph{not} suitable to modify the appearance of
    fixed point numbers, nor can it be used to format the mantissa (which is
    typeset like fixed point numbers). Use |dec sep|, |1000 sep| and
    |print sign| to customize the mantissa.
\end{key}

\begin{key}{/pgf/number format/retain unit mantissa=\mchoice{true,false} (initially true)}
    Allows to omit a unit mantissa.
    %
\begin{codeexample}[]
\pgfkeys{
    /pgf/number format/.cd,
    sci, retain unit mantissa=false}
\pgfmathprintnumber{10.5};
\pgfmathprintnumber{10};
\pgfmathprintnumber{1010};
\pgfmathprintnumber[precision=1]{-1010};
\end{codeexample}
    %
    The feature is applied after rounding to the desired precision: if the
    remaining mantissa is equal to~$1$, it will be omitted. It applies to all
    styles involving the scientific format (including |std|).
\end{key}

\begin{key}{/pgf/number format/\protect\atmarktext dec sep mark=\marg{text}}
    Will be placed right before the place where a decimal separator belongs to.
    However, \marg{text} will be inserted even if there is no decimal
    separator. It is intended as place-holder for auxiliary routines to find
    alignment positions.

    This key should never be used to change the decimal separator!
    Use |dec sep| instead.
\end{key}

\begin{key}{/pgf/number format/\protect\atmarktext sci exponent mark=\marg{text}}
    Will be placed right before exponents in scientific notation. It is
    intended as place-holder for auxiliary routines to find alignment
    positions.

    This key should never be used to change the exponent!
\end{key}

\begin{key}{/pgf/number format/assume math mode=\marg{boolean} (default true)}
    Set this to |true| if you don't want any checks for math mode. The initial
    setting checks whether math mode is active using |\pgfutilensuremath| for
    each final number.

    Use |assume math mode=true| if you know that math mode is active. In that
    case, the final number is typeset as-is, no further checking is performed.
\end{key}

\begin{stylekey}{/pgf/number format/verbatim}
    A style which configures the number printer to produce verbatim text
    output, i.e., it doesn't contain \TeX\ macros.
    %
\begin{codeexample}[preamble={\usetikzlibrary{fpu}}]
\pgfkeys{
    /pgf/fpu,
    /pgf/number format/.cd,
    sci,
    verbatim}
\pgfmathprintnumber{12.345};
\pgfmathprintnumber{0.00012345};
\pgfmathparse{exp(15)}
\pgfmathprintnumber{\pgfmathresult}
\end{codeexample}
    %
    The style resets |1000 sep|, |dec sep|, |print sign|, |skip 0.| and sets
    |assume math mode|. Furthermore, it installs a |sci generic| format for
    verbatim output of scientific numbers.

    However, it will still respect |precision|, |fixed zerofill|,
    |sci zerofill| and the overall styles |fixed|, |sci|, |int detect| (and
    their variants). It might be useful if you intend to write output files.
\end{stylekey}


%--------------------------------------------------
% \subsubsection{Defining own display styles}
% You can define own display styles, although this may require some insight into \TeX-programming. Here are two examples:
% \begin{enumerate}
%     \item A new fixed point display style: The following code defines a new style named `\texttt{my own fixed point style}' which uses $1{\cdot}00$ instead of $1.00$.
% \begin{lstlisting}
% \def\myfixedpointstyleimpl#1.#2\relax{%
%     #1{\cdot}#2%
% }%
% \def\myfixedpointstyle#1{%
%     \pgfutilensuremath{%
%     \ifpgfmathfloatroundhasperiod
%         \expandafter\myfixedpointstyleimpl#1\relax
%     \else
%         #1%
%     \fi
%     }%
% }
% \pgfkeys{/my own fixed point style/.code={%
%     \let\pgfmathprintnumber@fixed@style=\myfixedpointstyle}
% }%
% \end{lstlisting}
%     You only need to overwrite the macro \lstinline!\pgfmathprintnumber@fixed@style!. This macro takes one argument (the result of any numerical computations). The \TeX-boolean \lstinline!\ifpgfmathfloatroundhasperiod! is true if and only if the input number contains a period.
%
%     \item An example for a new scientific display style:
% \begin{lstlisting}
% % #1:
% %         0 == '0' (the number is +- 0.0),
% %         1 == '+',
% %         2 == '-',
% %         3 == 'not a number'
% %         4 == '+ infinity'
% %         5 == '- infinity'
% % #2: the mantissa
% % #3: the exponent
% \def\myscistyle#1#2e#3\relax{%
%     ...
% }
% \pgfkeys{/my own sci style/.code={%
%     \let\pgfmathfloatrounddisplaystyle=\myscistyle},
% }%
% \end{lstlisting}
% \end{enumerate}
%--------------------------------------------------

% Copyright 2018 by Till Tantau
%
% This file may be distributed and/or modified
%
% 1. under the LaTeX Project Public License and/or
% 2. under the GNU Free Documentation License.
%
% See the file doc/generic/pgf/licenses/LICENSE for more details.


\section{Object-Oriented Programming}
\label{section-oop}

This section describes the |oo| module.

\begin{pgfmodule}{oo}
    This module defines a relatively small set of \TeX\ commands for defining
    classes, methods, attributes and objects in the sense of object-oriented
    programming.
\end{pgfmodule}

In this chapter it is assumed that you are familiar with the basics of a
typical object-oriented programming language like Java, C++ or Eiffel.


\subsection{Overview}

\TeX\ does not support object-oriented programming, presumably because it was
written at a time when this style of programming was not yet ``en vogue''. When
one is used to the object-oriented style of thinking, some programming
constructs in \TeX\ often seem overly complicated. The object-oriented
programming module of \pgfname\ may help here. It is written completely using
simple \TeX\ macros and is, thus, perfectly portable. This also means, however,
that it is not particularly fast (but not too slow either), so you should use
it only for non-time-critical things.

Basically, the oo-system supports \emph{classes} (in the object-oriented sense,
this has nothing to do with \LaTeX-classes), \emph{methods},
\emph{constructors}, \emph{attributes}, \emph{objects}, \emph{object
identities}, and (thanks to Sa\v so \v Zivanovi\'c) \emph{inheritance} and
\emph{overloading.}

The first step is to define a class, using the macro |\pgfooclass| (all normal
macros in \pgfname's object-oriented system start with |\pgfoo|). This macro
gets the name of a class and in its body a number of \emph{methods} are
defined. These are defined using the |\method| macro (which is defined only
inside such a class definition) and they look a bit like method definitions in,
say, Java. Object attributes are declared using the |\attribute| command, which
is also defined only inside a class definition.

Once a class has been defined, you can create objects of this class. Objects
are created using |\pgfoonew|. Such an object has many characteristics of
objects in a normal object-oriented programming language: Each object has a
\emph{unique identity}, so when you create another object, this object is
completely distinct from all other objects. Each object also has a set of
private attributes, which may change over time. Suppose, for instance, that we
have a |point| class. Then creating a new object (called an instance) of this
class would typically have an |x|-attribute and a |y|-attribute. These can be
changed over time. Creating another instance of the |point| class creates
another object with its own |x|- and |y|-attributes.

Given an object, you can call a method for this object. Inside the method the
attributes of the object for which the method is being called can be accessed.

The life of an object always ends with the end of the \TeX\ scope in which it
was created. However, changes to attribute values are not local to scopes, so
when you change an attribute anywhere, this change persists till the end of the
life of the object or until the attribute is changed again.


\subsection{A Running Example: The Stamp Class}

As a running example we will develop a |stamp| class and |stamp| objects. The
idea is that a stamp object is able to ``stamp something'' on a picture. This
means that a stamp object has an attribute storing the ``stamp text'' and there
is a method that asks the object to place this text somewhere on a canvas. The
method can be called repeatedly and there can be several different stamp
objects, each producing a different text. Stamp objects can either be created
dynamically when needed or a library might define many such objects in an outer
scope.

Such stamps are similar to many things present in \pgfname\ such as arrow tips,
patterns, or shadings and, indeed, these could all have been implemented in
this object-oriented fashion (which might have been better, but the
object-oriented subsystem is a fairly new addition to \pgfname).


\subsection{Classes}

We start with the definition of the |stamp| class. This is done using the
|\pgfooclass| macro:

\begin{command}{\pgfooclass\opt{|(|\meta{list of superclasses}|)|}\marg{class name}\marg{body}}
    This command defines a class named \meta{class name}. The name of the class
    can contain spaces and most other characters, but no periods. So, valid
    class names are |MyClass| or |my class| or |Class_C++_emulation??1|. The
    \meta{list of superclasses} is optional just like the parenthesis around
    it.

    The \meta{body} is actually just executed, so any normal \TeX-code is
    permissible here. However, while the \meta{body} is being executed, the
    macros |\method| and |\attribute| are set up so that they can be used to
    define methods and attributes for this class (the original meanings are
    restored afterward).

    The definition of a class is local to the scope where the class has been
    defined.
    %
\begin{codeexample}[code only]
\pgfooclass{stamp}{
  % This is the class stamp

  \attribute text;
  \attribute rotation angle=20;

  \method stamp(#1) { % The constructor
    ...
  }

  \method apply(#1,#2) { % Causes the stamp to be shown at coordinate (#1,#2)
    ...
  }
}

% We can now create objects of type "stamp"
\end{codeexample}

    Concerning the list of base classes, the Method Resolution Order
    (\textsc{mro}) is computed using the C3 algorithm also used in Python, v2.3
    and higher. The linearization computed by the algorithm respects both
    local precedence ordering and monotonicity. Resolution of both methods and
    attributes depends on the \textsc{mro}: when a method method name is called
    on an object of class $C$, the system invokes method method name from the
    first class in the \textsc{mro} of $C$ which defines method method name;
    when an object is created, each attribute |attr| is initialized to the
    value specified in the first class in the \textsc{mro} of $C$ which
    declares attribute |attr|.
\end{command}

The \meta{body} of a class usually just consists of calls to the macros
|\attribute| and |\method|, which will be discussed in more detail in later
sections.


\subsection{Objects}

Once a class has been declared, we can start creating objects for this class.
For this the |\pgfoonew| command can be used, which has a peculiar syntax:

\begin{command}{\pgfoonew\opt{\meta{object handle or attribute}|=|}|new |\meta{class name}|(|\meta{constructor arguments}|)|}
    Causes a new object to be created. The class of the object will be
    \meta{class name}, which must previously have been declared using
    |\pgfooclass|. Once the object has been created, the constructor method of
    the object will be called with the parameter list set to \meta{constructor
    arguments}.

    The resulting object is stored internally and its lifetime will end exactly
    at the end of the current scope.

    Here is an example in which three stamp objects are created.
    %
\begin{codeexample}[code only]
\pgfoonew \firststamp=new stamp()
\pgfoonew \secondstamp=new stamp()
{
  \pgfoonew \thirdstamp=new stamp()
  ...
}
% \thirdstamp no longer exists, but \firststamp and \secondstamp do
% even if you try to store \thirdstamp in a global variable, trying
% to access it will result in an error.
\end{codeexample}

    The optional \meta{object handle or attribute} can either be an
    \meta{object handle} or an \meta{attribute}. When an \meta{object handle}
    is given, it must be a normal \TeX\ macro name that will ``point'' to the
    object (handles are discussed in more detail in
    Section~\ref{section-identities}). You can use this macro to call methods
    of the object as discussed in the following section. When an
    \meta{attribute} is given, it must be given in curly braces (the curly
    braces are used to detect the presence of an attribute). In this case, a
    handle to the newly created object is stored in this attribute.
    %
\begin{codeexample}[code only]
\pgfooclass{foo}
{
  \attribute stamp obj;
  \attribute another object;

  \method foo() {
    \pgfoonew{stamp obj}=new stamp()
    \pgfoonew{another object}=new bar()
  }
  ...
}
\end{codeexample}
    %
\end{command}

\begin{command}{\pgfoogc}
    This command causes the ``garbage collector'' to be invoked. The job of
    this garbage collector is to free the global \TeX-macros that are used by
    ``dead'' objects (objects whose life-time has ended). This macro is called
    automatically after every scope in which an object has been created, so you
    normally do not need to call this macro yourself.
\end{command}


\subsection{Methods}

Methods are defined inside the body of classes using the following command:

\begin{command}{\method \meta{method name}|(|\meta{parameter list}|)|\marg{method body}}
    This macro, which is only defined inside a class definition, defines a new
    method named \meta{method name}. Just like class names, method names can
    contain spaces and other characters, so \meta{method names} like
    |put_stamp_here| or |put stamp here| are both legal.

    Three method names are special: First, a method having either the same name
    as the class or having the name |init| is called the \emph{constructor} of
    the class. There are (currently) no destructors; objects simply become
    ``undefined'' at the end of the scope in which they have been created. The
    other two methods are called |get id| and |get handle|, which are always
    automatically defined and which you cannot redefine. They are discussed in
    Section~\ref{section-identities}.

    Overloading of methods by differing numbers of parameters is not possible,
    that is, it is illegal to have two methods inside a single class with the
    same name (despite possibly different parameter lists). However, two
    different classes may contain a method with the same name, that is, classes
    form namespaces for methods. Also, a class can (re)implement a method from
    a superclass.

    The \meta{method name} must be followed by a \meta{parameter list} in
    parentheses, which must be present even when the \meta{parameter list} is
    empty. The \meta{parameter list} is actually a normal \TeX\ parameter list
    that will be matched against the parameters inside the parentheses upon
    method invocation and, thus, could be something like |#1#2 foo #3 bar.|,
    but a list like |#1,#2,#3| is more customary. By setting the parameter list
    to just |#1| and then calling, say, |\pgfkeys{#1}| at the beginning of a
    method, you can implement Objective-C-like named parameters.

    When a method is called, the \meta{body} of the method will be executed.
    The main difference to a normal macro is that while the \meta{body} is
    executed, a special macro called |\pgfoothis| is set up in such a way that
    it references the object for which the method is executed.
\end{command}

In order to call a method for an object, you first need to create the object
and you need a handle for this object. In order to invoke a method for this
object, a special syntax is used that is similar to Java or C++ syntax:

\begin{pgfmanualentry}
    \pgfmanualentryheadline{\meta{object handle}\opt{|.|\meta{super class}}|.|\meta{method name}|(|\meta{parameters}|)|}%
    \pgfmanualbody
    This causes the method \meta{method name} to be called for the object
    referenced by the \meta{object handle}. The method is the one defined in
    the class of the object or, if it is not defined there, the method defined
    in the superclasses of the object's class (if there are several
    superclasses that define the same method, the method resolution order is
    used to determine which one gets called). If the optional \meta{super
    class} is specified, the method implementation of that class will be used
    rather than the implementation in the object's class. The \meta{parameters}
    are matched against the parameters of the method and, then, the method body
    is executed. The execution of the method body is \emph{not} done inside a
    scope, so the effects of a method body persist.
    %
\begin{codeexample}[code only]
\pgfooclass{stamp}{
  % This is the class stamp

  \method stamp() { % The constructor
  }

  \method apply(#1,#2) { % Causes the stamp to be shown at coordinate (#1,#2)
    % Draw the stamp:
    \node [rotate=20,font=\huge] at (#1,#2) {Passed};
  }
}

\pgfoonew \mystamp=new stamp()

\begin{tikzpicture}
  \mystamp.apply(1,2)
  \mystamp.apply(3,4)
\end{tikzpicture}
\end{codeexample}

    Inside a method, you can call other methods. If you have a handle for
    another object, you can simply call it in the manner described above. In
    order to call a method of the current object, you can use the special
    object handle |\pgfoothis|.

    \begin{command}{\pgfoothis}
        This object handle is well-defined only when a method is being executed.
        There, it is then set to point to the object for which the method is
        being called, which allows you to call another method for the same
        object.
        %
\begin{codeexample}[code only]
\pgfooclass{stamp}{
  % This is the class stamp

  \method stamp() {}

  \method apply(#1,#2) {
    \pgfoothis.shift origin(#1,#2)

    % Draw the stamp:
    \node [rotate=20,font=\huge] {Passed};
  }

  % Private method:
  \method shift origin(#1,#2) {
    \tikzset{xshift=#1,yshift=#2}
  }
}
\end{codeexample}
    \end{command}
\end{pgfmanualentry}

\begin{command}{\pgfoosuper|(|\meta{class},\meta{object handle}|).|\meta{method name}|(|\meta{arguments}|)|}
    This macro gives you finer control over which method gets invoked in case
    of multiple inheritance. This macro calls \meta{method name} of the object
    specified by \meta{object handle}, but which implementation of the method
    is called is determined as follows: it will be the implementation in the
    first class (in the method resolution order) after \meta{class} that
    defines \meta{method name}.
\end{command}


\subsection{Attributes}

Every object has a set of attributes, which may change over time. Attributes
are declared using the |\attribute| command, which, like the |\method| command,
is defined only inside the scope of |\pgfooclass|. Attributes can be modified
(only) by methods. To take the |stamp| example, an attribute of a |stamp|
object might be the text that should be stamped when the |apply| method is
called.

When an attribute is changed, this change is \emph{not} local to the current
\TeX\ group. Changes will persist till the end of the object's life or until
the attribute is changed once more.

To declare an attribute you should use the |\attribute| command:
%
\begin{command}{\attribute \meta{attribute name}\opt{|=|\meta{initial value}}|;|}
    This command can only be given inside the body of an |\pgfooclass| command.
    It declares the attribute named \meta{attribute name}. This name, like
    method or class names, can be quite arbitrary, but should not contain
    periods. Valid names are |an_attribute?| or |my attribute|.

    You can optionally specify an \meta{initial value} for the attribute; if
    none is given, the empty string is used automatically. The initial value is
    the value that the attribute will have just after the object has been
    created and before the constructor is called.
    %
\begin{codeexample}[code only]
\pgfooclass{stamp}{
  % This is the class stamp

  \attribute text;
  \attribute rotation angle = 20;

  \method stamp(#1) {
    \pgfooset{text}{#1} % Set the text
  }

  \method apply(#1,#2) {
    \pgfoothis.shift origin(#1,#2)

    % Draw the stamp:
    \node [rotate=\pgfoovalueof{rotation angle},font=\huge]
      {\pgfoovalueof{text}};
  }

  \method shift origin(#1,#2) { ... }

  \method set rotation (#1) {
    \pgfooset{rotation angle}{#1}
  }
}
\end{codeexample}
    %
\end{command}

Attributes can be set and read only inside methods, it is not possible to do so
using an object handle. Spoken in terms of traditional object-oriented
programming, attributes are always private. You need to define getter and
setter methods if you wish to read or modify attributes.

Reading and writing attributes is not done using the ``dot-notation'' that is
used for method calls. This is mostly due to efficiency reasons. Instead, a set
of special macros is used, all of which can \emph{only be used inside methods}.

\begin{command}{\pgfooset\marg{attribute}\marg{value}}
    Sets the \meta{attribute} of the current object to \meta{value}.
    %
\begin{codeexample}[code only]
\method set rotation (#1) {
  \pgfooset{rotation angle}{#1}
}
\end{codeexample}
    %
\end{command}

\begin{command}{\pgfooeset\marg{attribute}\marg{value}}
    Performs the same action as |\pgfooset| but in an |\edef| full expansion
    context.
\end{command}

\begin{command}{\pgfooappend\marg{attribute}\marg{value}}
    This method adds the given \meta{value} to the \meta{attribute} at the end.
\end{command}

\begin{command}{\pgfooprefix\marg{attribute}\marg{value}}
    This method adds the given \meta{value} to the \meta{attribute} at the
    beginning.
\end{command}

\begin{command}{\pgfoolet\marg{attribute}\marg{macro}}
    Sets the \meta{attribute} of the current value to the current value of
    \meta{macro} using \TeX's |\let| command.
    %
\begin{codeexample}[code only]
\method foo () {
  \pgfoolet{my func}\myfunc
  % Changing \myfunc now has no effect on the value of attribute my func
}
\end{codeexample}
    %
\end{command}

\begin{command}{\pgfoovalueof\marg{attribute}}
    Expands (eventually) to the current value of \meta{attribute} of the
    current object.
    %
\begin{codeexample}[code only]
\method apply(#1,#2) {
  \pgfoothis.shift origin(#1,#2)

  \node [rotate=\pgfoovalueof{rotation angle},font=\huge]
    {\pgfoovalueof{text}};
}
\end{codeexample}
    %
\end{command}

\begin{command}{\pgfooget\marg{attribute}\marg{macro}}
    Reads the current value of \meta{attribute} and stores the result in
    \meta{macro}.
    %
\begin{codeexample}[code only]
...
  \method get rotation (#1) {
    \pgfooget{rotation angle}{#1}
  }
...

\mystamp.get rotation(\therotation)
``\therotation'' is now ``20'' (or whatever).
\end{codeexample}
    %
\end{command}


\subsection{Identities}
\label{section-identities}

Every object has a unique identity, which is simply an integer. It is possible
to retrieve the object id using the |get id| method (discussed below), but
normally you will not need to do so because the id itself cannot be used to
access an object. Rather, you access objects via their methods and these, in
turn, can only be called via object handles.

Object handles can be created in four ways:
%
\begin{enumerate}
    \item Calling |\pgfoonew|\meta{object handle}|=...| will cause \meta{object
        handle} to be a handle to the newly created object.
    \item Using |\let| to create an alias of an existing object handle: If
        |\mystamp| is a handle, saying |\let\myotherstamp=\mystamp| creates a
        second handle to the same object.
    \item |\pgfooobj|\marg{id} can be used as an object handle to the object
        with the given \meta{id}.
    \item Using the |get handle| method to create a handle to a given object.
\end{enumerate}

Let us have a look at the last two methods.

\begin{command}{\pgfooobj\marg{id}}
    Provided that \meta{id} is the id of an existing object (an object whose
    life-time has not expired), calling this command yields a handle to this
    object. The handle can then be used to call methods:
    %
\begin{codeexample}[code only]
% Create a new object:
\pgfoonew \mystamp=new stamp()

% Get the object's id and store it in \myid:
\mystamp.get id(\myid)

% The following two calls have the same effect:
\mystamp.apply(1,1)
\pgfooobj{\myid}.apply(1,1)
\end{codeexample}
    %
\end{command}

The |get id| method can be used to retrieve the id of an object. This method is
predefined for every class and you should not try to define a method of this
name yourself.

\begin{predefinedmethod}{get id(\meta{macro})}
    Calling \meta{obj}|.get id(|\meta{macro}|)|  stores the id \meta{obj} in
    \meta{macro}. This is mainly useful when you wish to store an object for a
    longer time and you cannot guarantee that any handle that you happen to
    have for this object will be available later on.

    The only way to use the retrieved id later on is to call |\pgfooobj|.

    Different object that are alive (that are still within the scope in which
    they were created) will always have different ids, so you can use the id to
    test for equality of objects. However, after an object has been destroyed
    because its scope has ended, the same id may be used again for newly
    created objects.

    Here is a typical application where you need to call this method: You wish
    to collect a list of objects for which you wish to call a specific method
    from time to time. For the collection process you wish to offer a macro
    called |\addtoobjectlist|, which takes an object handle as parameter. It is
    quite easy to store this handle somewhere, but a handle is, well, just a
    handle. Typically, shortly after the call to |\addtoobjectlist| the handle
    will no longer be valid or even exist, even though the object still exists.
    In this case, you wish to store the object id somewhere instead of the
    handle. Thus, for the object passed to |\addtoobjectlist| you call the
    |get id| method and store the resulting id, rather than the handle.
\end{predefinedmethod}

There is a second predefined method, called |get handle|, which is also used to
create object handles.

\begin{predefinedmethod}{get handle(\marg{macro name})}
    Calling this method for an object will cause \meta{macro name} to become a
    handle to the given object. For any object handle |\obj| -- other than
    |\pgfoothis| -- the following two have the same effect:
    %
    \begin{enumerate}
        \item |\let|\meta{macro name}|=\obj|
        \item |\obj.get handle(|\meta{macro name}|)|
    \end{enumerate}

    The first method is simpler and faster. However, for |\pgfoothis| there is
    a difference: The call |\pgfoothis.get handle(|\meta{macro name}|)| will
    cause \meta{macro name} to be an object handle to the current object and
    will persist to be so even after the method is done. By comparison,
    |\let|\meta{macro name}|=\pgfoothis| causes |\obj| to be the same as the
    very special macro |\pgfoothis|, so |\obj| will always refer to the current
    object, which may change over time.
\end{predefinedmethod}


\subsection{The Object Class}
\label{section-object}

The object-oriented module predefines a basic class |object| that can be used
as a base class in different context.

\begin{ooclass}{object}
    This class current only implements one method:

    \begin{method}{copy(\meta{handle})}
        Creates a new object and initializes the values of its (declared)
        attributes to the values of the original. The method takes one
        argument: a control sequence which receives the handle of the copy.
    \end{method}
\end{ooclass}


\subsection{The Signal Class}
\label{section-signals}

In addition to the basic mechanism for defining and using classes and object,
the class |signal| is predefined. It implements a so-called signal--slot
mechanism.

\begin{ooclass}{signal}
    This class is used to implement a simple signal--slot mechanism. The idea
    is the following: From time to time special things happen about which a
    number of objects need to be informed. Different things can happen and
    different object will be interested in these things. A |signal| object can
    be used to signal that such special things of a certain kind have happened.
    For example, one signal object might be used to signal the event that ``a
    page has been shipped out''. Another signal might be used to signal that
    ``a figure is about to be typeset'', and so on.

    Objects can ``tune in'' to signals. They do so by \emph{connecting} one of
    their methods (then called a \emph{slot}) to the signal. Then, whenever the
    signal is \emph{emitted}, the method of the connected object(s) get called.
    Different objects can connect different slots to the same signal as long as
    the argument lists will fit. For example, the object that is used to signal
    the ``end of page has been reached'' might emit signals that have, say, the
    box number in which the finished page can be found as a parameter
    (actually, the finished page is always in box 255). Then one object could
    connect a method |handle page(#1)| to this signal, another might connect
    the method |emergency action(#1)| to this signal, and so on.

    Currently, it is not possible to ``unregister'' or ``detach'' a slot from a
    signal, that is, once an object has been connect to a signal, it will
    continue to receive emissions of this signal till the end of the life-time
    of the signal. This is even true when the object no longer exists (but the
    signal does), so care must be taken that signal objects are always created
    after the objects that are listening to them.

    \begin{method}{signal()}
        The constructor does nothing.
    \end{method}

    \begin{method}{connect(\meta{object handle},\meta{method name})}
        This method gets an \meta{object handle} as parameter and a
        \meta{method name} of this object. It will queue the object-method pair
        in an internal list and each time the signal emits something, this
        object's method is called.

        Be careful not to pass |\pgfoothis| as \meta{object handle}. This would
        cause the signal object to connect to itself. Rather, if you wish to
        connect a signal to a method of the current object you first need to
        create an alias using the |get handle| method:
        %
\begin{codeexample}[code only]
\pgfooclass{some class}{
  \method some class() {
    \pgfoothis.get handle(\me)
    \somesignal.connect(\me,foo)
    \anothersignal.connect(\me,bar)
  }
  \method foo () {}
  \method bar (#1,#2) {}
}
\pgfoonew \objA=new some class()
\pgfoonew \objB=new some class()
\end{codeexample}
    \end{method}

    \begin{method}{emit(\meta{arguments})}
        This method emits a signal to all connected slots. This means that for
        all objects that have previously been connected via a call of
        |connect|, the method (slot) that was specified during the call of
        |connect| is invoked with given \meta{arguments}.
        %
\begin{codeexample}[code only]
\anothersignal.emit(1,2)
% will call \objA.bar(1,2) and \objB.bar(1,2)
\end{codeexample}
    \end{method}
\end{ooclass}


\subsection{Implementation Notes}

For the curious, here are some notes on how the oo-system is implemented:
%
\begin{itemize}
    \item There is an object id counter that gets incremented each time an
        object is created. However, this counter is local to the current scope,
        which means that it is reset at the end of each scope, corresponding to
        the fact that at the end of a scope all objects created in this scope
        become invalid. Newly created objects will then have the same id as
        ``deleted'' objects.
    \item Attributes are stored globally. For each attribute of each object
        there is a macro whose name is composed of the object's id and the
        attribute name. Changes to object attributes are always global.
    \item A call to the garbage collector causes a loop to be executed that
        tries to find objects whose object number is larger than the current
        maximum alive objects. The global attributes of these objects are then
        freed (set to |\relax|) by calling a special internal method of these
        (dead) objects.

        The garbage collector is automatically called after each group in
        which an object was created using |\aftergroup|.
    \item When a method is called, before the method call some code is executed
        that sets a global counter storing the current object id to the object
        id of the object being called. After the method call some code is
        inserted that restores the global counter to its original value. This
        is done without scopes, so some tricky |\expandafter| magic is needed.
        Note that, because of this process, you cannot use commands like
        |\pgfutil@ifnextchar| at the end of a method.
    \item An object handle contains just the code to set up and restore the
        current object number to the number of the object being called.
\end{itemize}
